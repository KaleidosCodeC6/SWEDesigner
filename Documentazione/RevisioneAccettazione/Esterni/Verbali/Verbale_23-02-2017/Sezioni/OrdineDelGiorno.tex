\documentclass[../AnalisiDeiRequisiti.tex]{subfiles}
\begin{document}
	\section*{Ordine del giorno}
		\begin{enumerate}
			\item Qualità e complessità codice prodotto;
			\item Dominio trattato;
			\item Riusabilità;
			\item Ruolo dell'UML;
			\item Preferenze del committente;
			\item Suggerimenti per progettare la soluzione.
		\end{enumerate}
		Di seguito, il riassunto delle discussioni fatte.
		\begin{enumerate}
			\item Per il codice prodotto dal diagramma delle classi, la traduzione è
			banale e consiste nello scheletro delle classi insieme all'intestazione
			dei metodi; non ha senso parlare di complessità e qualità del codice.\\
			Per il diagramma delle attività, invece, la qualità del codice prodotto
			dipende in gran parte dalla completezza e grado di specializzazione
			(inteso come grado di raffinamento dell'attività rappresentata) del
			diagramma tradotto, oltre ovviamente alla qualità della logica di
			traduzione.
			\item Le realtà che si possono modellare con i diagrammi prodotti dal
			disegnatore devono fare parte di un dominio specifico; sarebbe
			impensabile produrre software, sufficientemente completo, dalla
			traduzione di diagrammi che modellano qualsiasi realtà.\\
			E' gradito il dominio dei giochi da tavolo.\\
			E' necessaria un'analisi approfondita del dominio scelto.
			\item È importante massimizzare la riusabilità dei componenti, per
			poterli sfruttare nel maggior numero di realtà modellate dai diagrammi
			(le quali apparterranno ad un unico dominio).\\
			Consigliato creare template: il focus su un dominio permette di creare
			modelli che si adattano a molti casi specifici di tale dominio.\\
			È stato esemplificato il dominio dei giochi da tavolo: molti giochi
			prevedono una scacchiera; bisogna, allora, prevedere un metodo che
			disegni la scacchiera, sarà poi il gioco specifico che deciderà come
			gestirla (ad esempio, dama userà tutte le caselle, Monopoli userà solo
			le caselle del contorno).
			\item Nei progetti reali, l'UML ha un'utilità solo nel breve termine;
			un obiettivo è quello di renderlo importante per tutta la durata del
			progetto e non una rappresentazione iniziale che diventa inutilizzabile
			o addirittura inutile.
			A tal senso viene lasciata libertà di essere irriverenti con l'UML,
			pensando a soluzioni anche non convenzionali rispetto allo standard.
			\item È gradito lo sviluppo del prodotto attraverso le tecnologie web,
			ma è anche data la possibilità di realizzarlo per desktop.\\
			Il server può essere locale, ma è lasciata la libertà di acquistare
			spazi web. In entrambi i casi bisogna simulare il costo dell'acquisto.\\
			Consigliato concentrarsi su uno specifico dominio.
			\item È preferibile implementare l'applicazione lato server in Java, in
			quanto facilita il test (compilatore) e non bisogna simulare
			l'ereditarietà.\\
			Il problema principale della traduzione da UML a codice è dato dalla
			mancanza di continuità tra i vari diagrammi. Valutare la possibilità di
			creare diagrammi ``ibridi'', che abbiano elementi sia di struttura che
			di comportamento; in questo modo si risparmierebbe l'integrazione di
			codici derivati da diagrammi diversi.\\
			È gradita la possibilità di tenere aggiornato il diagramma creato, in
			seguito a modifiche del codice precedentemente prodotto.\\
			Le classi e le attività del software sono spesso molte e complesse per
			essere tracciate con chiarezza nei diagrammi; bisogna perciò prevedere
			più viste di uno stesso diagramma attraverso livelli di dettaglio: layer
			per i diagrammi delle classi, innesti per i diagrammi delle
			attività, zoom.	
		\end{enumerate}
\end{document}
