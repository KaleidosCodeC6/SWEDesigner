\documentclass[../AnalisiDeiRequisiti.tex]{subfiles}
\begin{document}
	\section*{Ordine del giorno}
		\begin{enumerate}
			\item Chiarimento riguardo lo scopo del progetto e descrizione dell'idea di editor
			del gruppo;
			\item Verificare se l'idea di editor del gruppo è valida per il proponente;
			\item Chiarimento sul come trattare il linguaggio UML nell'editor;
			\item Chiedere se bisogna pensare alla possibilità di editare codice multithread;
			\item Chiarimento sul possibile aggiornamento del diagramma in seguito a modifiche
			al codice precedentemente generato;
			\item Chiedere consiglio su metodo di salvataggio da adottare se si decidesse di
			dare la possibilità all'utente di salvare i propri personali design pattern;
			\item Chiarimento riguardo la possibile realizzazione del gioco da tavolo "Hex".
		\end{enumerate}
		Di seguito, il riassunto delle discussioni fatte.
		\begin{enumerate}
		\item
		\kaleidoscode: Vorremmo chiarimenti sul focus principale del progetto: non siamo
		sicuri se dobbiamo realizzare un software incentrato sullo sviluppo Java/Javascript
		mediante il diagramma UML o un editor UML che in più produce codice. In alcuni casi
		infatti dobbiamo scegliere se essere fedeli ai vincoli che UML impone o se
		"valicarli" per un maggior livello di completezza del codice.
		
		\proponente: La cosa che vogliamo controllare è la generazione del codice e per fare
		ciò può essere necessario modificare leggermente il diagramma UML.
		Ad esempio mettiamo caso che voi vogliate utilizzare un editor esterno per la
		generazione dei diagrammi, come ad esempio ArgoUML, che mi salva l'xmi, poi
		leggendo l'xmi voi generate il codice. Potrebbe andare benissimo così, ma c'è un
		problema. Se dovete inventare un diagramma nuovo, non c'è lì dentro e allora bisogna
		capire come risolvere la situazione.\\
		Avete quindi due opzioni: la prima è usare un disegnatore UML standard e richiedere
		che venga usato sotto certe condizioni. Allora quello che dovete fare in questo
		caso è controllare che venga usato secondo le vostre direttive e semmai segnalare
		un eventuale errore. Quindi potete mettere le restrizioni che volete all'UML a
		patto che il vostro controllore le riconosca e gestisca correttamente. Se invece
		intendete fare voi stessi il disegnatore avete più libertà sotto questo punto di
		vista.\\
		Voi come pensavate di fare?
		
		\kaleidoscode: Per ora non ci siamo addentrati ancora nella progettazione, ma come
		idea di massima pensavamo di realizzare il disegnatore organizzandolo a livelli:
		il livello più esterno è un diagramma dei package. Clickando sui singoli elementi
		si passa a un diagramma delle classi, poi a una visualizzazione più dettagliata di
		una singola classe. Qui ogni metodo avrà associato il corrispondente diagramma
		delle attività. All'interno del diagramma delle attività pensavamo di specificare
		ulteriormente ogni attività con una sorta di diagramma di flusso modificato ad hoc.
		L'idea è quella di mettere a disposizione una serie di bolle organizzate in
		opportune librerie. Le bolle sono le unità di base del diagramma di flusso e sono
		già tradotte in codice. In questo modo, combinandole tra loro, il codice che viene
		generato dovrebbe verosimilmente essere privo di "buchi". Quindi fissando un
		dominio, che può essere, come da capitolato, quello dei giochi da tavolo, pensavamo di
		mettere a disposizione un insieme completo di bolle per essere in grado di garantire
		che, utilizzandole, l'editor possa tradurre il diagramma in codice senza errori.\\
		Qualora l'utente voglia in qualche modo estendere le possibilità offerte dalle bolle
		pre-realizzate, pensavamo di considerare un bolla personalizzabile in cui l'utente
		stesso può inserire il proprio codice, ma in questo caso verrebbe notificata la mancanza
		di sicurezza nella sua "buona" generazione a partire dal diagramma.
		
		\proponente: Va bene. A questo proposito, quello che stanno notando anche gli altri
		gruppi è che, visto il dominio relativamente ristretto, sono presenti molti pattern.
		Hanno pensato quindi di sfruttare e di far utilizzare i design pattern nel loro
		editor UML.
		Provate a considerare l'idea, magari potrebbe integrarsi bene con quello che avete
		già pensato. Tenete conto però che, aggiungendo anche i pattern, il diagramma
		UML potrebbe diventare un po' farraginoso e difficile da leggere se fate
		tante classi che magari usano tanti pattern.

		\kaleidoscode: Ma in che modo dobbiamo utilizzare questi pattern nel momento di
		passare a codice?

		\proponente: Quando si sta in un dominio così formato si vengono a creare delle
		soluzioni. Queste soluzioni alle volte si riescono a riciclare come ereditarietà,
		alle volte come altro, fatto sta che capita saltino fuori pattern che sono
		l'unione di più pezzi che collaborano tra loro. Soprattutto c'è anche un altro
		aspetto: una stessa classe può partecipare a più pattern. Quindi la cosa
		interessante è che una stessa classe può "giocare in più ruoli". In ogni caso è
		solo una possibilità che vi presento, se volete seguirla bene, altrimenti va bene
		lo stesso.

		\item
		\kaleidoscode: Potremmo pensare di implementare questa idea magari con un sistema di
		etichette, magari colorate, per dare un'idea di questi pattern.
		In ogni caso volevamo un feedback per capire se stiamo andando verso la giusta
		direzione o se abbiamo interpretato male qualcosa.
		
		\proponente: No, state procedendo correttamente.
		
		\item
		\kaleidoscode: Una discussione che era nata tra di noi riguardava il fatto che
		trattandosi di UML magari il professor Vardanega tende ad essere rigido riguardo
		l'aderenza agli standard, mentre da quanto abbiamo capito dal primo incontro è
		apprezzata un po' di "irriverenza" nei confronti dell'UML. 

		\proponente: È giusto. Il discorso è che se voi modificate l'UML con "arroganza",
		ovvero senza motivare la vostra scelta allora il professor Vardanega avrebbe tutti
		i motivi di contestarvi. Se vi inventate qualcosa di nuovo deve essere fortemente
		giustificato e deve avere una logica alla base. In quel caso sono sicuro che anche
		per lui non ci siano problemi. Comunque ricordatevi che non dovete essere "cattivi"
		con lo standard. Dovete semplicemente utilizzarlo come vi fa comodo.

		\item
		\kaleidoscode: Dovremo occuparci di trattare anche codice multithread?

		\proponente: No, senza dubbio no. C'è già tanto da dire così, l'importante è che sia
		chiaro fin dall'inizio. In ogni caso non lo richiedo assolutamente.
	
		\item
		\kaleidoscode: Un altro aspetto del quale non siamo sicuri riguarda l'aggiornamento
		del diagramma in seguito a modifiche al codice precedentemente generato.

		\proponente: No no no, non provate a farlo. Cioè se ci riuscite bene, ma guardate che
		è difficilissimo.

		\kaleidoscode: Il dubbio ci è venuto perché nel capitolato viene richiesto un
		aggiornamento del diagramma in seguito a "piccole modifiche al codice".
		
		\proponente: Si, ma visto che voi avete parlato di bolle nel caso in cui per esempio
		innestiate delle bolle l'una dentro l'altra la complessità di controllo diventa
		molto alta; ad esempio, diventa difficile per un "controllore" capire, se innestate una
		bolla if dentro un altro if, a quale if ci si riferisce.

		\kaleidoscode: L'idea più che altro derivava dal fatto che nel momento di aggiungere
		un elemento al diagramma pensavamo di creare un oggetto nascosto che contenesse le
		informazioni necessarie a popolare sia il diagramma stesso che il codice. Da qui
		l'idea, magari mediante colorazioni diverse, di rendere possibili piccole
		modifiche.
		
		\proponente: Allora dovete essere più liberali sulle bolle. Se invece volete essere
		più rigidi su quelle dovete rinunciare a questo aspetto.

		\item
		\kaleidoscode: Se decidiamo i dare la possibilità all'utente di
		salvare i propri personali design pattern potremmo considerare di gestirlo tramite
		cookies o saremmo costretti ad adottare altre soluzioni come ad esempio un
		salvataggio in remoto?

		\proponente: Un pattern avrebbe senza dubbio dimensioni troppo
		grandi per un cookie. Poi dipende dal formato che usate, ma è difficile che ci
		stia. Non ha senso usare cookies, piuttosto un local storage è
		molto meglio. Il cookie inoltre viene passato ogni volta che viene contattato
		il server. Se fosse di dimensioni generose, si genererebbe molto traffico.
		
		\item
		\kaleidoscode: Pensavamo di inserire tra gli obiettivi desiderabili lo sviluppo
		del gioco da tavolo "Hex": vorremmo sapere se la realizzazione di quest'ultimo va fatta
		esclusivamente con l'ausilio dell'applicativo che realizzeremo o se
		l'applicativo deve darci la "base" del codice che poi possiamo anche integrare
		manualmente per conto nostro.
		
		\proponente: State dicendo una cosa molto semplice, che mi sembra ragionevole:
		pensate che con il vostro applicativo riuscirete a coprire una certa
		"quota" dell'insieme. È possibilissimo, quindi va bene. Magari create, ad esempio,
		il pattern turno, scacchiera, ecc. e poi inserite una bolla dove scrivete il
		rimanente, poiché non siete ancora arrivati a fare uno strumento così raffinato.
		
		\kaleidoscode: Va bene. Grazie della disponibilità.
		
		\proponente: Bene, fatemi sapere se avete ulteriori dubbi.
		\end{enumerate}
\end{document}
