\documentclass[../PianoDiQualifica.tex]{subfiles}
\begin{document}
	\section{La strategia di gestione della qualità nel dettaglio}
		\subsection{Risorse}
			\subsubsection{Necessarie}
				Per la realizzazione del prodotto sono necessarie le risorse
				umane e tecnologiche elencate di seguito.
				\begin{itemize}
					\item \textbf{Risorse umane}: sono descritte
					dettagliatamente nel \pianodiprogettov:
					\begin{itemize}
						\item \responsabilediprogetto;
						\item \amministratore;
						\item \analista;
						\item \progettista;
						\item \programmatore;
						\item \verificatore.
					\end{itemize}
					\item \textbf{Risorse software}: sono descritte
					dettagliatamente nelle \normediprogettov. Si tratta di
					software che permettono:
					\begin{itemize}
						\item la comunicazione e la condivisione del lavoro
						tra gli elementi del team;
						\item la stesura della documentazione in
						formato \gl{\LaTeX};
						\item la creazione di diagrammi UML;
						\item la codifica nei linguaggi di programmazione scelti;
						\item la semplificazione delle attività di verifica;
						\item la gestione dei test sul codice.
					\end{itemize}
					\item \textbf{Risorse hardware}: ciascun componente del
					gruppo deve avere un computer con tutti i software necessari
					descritti nelle \normediprogettov. È necessario avere a
					disposizione almeno un luogo dove poter effettuare le
					riunioni interne.
				\end{itemize}
			\subsubsection{Disponibili}
				Ogni membro del team ha a disposizione uno o più computer
				personali dotati degli strumenti necessari.\\
				Le riunioni interne si svolgono presso le aule del dipartimento
				di Matematica dell'Università degli Studi di Padova.
\end{document}
