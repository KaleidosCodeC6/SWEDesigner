\documentclass[../PianoDiQualifica.tex]{subfiles}
\begin{document}
	\section{Capability Maturity Model}\label{app:CMM}
		Il Capability Maturity Model (CMM), divenuto CMMI (I per Integration), è un modello per il
		miglioramento dei processi di sviluppo del software il cui obiettivo è di aiutare
		un'organizzazione a migliorare le sue prestazioni in termini di qualità del software prodotto,
		produttività dell'organizzazione e riduzione dei tempi di sviluppo.\\
		Le lettere dell'acronimo indicano: 
		\begin{itemize}
			\item \textbf{Capability}: si indica la misura di quanto un singolo processo è adeguato
			allo scopo per il quale è stato definito e determina il range del risultato raggiungibile
			utilizzando	quel processo in termini di efficienza ed efficacia;
			\item \textbf{Maturity}: si indica la misura che comunica quanto è governato l'insieme di
			processi aziendali ed è influenzata dalle capability dei processi coinvolti;
			\item \textbf{Model}: si indica l'insieme dei requisiti che vanno ad essere sempre più
			stringenti per valutare il miglioramento dei processi aziendali;
			\item \textbf{Integration}: si indica l'architettura di integrazione delle diverse
			discipline (system, hardware, software) e tipologie di attività delle aziende.
		\end{itemize}
		\subsection{Struttura}
			Il modello coinvolge cinque aspetti:
			\begin{itemize}
				\item \textbf{Livelli di maturità}: un processo di continua maturità a cinque livelli
				dove il più alto indica lo stato ideale dove i processi sono sistematicamente gestiti
				da una combinazione di ottimizzazione e continuo miglioramento di processi;
				\item \textbf{Aree chiave di processo}: un'area chiave di processo identifica un
				insieme di attività che raggiungono obiettivi ritenuti importanti quando svolte
				assieme;
				\item \textbf{Obiettivi}: gli obiettivi di un'area chiave di processo indicano lo
				scopo, i confini e le intenzioni di quest'ultima;
				\item \textbf{Caratteristiche comuni}: includono pratiche che implementano e
				istituzionalizzano un'area chiave di processo; sono suddivise in cinque tipi: impegno
				nell'operare, abilità nell'operare, attività svolte, misurazioni ed analisi, verifica
				dell'implementazione;
				\item \textbf{Pratiche chiave}: descrivono gli elementi di infrastruttura e pratiche
				che contribuiscono maggiormente nell'implementazione ed istituzionalizzazione
				dell'area.
			\end{itemize}
		\subsection{Livelli}
			I livelli di maturità sopra citati sono i seguenti:
			\begin{enumerate}
				\item \textbf{Iniziale}: i processi sono imprevedibili, insufficientemente controllati
				e reattivi; risultano non documentati e perciò non sono ripetibili;
				\item \textbf{Ripetibile}: i processi sono sufficientemente documentati tanto da
				renderli ripetibili;
				\item \textbf{Definito}: i processi sono definiti, documentati e ripetibili;
				\item \textbf{Gestito}: i processi sono controllati e gestiti attraverso analisi e
				utilizzo di metriche concordate;
				\item \textbf{Efficiente}: la gestione dei processi punta al loro
				miglioramento/ottimizzazione.
			\end{enumerate}
	\section{Ciclo di Deming}\label{app:PDCA}
		Il ciclo di Deming o ciclo PDCA (Plan-Do-Check-Act) è un metodo di gestione iterativo
		focalizzato sul miglioramento continuo dei processi; l'acronimo "PDCA" definisce i passi
		in cui è diviso:
		\begin{itemize}
			\item Pianificare (\textbf{P}lan): definizione delle attività e dei processi da migliorare
			secondo misurazioni effettuate nonché scadenze e risorse utili al raggiungimento del
			miglioramento;
			\item Fare (\textbf{D}o): attuazione delle azioni pianificate al passo precedente con
			conseguente misurazione e raccolta di dati utili ai passi successivi;
			\item Verificare (\textbf{C}heck): studio dei risultati misurati e raccolti nel passo
			precedente e confronto con i risultati attesi;
			\item Agire (\textbf{A}ct): standardizzazione dei cambiamenti apportati nell'esecuzione
			di processo (solamente se l'esito del passo precedente è positivo).
		\end{itemize}
		Per poter applicare il ciclo PDCA è necessario che i processi siano documentati,
		analizzabili e ripetibili per poter individuare gli eventuali errori da correggere.
	\section{ISO/IEC 9126}\label{app:ISO9126}
		La normativa [ISO/IEC 9126] si suddivide in:
		\begin{itemize}
			\item Modello della qualità del software;
			\item Metriche per la qualità esterna;
			\item Metriche per la qualità interna;
			\item Metriche per la qualità in uso.
		\end{itemize}
		\subsection{Modello della qualità}
			\subsubsection{Qualità esterna ed interna}
				Il modello di qualità stabilito dallo standard è classificato da sei caratteristiche
				generali:
				\begin{itemize}
					\item \textbf{Funzionalità}: capacità del prodotto di fornire l'insieme di funzioni
					per soddisfare le richieste e gli obiettivi dell'utente;
					\item \textbf{Affidabilità}: capacità del prodotto di mantenere un certo livello di
					prestazioni quando utilizzato in particolari condizioni in un periodo temporale
					definito;
					\item \textbf{Efficienza}: capacità del prodotto di fornire determinate prestazioni
					in relazione alla quantità di risorse utilizzate;
					\item \textbf{Usabilità}: capacità del prodotto di essere capito e usato
					dall'utente	in specifiche condizioni;
					\item \textbf{Manutenibilità}: capacità del software di poter essere modificabile,
					correggendolo, migliorandolo o adattandolo;
					\item \textbf{Portabilità}: capacità del software di essere trasportato da un
					ambiente di lavoro ad un altro.
				\end{itemize}
				Sono presenti anche varie sotto-caratteristiche misurabili attraverso metriche.
			\subsubsection{Qualità in uso}
				La qualità in uso rappresenta la qualità del prodotto software dal punto di vista
				dell'utente ed è classificata da quattro caratteristiche:
				\begin{itemize}
					\item \textbf{Efficacia}: capacità del software di mettere in grado gli utenti di
					raggiungere i loro obiettivi con accuratezza e completezza;
					\item \textbf{Produttività}: capacità di mettere in grado gli utenti di utilizzare
					una quantità di risorse relativamente all'efficacia ottenuta in uno specifico
					contesto d'uso;
					\item \textbf{Soddisfazione}: capacità del prodotto di soddisfare gli utenti;
					\item \textbf{Sicurezza}: capacità del prodotto di raggiungere accettabili livelli
					di rischio di danni a persone, software, strumenti o all'ambiente operativo.
				\end{itemize}
		\subsection{Metriche per la qualità}
			\subsubsection{Esterna}
				Le metriche esterne misurano i comportamenti del software rilevati da test, operatività
				e osservazione durante la sua esecuzione in un contesto tecnico rilevante.
			\subsubsection{Interna}
				Le metriche interne sono applicate a software non eseguibile (codice sorgente) e 
				documentazione durante la progettazione e la codifica. Le misure effettuate permettono
				di prevedere il livello di qualità esterna ed in uso del prodotto finale data
				l'influenza degli attributi interni su quelli esterni e quelli in uso.
				Le metriche interne permettono di individuare eventuali problemi che potrebbero
				influire sulla qualità finale del prodotto prima che sia realizzato effettivamente
				il codice eseguibile.
\end{document}			