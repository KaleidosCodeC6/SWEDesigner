\documentclass[../PianoDiQualifica.tex]{subfiles}
\begin{document}
	\section{Visione generale della strategia}
		Per garantire la qualità dei prodotti realizzati durante lo sviluppo del
		progetto è indispensabile definire e perseguire strategie che assicurino la
		qualità dei processi adottati nonché il loro continuo miglioramento;
		inoltre, è necessario definire metriche e pianificare attività che valutino in
		modo preciso la qualità dei prodotti ottenuti e dei processi adottati. A tal
		scopo verranno adottate le seguenti strategie:
		\begin{itemize}
			\item Definizione accurata di norme che regolamentano e standardizzano i
			processi coinvolti nel progetto in termini di:
				\begin{itemize}
					\item Processo di fornitura;
					\item Processo di sviluppo;
					\item Processi di supporto;
					\item Processi organizzativi.
				\end{itemize}
			\item Descrizione dettagliata delle strategie di pianificazione adottate
			per lo sviluppo del progetto in termini di:
				\begin{itemize}
					\item Modello di sviluppo adottato;
					\item Analisi dei rischi che si possono incontrare;
					\item Pianificazione delle attività e dei tempi;
					\item Stima preventiva delle risorse che saranno impiegate;
					\item Assegnazione delle risorse al fine di portare a termine le
					attività pianificate nei tempi previsti;
					\item Consuntivo, durante lo sviluppo del progetto, delle risorse
					impiegate.
				\end{itemize}
			\item Ad ogni processo coinvolto nello sviluppo del progetto è stato scelto
			di applicare il ciclo di \gl{PDCA} affiancato dal CMM.\footnote{Per maggiori informazioni,
			consultare le appendici \ref{app:CMM} e \ref{app:PDCA}}
			Essi infatti permettono il controllo, la valutazione e il miglioramento
			continuo dei processi nonché la determinazione del livello di maturità
			dell'organizzazione nel gestirli.
		\end{itemize}
		\subsection{Obiettivi di qualità di processo}
			Prendendo come riferimento la normativa [ISO/IEC 12207], il gruppo \kaleidoscode\ ha
			definito i seguenti obiettivi di qualità di processo che si impegna a perseguire.
			\subsubsection{Rispetto della pianificazione concordata}
				È necessario pianificare le attività per la realizzazione del progetto ed è
				quindi fondamentale rispettare tale pianificazione per garantire la consegna
				del prodotto secondo le tempistiche concordate.\\
				Misura e metrica adottate sono descritte nella sezione \ref{MetrichePerIProcessi}.
			\subsubsection{Rispetto del budget concordato}
				È necessario stabilire un budget per il costo della realizzazione del prodotto ed è
				quindi fondamentale far rientrare i costi per le risorse nella spesa prevista.\\
				Misura e metrica adottate sono descritte nella sezione \ref{MetrichePerIProcessi}.
			\subsubsection{Leggibilità della documentazione prodotta}
				Durante la realizzazione del prodotto è necessario redigere la documentazione
				delle attività di pianificazione, gestione, sviluppo, verifica e validazione oltre
				che i loro prodotti.\\
				È quindi fondamentale che i documenti prodotti siano, per quanto tecnici,
				facilmente leggibili.\\
				Misura e metrica adottate sono descritte nella sezione \ref{MetrichePerIDocumenti}.
			\subsubsection{Rispetto delle norme di progettazione}
				In progettazione si definiranno i moduli e le componenti del prodotto. Sarà necessario
				rispettare le norme di progettazione definite nel documento \normediprogetto\
				mantenendone un valore ideale di $0$ violazioni.\\
				Inoltre, sarà utilizzata una misura e metrica, descritte nella sezione
				\ref{MetricheProgettuali}, per valutarne la qualità.
			\subsubsection{Rispetto delle norme di codifica}
				Nel periodo di codifica, bisognerà sviluppare le unità software ideate in
				progettazione. Per questo motivo, sarà necessario rispettare le norme di codifica
				concordate nel documento \normediprogetto\ mantenendone un valore ideale di $0$
				violazioni.
			\subsubsection{Corretto funzionamento delle componenti del sistema}
				Sarà necessario sottoporre le unità software sviluppate ad una serie di test\footnote{Definiti nella sezione \ref{Test}}
				per valutarne il corretto funzionamento;
				l'obiettivo è di effettuare il $100\%$ dei test indicati per ogni tipologia.
		\subsection{Obiettivi di qualità di prodotto}
			Prendendo come riferimento la normativa [ISO/IEC 9001] ed in particolare
			[ISO/IEC 9126]\footnote{Per maggiori informazioni, consultare l'appendice
			\ref{app:ISO9126}}, il gruppo \kaleidoscode\ ha definito i seguenti obiettivi di
			qualità che si impegna a far raggiungere al prodotto \progetto.
			\subsubsection{Funzionalità}
				Si garantisce che \progetto\ abbia tutte le funzionalità
				definite e concordate con il \proponente\ nel documento
				\analisideirequisitiv. L'implementazione di
				ogni requisito deve essere quanto più completa ed economica.
				\begin{itemize}
					\item \textbf{Misura}: si è deciso di utilizzare il numero totale di
					funzionalità del prodotto che soddisfano i requisiti definiti.
					\item \textbf{Metrica}: la sufficienza è raggiunta quando
					vengono soddisfatti almeno tutti i requisiti obbligatori.
				\end{itemize}
			\subsubsection{Affidabilità}
				Il sistema deve funzionare nella sua completezza.
				\begin{itemize}
					\item \textbf{Misura}: sono stati definiti dei test a cui sottoporre
					il prodotto realizzato\footnote{Consultabili nella sezione \ref{Test}};
					l'unità di misura sarà quindi il numero di test	superati dal sistema.
					\item \textbf{Metrica}: la sufficienza è raggiunta quando il sistema supera
					almeno l'$80$\% dei test.
				\end{itemize}
			\subsubsection{Efficienza}
				Il sistema deve minimizzare l'utilizzo delle risorse impiegate e
				fornire le funzionalità richieste nel minor tempo possibile.\\
				Misure e metriche adottate sono descritte nella sezione \ref{MetrichePerIlCodice}.
			\subsubsection{Manutenibilità}
				Il codice prodotto per realizzare il sistema deve essere comprensibile
				ed estensibile.\\
				Misure e metriche adottate sono descritte nella sezione \ref{MetrichePerIlCodice}.
			\subsubsection{Portabilità}
				Il sistema è un applicativo web. Per questo motivo, si garantisce che
				il \gl{front-end} sarà completamente funzionante ed utilizzabile
				almeno dal \gl{browser}	Google Chrome, dove verrà testato il prodotto
				durante lo sviluppo. Inoltre, si perseguirà l'obiettivo di garantire la
				completa funzionalità del prodotto anche su altri browser, in particolare:
				Mozilla Firefox e Microsoft Edge.\\
				Misure e metriche adottate sono descritte nella sezione \ref{MetrichePerIlCodice}.
		\subsection{Organizzazione}
			La gestione della strategia di verifica si basa sull'attuazione delle
			relative attività descritte nelle \normediprogettov. Tali attività vengono
			eseguite per ogni processo attuato allo scopo di verifica della qualità del
			processo stesso e dell'eventuale prodotto ottenuto facendo riferimento
			anche alle metriche definite nella sezione \ref{Misure&Metriche}.\\
			In ogni documento è presente un diario delle modifiche che permette di concentrare
			l'attività di verifica solo nelle parti modificate dopo l'ultima eseguita.\\
			Data la diversa natura dei prodotti ottenuti dai periodi del progetto si
			applicherà, per ognuno di essi, una diversa procedura di verifica:
			\begin{itemize}
				\item \textbf{Analisi e Analisi di dettaglio}: si effettuerà
				una prima stesura dei documenti illustrati nel \pianodiprogettov;
				\begin{itemize}
					\item Verrà controllata la correttezza ortografica con \textit{LanguageTool 3.6},
					opportunamente integrato in \textit{TexStudio};
					\item Verrà controllata la correttezza lessicale con un'attenta ed accurata
					rilettura affiancata dal controllo di \textit{LanguageTool 3.6};
					\item Verrà controllata la correttezza dei contenuti rispetto alle aspettative
					del documento con un'attenta rilettura;
					\item Verrà controllato il corretto tracciamento e la corrispondenza di ciascun
					requisito con un caso d'uso, mediante l'utilizzo dell'applicativo web creato
					appositamente\footnote{Consultare \normediprogettov\ per maggiori informazioni
					sullo strumento di tracciamento utilizzato.};
					\item Verrà controllato che la stesura di ciascun documento rispetti le
					norme definite in \normediprogettov;
					\item Verranno controllate le rappresentazioni grafiche, figure e tabelle
					assicurandosi che per ciascuna di esse sia presente un'opportuna didascalia e
					un relativo indice nel corrispondente documento;
				\end{itemize}
				\item \textbf{Progettazione architetturale}: verrà controllato che tutti i requisiti
				corrispondano ad un componente individuato in questo periodo e se ne assicurerà la
				tracciabilità;
				\item \textbf{Progettazione di dettaglio e Codifica}: durante ciascuna delle
				iterazioni in questo periodo i \programmatori\ svolgeranno l'attività di codifica e di
				esecuzione dei test previsti per la verifica del codice prodotto.
				Tali attività avverranno nel modo più automatizzato possibile seguendo le norme
				descritte in \normediprogettov.
				I \verificatori\ avranno il compito di supervisionare le attività
				controllando la presenza di eventuali errori;
				\item \textbf{Validazione e verifica}: verrà effettuato il collaudo
				del prodotto, in modo da assicurarne il corretto funzionamento al momento
				della consegna.
			\end{itemize}
			Per ogni periodo a partire dalla progettazione architetturale verranno inoltre effettuati
			tutti i controlli opportuni descritti al primo punto di questo paragrafo nei nuovi
			documenti redatti e in presenza	di modifiche o integrazioni	ai documenti
			precedentemente stesi.
		\subsection{Scadenze temporali}
			Dato l'obiettivo di rispettare le scadenze fissate nel \pianodiprogettov, è
			indispensabile pianificare anche l'attività di verifica della documentazione e del
			codice prodotto in modo che risulti sistematica e organizzata. Grazie
			all'applicazione di tale strategia l'individuazione e la correzione degli
			errori avverrà il prima possibile, impedendo la loro rapida diffusione e
			mitigando la possibilità che gli stessi si ripresentino in futuro,
			diminuendo così il rischio di ritardi. Tale pianificazione è documentata nel
			\pianodiprogettov\ il quale contiene, nella sottosezione ``Scadenze'', le
			scadenze temporali che il gruppo \kaleidoscode\ si impegna a rispettare.
\end{document}
