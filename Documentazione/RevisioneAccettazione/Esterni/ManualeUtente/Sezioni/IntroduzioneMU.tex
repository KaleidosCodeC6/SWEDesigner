\documentclass[../ManualeUtente.tex]{subfiles}
\begin{document}
	\section{Introduzione}
		\subsection{Scopo del documento}
			Questo documento funge da guida e da riferimento per l'utilizzo del prodotto
			\progetto.
		\subsection{Scopo del prodotto}
			\progetto\ ha lo scopo di permettere la
			costruzione di diagrammi UML con la possibilità di generare il relativo
			codice Java e/o Javascript.
			Più precisamente, il sistema offre le seguenti funzionalità principali:
			\begin{itemize}
				\item Fornisce la possibilità di realizzare l'architettura di un prodotto software
				mediante diagrammi in linguaggio UML (diagramma di package, classi e
				bubble\footnote{Maggiori informazioni nella sezione \ref{sez:BubbleDiagram}.}) opportunamente
				adattati allo scopo di guidare lo sviluppo del relativo codice;
				\item Fornisce la possibilità di generare il codice sorgente dai diagrammi realizzati
				con l'editor del prodotto\footnote{I diagrammi creati devono essere completi e corretti.}.
			\end{itemize}
			Il prodotto è utilizzabile da un browser desktop che supporti le tecnologie HTML5, CSS3
			e Javascript.
		\subsection{Pubblico considerato}
			Il corrente manuale fornisce le informazioni essenziali per utilizzare \progetto. Per i manutentori
			del prodotto o per chi fosse interessato alla sua integrazione/incremento, può invece consultare il
			manuale	sviluppatore ipertestuale reperibile all'indirizzo:\\
			\url{https://github.com/KaleidosCodeTeam/SWEDesigner-source}.\\
			Nelle spiegazioni si assume che l'utilizzatore di \progetto\ abbia una conoscenza quantomeno
			basilare sulla progettazione di software (in linguaggio UML) e sulla programmazione ad oggetti.\\
			In appendice \ref{sez:Glossario} è possibile consultare un glossario di termini utili di cui si
			farà uso nel manuale.
\end{document}
