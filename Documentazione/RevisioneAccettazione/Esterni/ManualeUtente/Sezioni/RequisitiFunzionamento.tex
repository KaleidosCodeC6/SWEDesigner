\documentclass[../ManualeUtente.tex]{subfiles}
\begin{document}
	\section{Requisiti per il funzionamento}
		\progetto\ è un'applicazione web, perciò, per il corretto funzionamento
		è necessario accedervi attraverso un browser web che supporti le
		tecnologie HTML5, CSS3 e che abbia attivato Javascript.\\
		Il funzionamento del prodotto è assicurato sul browser Google Chrome (almeno v58).
		L'applicazione funziona anche sul browser Microsoft Edge.
	\section{Istruzioni per l'installazione}\label{sez:installIstr}
		Per installare correttamente le componenti necessarie per il funzionamento di \progetto\ nel proprio computer,
		bisogna eseguire la seguente procedura come Utente Amministratore.
		\subsection{Per sistema operativo Linux/MacOS}
			\begin{enumerate}
				\item Scaricare ed installare Apache Web Server (usare configurazione della
				Apache's Web Page Root di default)
				\item Scaricare ed installare NodeJS (v6.11.1 o superiore)
	 			\item Spostare la cartella SWEDesigner-source nella directory ./var/www/html
	 			(è necessario avere i permessi di scrittura)
	 			\item Da terminale, spostarsi nella directory ./var/www/html/SWEDesigner-source e digitare il comando
	 			"./installaserver.sh"
			\end{enumerate}
		\subsection{Per sistema operativo Windows}
			\begin{enumerate}
				\item Scaricare ed installare Apache Web Server
	 			\item Scaricare ed installare NodeJS (v6.11.1 o superiore)
 				\item Scaricare ed installare (globalmente) i seguenti pacchetti node:
 				\begin{itemize}
 					\item express
 					\item multer
 					\item mysql
 					\item json-fn
 					\item archiver
 					\item body-parser
 				\end{itemize}
 				\item Spostare la cartella SWEDesigner-source nella directory della Apache's Web Page Root
 				(è necessario avere i permessi di scrittura)
			\end{enumerate}
	\section{Istruzioni per il funzionamento}\label{sez:funcIstr}
		\subsection{Per sistema operativo Linux/MacOS}
		 	\begin{enumerate}
		 		\item Da terminale, spostarsi nella directory ./var/www/html/SWEDesigner-source e digitare il comando
		 		"./runserver.sh" per avviare il server
	 			\item Aprire il browser e collegarsi all'indirizzo url\\
	 			http://localhost/SWEDesigner-source/src/webapp/SWEDesigner.html
		 	\end{enumerate}
		\subsection{Per sistema operativo Windows}
			\begin{enumerate}
		 		\item Da Prompt dei Comandi, spostarsi nella directory della\\
		 		Apache's Web Page Root\textbackslash SWEDesigner-source\textbackslash src\textbackslash server\textbackslash requestHandler e digitare il comando "node main.js"
		 		per avviare il server
		 		\item Aprire il browser e collegarsi all'indirizzo url\\
		 		http://localhost/SWEDesigner-source/src/webapp/SWEDesigner.html
		 	\end{enumerate}
\end{document}
