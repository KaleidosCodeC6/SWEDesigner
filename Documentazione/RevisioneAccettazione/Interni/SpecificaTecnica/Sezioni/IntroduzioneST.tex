\documentclass[../PianoDiQualifica.tex]{subfiles}
\begin{document}
	\section{Introduzione}
		\subsection{Scopo del documento} 
			Con il presente documento si intende definire la progettazione ad alto livello
			del progetto \progetto.\\
			Verrà presentata innanzi tutto l'architettura generale secondo la quale verranno
			organizzate le componenti software. Successivamente verranno descritti
			i \gl{Design pattern} utilizzati.
		\subsection{Scopo del prodotto}
			Lo scopo del progetto è la realizzazione di un software di
			costruzione di diagrammi \gl{UML} con la relativa generazione
			di codice \gl{Java} e \gl{Javascript} utilizzando tecnologie
			web. Il prodotto deve essere conforme ai vincoli qualitativi
			richiesti dal committente.
		\subsection{Glossario}
			Al fine di evitare ogni ambiguità di linguaggio e massimizzare la
			comprensione dei documenti i termini tecnici, di dominio, gli
			acronimi e le parole che necessitano di essere chiarite sono
			riportate nel documento \glossariov.\\
			La prima occorrenza di ciascuno di questi vocaboli è
			marcata da una ``G'' maiuscola in pedice.
		\subsection{Riferimenti utili}
			\subsubsection{Riferimenti normativi}
    			\begin{itemize}
    				\item \textbf{\gl{Capitolato} d'appalto}:\\
    				\url{http://www.math.unipd.it/~tullio/IS-1/2016/Progetto/C6.pdf} (09/03/2017);
    				\item \textbf{Norme di progetto}: \normediprogettov;
    				\item \textbf{Analisi dei requisiti}: \analisideirequisitiv;
    				\item \textbf{Verbali esterni}:
    				\begin{itemize}
    					\item Verbale incontro con \proponente\ in data 05/05/2017.
    				\end{itemize}
				\end{itemize}
			\subsubsection{Riferimenti informativi}	
				\begin{itemize}
					\item \textbf{Slide dell'insegnamento di Ingegneria del Software
					1\ap{o} semestre}:
					\begin{itemize}
						\item Design pattern strutturali:\\
						\url{http://www.math.unipd.it/~tullio/IS-1/2016/Dispense/E04.pdf} (02/05/2017);
						\item Design pattern creazionali:\\
						\url{http://www.math.unipd.it/~tullio/IS-1/2016/Dispense/E05.pdf} (02/05/2017);
						\item Design pattern comportamentali:\\
						\url{http://www.math.unipd.it/~tullio/IS-1/2016/Dispense/E06.pdf} (02/05/2017);
						\item Design pattern architetturali:\\
						\url{http://www.math.unipd.it/~tullio/IS-1/2016/Dispense/E07.pdf} (02/05/2017),\\
						\url{http://www.math.unipd.it/~tullio/IS-1/2016/Dispense/E08.pdf} (02/05/2017);
						\item Stili architetturali:\\
						\url{http://www.math.unipd.it/~tullio/IS-1/2016/Dispense/E09.pdf} (02/05/2017);
					\end{itemize}
					\item \textbf{Design Patterns: Elements of reusable object-oriented software}\\
					E. Gamma, R. Helm, R. Johnson, J. Vlissides - 1st Edition (2002)
					\begin{itemize}
						\item Capitolo 3: Creational patterns;
						\item Capitolo 4: Structural patterns;
						\item Capitolo 5: Behavioral patterns.
					\end{itemize}
				\end{itemize}
\end{document}
