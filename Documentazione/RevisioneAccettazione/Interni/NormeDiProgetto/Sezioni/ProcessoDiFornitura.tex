\documentclass[../NormeDiProgetto.tex]{subfiles}
\begin{document}
	\section{Processi primari}
		\subsection{Fornitura}
			\subsubsection{Scopo}
				Lo scopo del \textit{Processo di fornitura} è quello di consegnare un prodotto e/o un
				servizio che soddisfi i requisiti concordati.
			\subsubsection{Risultati}
				I risultati ottenuti in seguito ad una corretta attuazione del
				\textit{Processo di fornitura} sono:
				\begin{itemize}
					\item Stabilire un accordo tra Fornitore e Proponente in merito allo sviluppo, al
					mantenimento, al funzionamento, alla consegna e all'installazione del prodotto
					e/o servizio;
					\item Realizzare un prodotto e/o servizio che soddisfi i requisiti concordati;
					\item Consegnare il prodotto al Proponente in conformità con i requisiti
					concordati;
					\item Installare il prodotto in conformità con i requisiti concordati.
				\end{itemize}
			\subsubsection{Descrizione}
				Prendendo come riferimento lo standard [ISO/IEC 12207], il Fornitore deve svolgere le seguenti
				attività:
				\begin{itemize}
					\item Identificazione opportunità (\studiodifattibilita);
					\item Accordo contrattuale;
					\item Esecuzione del contratto;
					\item Consegna e supporto del prodotto e/o servizio;
					\item Chiusura.
				\end{itemize}
			\subsubsection{Identificazione opportunità}
				Successivamente alla pubblicazione dei capitolati d'appalto, il
				\responsabilediprogetto\ ha il compito di convocare il numero
				di riunioni necessarie al confronto tra i membri del gruppo sui
				capitolati proposti. Gli \analisti\ hanno così modo di ricavare sufficienti
				informazioni riguardanti le conoscenze e preferenze di ogni
				membro del gruppo. Sulla base delle decisioni prese, gli
				\analisti\ devono redigere uno \studiodifattibilita\
				dei capitolati secondo:
				\begin{itemize}
					\item \textbf{Dominio tecnologico}: conoscenze sulle
					tecnologie impiegate nello sviluppo del progetto in questione;
					\item \textbf{Dominio applicativo}: conoscenze sul dominio di
					applicazione del prodotto;
					\item \textbf{Individuazione di rischi e criticità}: punti
					critici ed eventuali rischi percorribili durante lo sviluppo.
				\end{itemize}
				Nello \studiodifattibilitav\ sono racchiuse le motivazioni che hanno spinto il nostro
				gruppo a candidarsi come Fornitore per il proponente \proponente.
			\subsubsection{Accordo contrattuale}
				Il Fornitore deve accordarsi con il proponente \proponente\ per chiarire, definire e
				accettare le richieste presenti nel documento di presentazione del capitolato fornito.
			\subsubsection{Esecuzione del contratto}
				Il Fornitore è tenuto a collaborare con il proponente \proponente\ per tutta la durata
				del progetto al fine di raggiungere i seguenti obbiettivi:
				\begin{itemize}
					\item Chiarire ogni dubbio riguardante i vincoli sui requisiti;
					\item Chiarire ogni dubbio riguardante i vincoli di progetto.
				\end{itemize}
				Il Fornitore è tenuto a procurare al proponente \proponente\ e al committente
				\vardanega\ i seguenti documenti:
				\begin{itemize}
					\item \pianodiprogetto;
					\item \analisideirequisiti;
					\item \pianodiqualifica.
				\end{itemize}
			\subsubsection{Consegna e supporto del prodotto e/o servizio}
				Dopo aver terminato le fasi di sviluppo, verifica e validazione, il Fornitore è tenuto
				a consegnare al committente \vardanega\ il prodotto realizzato in conformità con i
				requisiti richiesti. Dovrà quindi consegnare un CD-ROM/DVD comprensivo di:
				\begin{itemize}
					\item Utilità di installazione e istruzioni per l'uso;
					\item Sorgenti completi e utilità di compilazione;
					\item Documentazione completa e finale;
					\item Eventuali utilità di collaudo.
				\end{itemize}
				Il Fornitore, dopo la consegna del prodotto, non si occuperà della fase di
				manutenzione del suddetto.
			\subsubsection{Chiusura}
				La chiusura dell'accordo contrattuale tra Fornitore e Proponente è sancita dalla
				consegna del prodotto realizzato.
\end{document}