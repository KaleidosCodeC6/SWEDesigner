\documentclass[../NormeDiProgetto.tex]{subfiles}
\begin{document}
	\section{Introduzione}
		\subsection{Scopo del documento}
			Questo documento definisce le norme che i membri del gruppo
			\kaleidoscode\ adotteranno nello svolgimento del progetto
			\progetto.\\
			Tutti i membri del gruppo sono tenuti a leggere il documento
			e a seguirne le norme descritte per uniformare il materiale
			prodotto, ridurre il numero di errori e migliorare l'efficienza.\\
			In particolare verranno definite norme riguardanti:
			\begin{itemize}
				\item Interazioni tra i membri del gruppo;
				\item Stesura di documenti e convenzioni;
				\item Modalità di lavoro durante le varie fasi del progetto;
				\item Ambiente di lavoro.
			\end{itemize}
		\subsection{Scopo del prodotto}
			Lo scopo del progetto è la realizzazione di un software di
			costruzione di diagrammi \gl{UML} con la relativa generazione
			di codice \gl{Java} e \gl{Javascript} utilizzando tecnologie
			web.
		\subsection{Glossario}
			Al fine di evitare ogni ambiguità di linguaggio e massimizzare la
			comprensione dei documenti, i termini tecnici, di dominio, gli
			acronimi e le parole che necessitano di essere chiarite sono
			riportate nel documento \glossariov.\\
			La prima occorrenza di questi vocaboli è
			marcata da una ``G'' maiuscola in pedice.
		\subsection{Riferimenti utili}
			\subsubsection{Riferimenti normativi}
    			\begin{itemize}
    				\item \textbf{\gl{ISO} 12207}:\\
    				\url{https://en.wikipedia.org/wiki/ISO/IEC_12207} (03/03/2017).
				\end{itemize}
			\subsubsection{Riferimenti informativi}	
				\begin{itemize}
					\item \textbf{Specifiche \gl{UTF-8}}:\\
					\url{http://www.unicode.org/versions/Unicode6.1.0/ch03.pdf} (02/04/2017);
					\item \textbf{\gl{Capitolato} d'appalto}:\\
					\url{http://www.math.unipd.it/~tullio/IS-1/2016/Progetto/C6.pdf} (02/04/2017);
					\item \textbf{Notazione CamelCase}:\\
					\url{https://en.wikipedia.org/wiki/Camel_case} (02/04/2017);
					\item \textbf{Stile d'indentazione K\&R}:\\
					\url{https://it.wikipedia.org/wiki/Stile_d\%27indentazione#Stile_K.26R} (02/04/2017);
					\item \textbf{Glossario}: \glossariov;
					\item \textbf{Piano di Progetto}: \pianodiprogettov;
					\item \textbf{Piano di Qualifica}: \pianodiqualificav.
				\end{itemize}
\end{document}
