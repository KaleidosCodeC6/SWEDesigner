\documentclass[../StudioDiFattibilita.tex]{subfiles}
\begin{document}
	\section{Capitolato scelto}
		\subsection{Descrizione}
			\kaleidoscode\ ha scelto di presentare una proposta d'appalto per il capitolato C6: \progetto.\\
			Lo scopo del progetto è di sviluppare un editor di diagrammi UML che generi il relativo codice \gl{Java} e
			\gl{Javascript} automaticamente.
			Viene specificatamente richiesto di poter realizzare, attraverso l'editor, il diagramma delle classi
			e il diagramma delle attività (tra i vari tipi previsti dall'UML) e di ricavarne rispettivamente lo
			scheletro delle classi e il corpo dei metodi nei due linguaggi di programmazione indicati.
		\subsection{Studio del dominio}
			\subsubsection{Dominio applicativo}
				Il prodotto richiesto dal capitolato si colloca nel dominio degli strumenti per la realizzazione
				di nuovo software.\\
				Finora, la relazione tra i diagrammi UML sviluppati durante la fase progettazione e il codice
				prodotto nella fase di realizzazione non è mai stata forte.
				È quindi richiesto di esplorare iterazioni ed eventuali estensioni che avvicinino le due fasi,
				per rendere possibile l'ottenimento di codice funzionante dai soli diagrammi UML, almeno
				all'interno di un dominio circoscritto.
			\subsubsection{Dominio tecnologico}
				È richiesto che il sistema venga realizzato utilizzando tecnologie web.\\
				In particolare, la parte server deve essere realizzata in Java con server \gl{Tomcat} oppure in
				Javascript con server \gl{Node.Js}, mentre la parte client deve essere eseguibile in un \gl{browser}
				\gl{HTML}5 e deve utilizzare fogli di stile \gl{CSS} e Javascript rispettivamente per la presentazione e il comportamento.
		\subsection{Valutazione}
			\subsubsection{Aspetti positivi}
				Gli aspetti positivi individuati dal gruppo sono:
				\begin{itemize}
					\item Alto interesse nell'affrontare un progetto che preveda una fase di ricerca e permetta
					di confrontarsi con un problema aperto;
					\item Lavorare all'interno del dominio tecnologico sopra riportato è altamente formativo
					vista l'ampia diffusione di cui godono le tecnologie richieste all'interno del mondo del
					lavoro;
					\item La diffusione delle tecnologie richieste ha portato ad un'ampia disponibilità di
					documentazione a riguardo e ad una notevole quantità di software open-source che offre buone
					possibilità alla riusabilità di codice.
				\end{itemize}
			\subsubsection{Fattori di rischio}
				I fattori di rischio individuati dal gruppo sono:
				\begin{itemize}
					\item Inesperienza sulle tecnologie adottate: ciascun membro conosce solo superficialmente il
					dominio tecnologico di sviluppo;
					\item La conversione da diagramma UML a codice risulta non essere sempre possibile.	Di
					conseguenza, sarà necessario trovare una soluzione per gestire l'eventuale situazione.
				\end{itemize}
		\subsection{Analisi di mercato}
			Si riporta dal capitolato:\\
			``L'innovazione oggi è la costante di qualunque settore di attività lavorativa. Motore
			dell'innovazione è il software, che permette di inserire elementi di agilità ed intelligenza in ogni
			attività umana, dalla fornitura di servizi alle realizzazioni meccaniche. La costante richiesta di
			nuovo software si scontra con la cronica mancanza di esperti e con la bassa qualità del software
			prodotto; per aumentare la qualità e la velocità di produzione occorre rendere questa attività un
			processo industriale ingegnerizzato allontanandosi dall'artigianalità che ancora a volte lo
			caratterizza.''\\
			Lo svolgimento del progetto porterà allo sviluppo di un prodotto che tenterà di rendere più agevole
			la progettazione di software di qualità.
		\subsection{Conclusioni}
			Per il grande interesse suscitato, la voglia di mettersi in gioco per cercare una soluzione ad un
			problema attuale e l'importante valore formativo del progetto, il gruppo ha deciso di presentare una
			proposta d'appalto per il corrispondente capitolato.
\end{document}
