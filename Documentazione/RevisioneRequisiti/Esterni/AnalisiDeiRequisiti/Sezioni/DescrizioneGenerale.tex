\documentclass[../AnalisiDeiRequisiti.tex]{subfiles}
\begin{document}
	\section{Descrizione generale}
		\subsection{Obiettivo del prodotto}
			Il prodotto si pone l'obiettivo di facilitare lo sviluppo di software in
			linguaggio Java e/o Javascript e di aumentare la correlazione del rapporto tra
			i diagrammi UML prodotti in fase di progettazione ed il codice frutto della
			fase di sviluppo e realizzazione.\\
			Il prodotto cerca quindi di rafforzare il legame tra la fase di progettazione
			e quella di realizzazione.
		\subsection{Funzioni del prodotto}
			L'applicativo fornisce un'interfaccia grafica per l'apertura, la nuova
			creazione ed il salvataggio di progetti (l'insieme dei diagrammi UML prodotti).
			All'interno degli editor sarà possibile creare, modificare e rimuovere gli
			elementi appartenenti ai rispettivi diagrammi ed inoltre, sarà possibile
			visualizzare ed esportare il relativo codice Java e/o Javascript generato.
			All'interno di \progetto\ è presente anche l'editor di bubble, ovvero un editor
			che, associato al diagramma delle attività, ne aiuta la traduzione in codice.\\
			Una bubble è quindi un'attività atomicamente traducibile in codice.\\
			È possibile modificare le bubble all'interno dell'editor, scrivendo delle istruzioni
			utili alla fase di traduzione dei diagrammi, ma in questo caso verrà segnalata la
			possibile presenza di errori e non sarà garantita la corretta compilazione
			del codice.
		\subsection{Caratteristiche degli utenti}
			Il prodotto si rivolge a programmatori e, in generale, utenti che conoscono
			il linguaggio UML. L'applicativo prevede una sola tipologia di utente,
			l'utilizzatore finale, poiché non sono emersi requisiti riguardanti
			l'esistenza di una gerarchia di quest'ultimi.
		\subsection{Piattaforma di esecuzione}
			L'applicativo sarà realizzato utilizzando tecnologie web quali \gl{HTML}5,
			\gl{CSS} e Javascript, pertanto la piattaforma di esecuzione potrà essere un
			qualunque \gl{browser} desktop avente Javascript attivo.
		\subsection{Vincoli generali}
			Per poter utilizzare il prodotto sarà necessario disporre di un computer
			connesso a \gl{Internet} avente un browser web nel quale è attivato
			Javascript. Non sono richiesti particolari requisiti hardware.
\end{document}