
%PROBABILE ERRORE: Spesso ho scritto la vocale nell'articolo precedente a parole che cominciano in "a-" invece di apostrofarlo. Ne ho corretti molti ma alcuni possono essere sfuggiti
%				Scrivo ogni volta "almento" invece di "almeno"

% SEZIONE: EDITARE IL DIAGRAMMA DELLE ATTIVITÀ

\paragraph{Aggiungere pin}
\begin{itemize}
	\item \textbf{Attori}: Utente;
	\item \textbf{Descrizione}: L’utente deve poter aggiungere pin ad una attività;
	\item \textbf{Precondizione}: L’utente si trova nella schermata dell’editor del diagramma delle attività ed è presente almeno un’attività;
	\item \textbf{Postcondizione}: L’utente ha modificato una attività aggiungendo un pin in input o in output con campi vuoti;
\end{itemize}

\paragraph{Modificare pin}
\begin{itemize}
	\item \textbf{Attori}: Utente;
	\item \textbf{Descrizione}: L’utente deve poter modificare i pin da lui inseriti così da impostare la variabile da essi rappresentata;
	\item \textbf{Precondizione}: L’utente si trova nella schermata dell’editor del diagramma delle attività ed è presente almeno un’attività dotata di un pin;
	\item \textbf{Postcondizione}: L’utente ha modificato il pin, che ora rappresenta correttamente la variabile corrispondente;
\end{itemize}

\paragraph{Rimuovere pin}
\begin{itemize}
	\item \textbf{Attori}: Utente;
	\item \textbf{Descrizione}: L’utente deve poter eliminare un pin da un’attività;
	\item \textbf{Precondizione}: L’utente si trova nella schermata dell’editor del diagramma delle attività ed è presente almeno un’attività dotata di un pin;
	\item \textbf{Postcondizione}: L’utente ha eliminato il pin desiderato dalla attività;
\end{itemize}

\paragraph{Aggiungere trasformazione}
\begin{itemize}
	\item \textbf{Attori}: Utente;
	\item \textbf{Descrizione}: L’utente deve poter collegare tra di loro i pin di output di un’attività e i pin di input di una seconda mediante una trasformazione;
	\item \textbf{Precondizione}: L’utente si trova nella schermata dell’editor del diagramma delle attività e sono presenti almeno due attività di cui una dotata di pin in output e una dotata di pin in input, con tipo tra di loro compatibile;
	\item \textbf{Postcondizione}: L’utente ha collegato il pin di output della prima classe con il pin di input della seconda mediante una trasformazione con espressione di trasformazione vuota;
\end{itemize}

\paragraph{Modificare trasformazione}
\begin{itemize}
	\item \textbf{Attori}: Utente;
	\item \textbf{Descrizione}: L’utente deve poter modificare una trasformazione, in particolare editando l’\textit{espressione di trasformazione};
	\item \textbf{Precondizione}: L’utente si trova nella schermata dell’editor del diagramma delle attività ed è presente almeno una trasformazione che collega pin di output e di input di due attività;
	\item \textbf{Postcondizione}: L’utente ha modificato l’espressione di trasformazione voluta;
\end{itemize}

\paragraph{Rimuovere trasformazione}
\begin{itemize}
	\item \textbf{Attori}: Utente;
	\item \textbf{Descrizione}: L’utente deve poter eliminare una trasformazione che colleghi due pin di due attività;
	\item \textbf{Precondizione}: L’utente si trova nella schermata dell’editor del diagramma delle attività ed è presente almeno una trasformazione che collega pin di output e di input di due attività;
	\item \textbf{Postcondizione}: L’utente ha eliminato la trasformazione che collegava i due pin;
\end{itemize}

\paragraph{Aggiungere invio di segnale}
\begin{itemize}
	\item \textbf{Attori}: Utente;
	\item \textbf{Descrizione}: L'utente deve poter aggiungere ad un'attività un invio di segnale;
	\item \textbf{Precondizione}: L'utente si trova nella schermata dell'editor del diagramma delle attività ed è presente almeno un'attività;
	\item \textbf{Postcondizione}: L'utente ha modificato l'attività desiderata aggiungendo alla stessa un invio di segnale vuoto;
\end{itemize}

\paragraph{Modificare invio di segnale}
\begin{itemize}
	\item \textbf{Attori}: Utente;
	\item \textbf{Descrizione}: L'utente deve poter modificare l'invio di segnale associato ad un'attività;
	\item \textbf{Precondizione}: L'utente si trova nella schermata dell'editor del diagramma delle attività ed è presente almeno un'attività con associato un invio di segnale;
	\item \textbf{Postcondizione}: L'utente ha modificato i parametri dell'invio di segnale associato all'attività;
\end{itemize}

\paragraph{Rimuovere invio di segnale}
\begin{itemize}
	\item \textbf{Attori}: Utente;
	\item \textbf{Descrizione}: L'utente deve poter rimuovere un invio di segnale associato ad un'attività;
	\item \textbf{Precondizione}: L'utente si trova nella schermata dell'editor del diagramma delle attività ed è presente almeno un'attività con associato un invio di segnale;
	\item \textbf{Postcondizione}: L'utente ha eliminato l'invio di segnale associato all'attività;
\end{itemize}

\paragraph{Aggiungere attesa di segnale}
\begin{itemize}
	\item \textbf{Attori}: Utente;
	\item \textbf{Descrizione}: L'utente deve poter aggiungere ad un'attività un'attesa di segnale;
	\item \textbf{Precondizione}: L'utente si trova nella schermata dell'editor del diagramma delle attività ed è presente almeno un'attività;
	\item \textbf{Postcondizione}: L'utente ha modificato l'attività desiderata aggiungendo alla stessa un'attesa di segnale vuota;
\end{itemize}

\paragraph{Modificare attesa di segnale}
\begin{itemize}
	\item \textbf{Attori} Utente;
	\item \textbf{Descrizione}: L'utente si trova nella schermata dell'editor del diagramma delle attività ed è presente almeno un'attività con associata un'attesa di segnale;
	\item \textbf{Precondizione}: L'utente si trova nella schermata dell'editor del diagramma delle attività ed è presente almeno un'attività con associata un'attesa di segnale;
	\item \textbf{Postcondizione}: L'utente ha modificato i parametri dell'attesa di segnale associata all'attività;
\end{itemize}

\paragraph{Rimuovere attesa di segnale}
\begin{itemize}
	\item \textbf{Attori}: Utente;
	\item \textbf{Descrizione}: L'utente deve poter rimuovere un'attesa di segnale associata ad un'attività;
	\item \textbf{Precondizione}: L'utente si trova nella schermata dell'editor del diagramma delle attività ed è presente almeno un'attività con associata un'attesa di segnale;
	\item \textbf{Postcondizione}: L'utente ha eliminato l'attesa di segnale associata all'attività;
\end{itemize}

\paragraph{Aggiungere evento temporale}
\begin{itemize}
	\item \textbf{Attori}: Utente;
	\item \textbf{Descrizione}: L'utente deve poter aggiungere un evento temporale al diagramma delle attività;
	\item \textbf{Precondizione}: L'utente si trova nella schermata dell'editor del diagramma delle attività;
	\item \textbf{Postcondizione}: L'utente ha aggiunto un evento temporale vuoto al diagramma delle attività;
\end{itemize}

\paragraph{Modificare evento temporale}
\begin{itemize}
	\item \textbf{Attori}: Utente;
	\item \textbf{Descrizione}: L'utente deve poter modificare il tipo e i parametri di un evento temporale;
	\item \textbf{Precondizione}: L'utente si trova nella schermata dell'editor del diagramma delle attività ed è presente almeno un evento temporale;
	\item \textbf{Postcondizione}: L'utente ha modificato a suo piacimento il tipo ed i parametri dell'evento temporale;
\end{itemize}

\paragraph{Rimuovere evento temporale}
\begin{itemize}
	\item \textbf{Attori}: Utente;
	\item \textbf{Descrizione}: L'utente deve poter eliminare un evento temporale;
	\item \textbf{Precondizione}: L'utente si trova nella schermata dell'editor del diagramma delle attività ed è presente almeno un evento temporale;
	\item \textbf{Postcondizione}: L'utente ha eliminato l'evento temporale;
\end{itemize}


% SEZIONE: EDITARE IL BUBBLE FLOWCHART
% CONTROLLO: DEVE ESSERE OMOGENEA CON LE CORRISPONDENTI SEZIONI DEL DIAGRAMMA DELLE ATTIVITÀ

\paragraph{Aggiungere una bubble}
\begin{itemize}
	\item \textbf{Attori}: Utente;
	\item \textbf{Descrizione}: L'utente deve poter aggiungere una bubble di un tipo desiderato al bubble flowchart;
	\item \textbf{Precondizione}: L'utente si trova nella schermata dell'editor del bubble flowchart;
	\item \textbf{Postcondizione}: L'utente ha aggiunto una bubble vuota del tipo voluto al bubble flowchart;
\end{itemize}

\paragraph{Modificare una bubble}
\begin{itemize}
	\item \textbf{Attori}: Utente;
	\item \textbf{Descrizione}: L'utente deve poter modificare i parametri di una bubble;
	\item \textbf{Precondizione}: L'utente si trova nella schermata dell'editor del bubble flowchart ed è presente almento una bubble;
	\item \textbf{Postcondizione}: L'utente ha modificato a suo piacimento i parametri della bubble;
\end{itemize}

\paragraph{Rimuovere una bubble}
\begin{itemize}
	\item \textbf{Attori}: Utente;
	\item \textbf{Descrizione}: L'utente deve poter rimuovere una bubble;
	\item \textbf{Precondizione}: L'utente si trova nella schermata dell'editor del bubble flowchart ed è presente almeno una bubble;
	\item \textbf{Postcondizione}: L'utente ha eliminato la bubble;
\end{itemize}

\paragraph{Aggiungere un elemento di decisione}
\begin{itemize}
	\item \textbf{Attori}: Utente;
	\item \textbf{Descrizione}: L'utente deve poter aggiungere un elemento di decisione al bubble flowchart;
	\item \textbf{Precondizione}: L'utente si trova nella schermata dell'editor del bubble flowchart;
	\item \textbf{Postcondizione}: L'utente ha aggiunto un elemento di decisione vuoto al bubble flowchart;
\end{itemize}

\paragraph{Modificare un elemento di decisione}
\begin{itemize}
	\item \textbf{Attori}: Utente;
	\item \textbf{Descrizione}: L'utente deve poter modificare i parametri di un elemento di decisione;
	\item \textbf{Precondizione}: L'utente si trova nella schermata dell'editor del bubble flowchart ed è presente almeno un elemento di decisione;
	\item \textbf{Postcondizione}: L'utente ha modificato a suo piacimento i parametri dell'elemento di decisione;
\end{itemize}

\paragraph{Eliminare un elemento di decisione}
\begin{itemize}
	\item \textbf{Attori}: Utente;
	\item \textbf{Descrizione}: L'utente deve poter eliminare un elemento di decisione;
	\item \textbf{Precondizione}: L'utente si trova nella schermata dell'editor del bubble flowchart ed è presente almeno un elemento di decisione;
	\item \textbf{Postcondizione}: L'utente ha eliminato l'elemento di decisione;
\end{itemize}


% SEZIONE: LEGGERE IL CODICE GENERATO

\paragraph{Esportare il codice}
\begin{itemize}
	\item \textbf{Attori}: Utente;
	\item \textbf{Descrizione}: L'utente deve poter esportare il codice generato nei file sorgente appropriati per il linguaggio corrispondente;
	\item \textbf{Precondizione}: L'utente ha disegnato i diagrammi che rappresentano il codice da lui desiderato mediante gli editor messi a disposizione dal programma e questi sono attualmente aperti nel programma;
	\item \textbf{Postcondizione}: Il programma ha generato in una cartella a scelta dell'utente tutti i file sorgenti voluti, organizzati secondo quanto specificato dall'utente tramite i diagrammi. Questi file contengono codice corretto e compilabile. Qualora il programma non avesse potuto tradurre efficacemente una parte del diagramma dell'utente, il programma ha comunicato un avvertimento all'utente e commentato opportunamente il codice nel sorgente;
\end{itemize}

% SEZIONE: SALVARE I DIAGRAMMI PRODOTTI

\paragraph{Salvare i diagrammi prodotti}
\begin{itemize}
	\item \textbf{Attori}: Utente;
	\item \textbf{Descrizione}: L'utente deve poter salvare il lavoro fatto fino a quel momento;
	\item \textbf{Precondizione}: L'utente ha disegnato i diagrammi che rappresentano il codice da lui desiderato mediante gli editor messi a disposizione dal programma e questi sono attualmente aperti nel programma;
	\item \textbf{Postcondizione}: Il programma ha generato in una cartella a scelta dell'utente un file contenente tutte le informazioni necessarie al programma per ripristinare il suo stato attuale qualora venisse aperto questo file;
\end{itemize}























