

%  !!!!!!!!! due punti da decidere !!!!!!!!!


\documentclass[a4paper]{report}
\usepackage[english, italian]{babel}
\usepackage[T1]{fontenc}
\usepackage[utf8]{inputenc}
\usepackage{url}
\usepackage{graphicx}
\graphicspath{{../Figures/}}
\usepackage[hidelinks]{hyperref}
\usepackage{booktabs}
\usepackage{tabularx}
\usepackage{pifont}
\usepackage[table]{xcolor}	
\usepackage{float}

\begin{document}
\chapter{Casi d'uso }
\section{Caso d'uso UC1: Editare il diagramma delle classi}
\begin{itemize}
	\item \textbf{Attori}: Utente;
	
	\item \textbf{Descrizione}: L'utente ha avviato correttamente il programma e ha aperto un progetto. Ora l'utente può editare il diagramma delle classi ;
	
	\item \textbf{Precondizione}: L'utente ha avviato correttamente il programma e ha aperto un progetto;
	\item \textbf{Flusso principale degli eventi}:
	\begin{enumerate}
		\item L'utente può creare una nuova classe (UC1.1);
		\item L'utente può modificare una classe (UC1.2);
		\item L'utente può eliminare una classe (UC1.3);
		\item L'utente può definire una nuova relazione (UC1.4);
		\item L'utente può modificare una relazione (UC1.5);
		\item L'utente può eliminare una relazione (UC1.6);
		\item L'utente può creare una nuova interfaccia (UC1.7);
		\item L'utente può modificare una interfaccia (UC1.8);
		\item L'utente può eliminare una interfaccia (UC1.9);
		\item L'utente può cambiare layer di visualizzazione (UC1.10).
	\end{enumerate}
	
	\item \textbf{Scenari alternativi}: Viene annullata la modifica, il sistema
	rimane nello stato precedente al tentativo di modifica
	
	\item \textbf{Postcondizione}: Il sistema apporta le modifiche desiderate al diagramma delle classi.
\end{itemize}

\subsection{Caso d'uso UC1.1: Creare una nuova classe}
\begin{itemize}
	\item \textbf{Attori}: Utente;
	
	\item \textbf{Descrizione}: L'utente può aggiungere una nuova classe vuota al diagramma delle classi;
	
	\item \textbf{Precondizione }: Il programma è in esecuzione con un progetto apreto nel diagramma delle classi;
		
	!!!!!!!! da decidere se e' possibile annullare !!!!!!!!
	
	\item \textbf{Scenari alternativi}: Viene annullata la modifica, il sistema
	rimane nello stato precedente al tentativo di creazione;
	
	\item \textbf{Postcondizione}: Viene aggiunta una nuova classe al diagramma delle classi.
\end{itemize}

\subsection{Caso d'uso UC1.2: Modificare una classe}
\begin{itemize}
	\item \textbf{Attori}: Utente;
	
	\item \textbf{Descrizione}: L'utente vuole modificare uno dei vari attributi della classe selezionata ;
	
	\item \textbf{Precondizione}: L'utente ha selezionato una classe da modificare all'interno di un progetto;
	
	\item \textbf{Flusso principale degli eventi}:
	\begin{enumerate}
		\item L'utente può creare impostare il nome della classe (UC1.2.1);
		\item L'utente può aggiungere un campo dati a una classe (UC1.2.2);
		\item L'utente può eliminare un campo dati a una classe (UC1.2.3);
		\item L'utente può modificare un campo dati a una classe (UC1.2.4);
		\item L'utente può aggiungere un' operazione a una classe (UC1.2.5);
		\item L'utente può eliminare un' operazione a una classe (UC1.2.6);
		\item L'utente può modificare un'operazione a una classe (UC1.2.7);
	\end{enumerate}

	\item \textbf{Scenari alternativi}: Viene annullata la modifica, il sistema 
	rimane nello stato precedente al tentativo di modifica;


	\item \textbf{Postcondizione}: Le varie modifiche decise dall'utente verranno applicate alla classe nel diagramma delle classi .
\end{itemize}

\subsubsection{Caso d'uso UC1.2.1: Impostare il nome della classe}
\begin{itemize}
	\item \textbf{Attori}: Utente;
	
	\item \textbf{Descrizione}: L'utente vuole impostare il nome di una classe 
	
	!!!!culoculoculo appena creata (da decidere ) culoculoculo!!!!! ;
	
	\item \textbf{Precondizione}: L'utente ha creato una classe all'interno del diagramma delle classi;
	
	\item \textbf{Scenari alternativi}: Viene annullata la modifica, la classe
	rimane con il nome precedente al tentativo di motifica;
	
	\item \textbf{Postcondizione}: Il nuovo nome deciso dall'utente viene impostato come nome della classe all'interno del diagramma delle classi.
\end{itemize}

\subsubsection{Caso d'uso UC1.2.2: Aggiungere un campo dati alla classe}
\begin{itemize}
	\item \textbf{Attori}: Utente;

	\item \textbf{Descrizione}: L'utente vuole aggiungere un nuovo campo dati a una classe;
	
	\item \textbf{Precondizione}: L'utente ha selezionato una classe all'interno del diagramma delle classi alla quale vuole aggiungere un campo dati;
	
	\item \textbf{Scenari alternativi}: Viene annullata la modifica, la classe
	rimane nello stato precedente al tentativo di aggiunta;
	
	\item \textbf{Postcondizione}: Il campo dati viene aggiunto alla classe con i parametri decisi dall'utente.
\end{itemize}

\subsubsection{Caso d'uso UC1.2.3: Eliminare un campo dati alla classe}
\begin{itemize}
	\item \textbf{Attori}: Utente;
	
	\item \textbf{Descrizione}: L'utente vuole eliminare un campo dati all'interno di una classe;
	
	\item \textbf{Precondizione}: L'utente ha selezionato il campo dati che vuole eliminare;
	
	\item \textbf{Scenari alternativi}: Viene annullata la modifica, la classe
	rimane nello stato precedente al tentativo di cancellazione;
	
	
	\item \textbf{Postcondizione}: Il campo dati viene rimosso dalla classe dopo eventuali avvisi nel caso ci siano dipendenze da controllare .
\end{itemize}

\subsubsection{Caso d'uso UC1.2.4: Modificare un campo dati alla classe}
\begin{itemize}
	\item \textbf{Attori}: Utente;
	
	\item \textbf{Descrizione}: L'utente vuole modificare un campo dati all'interno di una classe del diagramma delle classi;
	
	\item \textbf{Precondizione}: L'utente ha selezionato il campo dati che vuole modificare all'interno di una classe;
	
	\item \textbf{Scenari alternativi}: Viene annullata la modifica, il campo dati
	rimane nello stato precedente al tentativo di modifica;
	
	\item \textbf{Postcondizione}: Le modifiche decise dall'utente vengono applicate al campo dati all'interno della classe.
\end{itemize}

\subsubsection{Caso d'uso UC1.2.5: Aggiungere un' operazione alla classe}
\begin{itemize}
	\item \textbf{Attori}: Utente;
	
	\item \textbf{Descrizione}: L'utente vuole aggiungere un' operazione a una classe;
	
	\item \textbf{Precondizione}: L'utente ha selezionato una classe all'interno del diagramma delle classi alla quale vuole aggiungere un'operazione;
	
	\item \textbf{Scenari alternativi}: Viene annullata la modifica, la classe
	rimane nello stato precedente al tentativo di aggiunta;
	
	\item \textbf{Postcondizione}: L'operazione viene aggiunta alla classe con i parametri decisi dall'utente.
\end{itemize}

\subsubsection{Caso d'uso UC1.2.6: Eliminare un' operazione dalla classe}
\begin{itemize}
	\item \textbf{Attori}: Utente;
	
	\item \textbf{Descrizione}: L'utente vuole eliminare un'operazione all'interno di una classe;
	
	\item \textbf{Precondizione}: L'utente ha selezionato L'operazione che vuole eliminare;
	
	\item \textbf{Scenari alternativi}: Viene annullata la modifica, la classe
	rimane nello stato precedente al tentativo di cancellazione;
	
	\item \textbf{Postcondizione}: L'operazione viene rimossa dalla classe dopo
	 eventuali avvisi nel caso ci siano dipendenze da controllare .
\end{itemize}

\subsubsection{Caso d'uso UC1.2.7: Modificare un' operazione alla classe}
\begin{itemize}
	\item \textbf{Attori}: Utente;
	
	\item \textbf{Descrizione}: L'utente vuole modificare un' operazione all'interno di una classe del diagramma delle classi;
	
	\item \textbf{Precondizione}: L'utente ha selezionato l'operazione che vuole modificare all'interno di una classe;
	
	\item \textbf{Scenari alternativi}: Viene annullata la modifica, l'operazione 
	rimane nello stato precedente al tentativo di modifica;
	
	\item \textbf{Postcondizione}: Le modifiche decise dall'utente vengono applicate all'operazione all'interno della classe.
\end{itemize}

\end{document}