\documentclass[../AnalisiDeiRequisiti.tex]{subfiles}
\begin{document}
	\section{Introduzione}
		\subsection{Scopo del documento}
			Nel presente documento sono elencati e descritti dettagliatamente i requisiti
			rilevati dall'analisi del capitolato d'appalto C6 e dagli incontri effettuati
			con il proponente ed il committente del progetto \progetto.
		\subsection{Scopo del prodotto}
			Lo scopo del progetto è la realizzazione di un software di
			costruzione di diagrammi \gl{UML} con la relativa generazione
			di codice \gl{Java} e \gl{Javascript} utilizzando tecnologie
			web.
		\subsection{Glossario}
			Al fine di evitare ogni ambiguità di linguaggio e massimizzare la
			comprensione dei documenti, i termini tecnici, di dominio, gli
			acronimi e le parole che necessitano di essere chiarite, sono
			riportate nel documento \glossariov.\\
			Ogni occorrenza di vocaboli presenti nel \textit{Glossario} è
			marcata da una ``G'' maiuscola in pedice.
		\subsection{Riferimenti utili}
			\subsubsection{Riferimenti normativi}
    			\begin{itemize}
    				\item \textbf{Capitolato d'appalto}:\\
					\url{http://www.math.unipd.it/~tullio/IS-1/2016/Progetto/C6.pdf} (15/03/2017);
					\item \textbf{Norme di Progetto}: \normediprogettov;
    				\item \textbf{Verbali esterni}:
    					\begin{itemize}
    						\item Verbale incontro con \proponente\ in data 23/02/2017;
    						\item Verbale incontro con \proponente\ in data 17/03/2017;
    						\item Verbale incontro con il committente \vardanega\ in data .	% da inserire
    					\end{itemize}
				\end{itemize}
			\subsubsection{Riferimenti informativi}	
				\begin{itemize}
					\item \textbf{Slide dell'insegnamento di Ingegneria del Software
					1\ap{o} semestre}:
						\begin{itemize}
							\item Analisi dei requisiti;
							\item Diagrammi delle classi;
							\item Diagrammi dei package;
							\item Diagrammi di attività.
						\end{itemize}
						\url{http://www.math.unipd.it/~tullio/IS-1/2016/} (15/03/2017);
					\item \textbf{Glossario}: \glossariov.
				\end{itemize}
\end{document}
