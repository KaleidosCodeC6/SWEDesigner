\paragraph{Impostare visibilità classe}
\begin{itemize}
	\item \textbf{Attori}: Utente;
	\item \textbf{Descrizione}: L'utente può impostare i parametri di visibilità di una classe;
	\item \textbf{Precondizione}: L’utente si trova nella schermata dell’editor del diagramma delle classi e identifica la classe che desidera modificare;
	\item \textbf{Postcondizione}: I parametri di visibilità richiesti sono stati impostati;
\end{itemize}

\paragraph{Commentare classe}
\begin{itemize}
	\item \textbf{Attori}: Utente;
	\item \textbf{Descrizione}: L'utente può commentare una classe;
	\item \textbf{Precondizione}: L’utente si trova nella schermata dell’editor del diagramma delle classi e identifica la classe che desidera modificare;
	\item \textbf{Postcondizione}: Il commento relativo alla classe viene impostato;
\end{itemize}

\paragraph{Marchiare classe statica}
\begin{itemize}
	\item \textbf{Attori}: Utente;
	\item \textbf{Descrizione}: L'utente può impostare come "statica" una classe;
	\item \textbf{Precondizione}: L’utente si trova nella schermata dell’editor del diagramma delle classsi e identifica la classe che desidera definire "statica";
	\item \textbf{Postcondizione}: La classe viene marchiata come "statica";
\end{itemize}

\paragraph{Marchiare classe astratta}
\begin{itemize}
	\item \textbf{Attori}: Utente;
	\item \textbf{Descrizione}: L'utente può impostare come "astratta" una classe;
	\item \textbf{Precondizione}: L’utente si trova nella schermata dell’editor del diagramma delle classsi e identifica la classe che desidera definire "astratta";
	\item \textbf{Postcondizione}: La classe viene marchiata come "astratta";
\end{itemize}

\paragraph{Marchiare classe finale}
\begin{itemize}
	\item \textbf{Attori}: Utente;
	\item \textbf{Descrizione}: L'utente può impostare come "finale" una classe;
	\item \textbf{Precondizione}: L’utente si trova nella schermata dell’editor del diagramma delle classsi e identifica la classe che desidera definire "finale";
	\item \textbf{Postcondizione}: La classe viene marchiata come "finale";
\end{itemize}

\paragraph{Marchiare classe “frozen”}
\begin{itemize}
	\item \textbf{Attori}: Utente;
	\item \textbf{Descrizione}: L'utente può impostare come "frozen" una classe;
	\item \textbf{Precondizione}: L’utente si trova nella schermata dell’editor del diagramma delle classsi e identifica la classe che desidera definire "frozen";
	\item \textbf{Postcondizione}: La classe viene marchiata come "frozen";
\end{itemize}

\paragraph{Marchiare classe “readOnly”}
\begin{itemize}
	\item \textbf{Attori}: Utente;
	\item \textbf{Descrizione}: L'utente può impostare come "readOnly" una classe;
	\item \textbf{Precondizione}: L’utente si trova nella schermata dell’editor del diagramma delle classsi e identifica la classe che desidera definire "readOnly";
	\item \textbf{Postcondizione}: La classe viene marchiata come "readOnly";
\end{itemize}

\paragraph{Marchiare classe “enum”}
\begin{itemize}
	\item \textbf{Attori}: Utente;
	\item \textbf{Descrizione}: L'utente può impostare come "enum" una classe;
	\item \textbf{Precondizione}: L’utente si trova nella schermata dell’editor del diagramma delle classsi e identifica la classe che desidera definire "enum";
	\item \textbf{Postcondizione}: La classe viene marchiata come "enum";
\end{itemize}

\paragraph{Definire classe parametrica}
\begin{itemize}
	\item \textbf{Attori}: Utente;
	\item \textbf{Descrizione}: L'utente può impostare come parametrica una classe;
	\item \textbf{Precondizione}: L’utente si trova nella schermata dell’editor del diagramma delle classsi e identifica la classe che desidera definire parametrica;
	\item \textbf{Postcondizione}: La classe viene marchiata come parametrica;
\end{itemize}

\paragraph{Aggiungere attesa di segnale}
\begin{itemize}
	\item \textbf{Attori}: Utente;
	\item \textbf{Descrizione}: L'utente deve poter aggiungere ad un'attività un'attesa di segnale;
	\item \textbf{Precondizione}: L'utente si trova nella schermata dell'editor del diagramma delle attività ed è presente almeno un'attività;
	\item \textbf{Postcondizione}: L'utente ha modificato l'attività desiderata aggiungendo alla stessa un'attesa di segnale vuota;
\end{itemize}

\paragraph{Raffinare classe parametrica}
\begin{itemize}
	\item \textbf{Attori} Utente;
	\item \textbf{Descrizione}: L'utente si trova nella schermata dell'editor del diagramma delle classi e identifica la classe parametrica che desidera raffinare;
	\item \textbf{Precondizione}: L'utente si trova nella schermata dell'editor del diagramma delle attività ed è presente almeno un'attività con associata un'attesa di segnale;
	\item \textbf{Postcondizione}: La classe parametrica viene raffinata;
\end{itemize}