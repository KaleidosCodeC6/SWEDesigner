\documentclass[../AnalisiDeiRequisiti.tex]{subfiles}

	
		
% DA INSERIRE EVENTUALI REQUISITI, INCLUSIONI, ESTENSIONI, GENERALIZZAZIONI.
% DA INSERIRE RIFERIMENTI A CASI D'USO
% DA INSERIRE DIAGRAMMI CASI D'USO

\begin{document}
	\section{Casi d'uso }
	I seguenti casi d'uso sono frutto dell'analisi del capitolato, della discussione degli
	\analisti\ e degli incontri con	\proponente\ ed il committente \vardanega.
	Tali casi d'uso hanno quindi origine sia interna che esterna al gruppo.\\
	Le aspettative di esperienza utente derivano dalla sua conoscenza del
	linguaggio UML.\\
	Ciascun caso d'uso è classificato gerarchicamente con la seguente dicitura:
	\begin{center}
		UC[Codice del padre].[Codice identificativo]
	\end{center}
	Il codice identificativo può includere diversi livelli di gerarchia che saranno
	separati da un punto.
	\section{UC1: Editare il diagramma delle classi}
	\begin{itemize}
		\item \textbf{Attori: }Utente;
		\item \textbf{Scopo e descrizione: }L'utente ha avviato correttamente il programma e ha aperto un progetto.Ora l'utente può editare il diagramma delle classi ;
		
		\item \textbf{Flusso principale degli eventi: } ;
		\begin{itemize}
			\item L'utente può creare una nuova classe (UC1.1);
			\item L'utente può modificare una classe (UC1.2);
			\item L'utente può eliminare una classe (UC1.3);
			\item L'utente può definire una nuova relazione (UC1.4);
			\item L'utente può modificare una relazione (UC1.5);
			\item L'utente può eliminare una relazione (UC1.6);
			\item L'utente può creare una nuova interfaccia (UC1.7);
			\item L'utente può modificare una interfaccia (UC1.8);
			\item L'utente può eliminare una interfaccia (UC1.9);
			\item L'utente può cambiare layer di visualizzazione (UC1.10);
		\end{itemize}
		\item \textbf{Pre condizione: }L'utente ha avviato correttamente il programma e ha aperto un progetto;
		\item \textbf{Post condizione: }Il sistema apporta le modifiche desiderate al diagramma delle classi.
	\end{itemize}
	
	\subsection{UC1.2: Creare una nuova classe}
	\begin{itemize}
		\item \textbf{Attori:} Utente;
		\item \textbf{Scopo e descrizione:}L'utente può aggiungere una nuova classe vuota al diagramma delle classi;
		\item \textbf{Pre condizione:}Il programma è in esecuzione con un progetto aperto nel diagramma delle classi;
		\item \textbf{Post condizione:}Viene aggiunta una nuova classe al diagramma delle classi.
	\end{itemize}
	
	\subsection{UC1.3: Modificare una classe}
	\begin{itemize}
		\item \textbf{Attori:} Utente;
		\item \textbf{Scopo e descrizione:}L'utente vuole modificare uno dei vari attributi della classe selezionata ;
		\item \textbf{Flusso principale degli eventi: } ;
		\begin{itemize}
			\item L'utente può creare impostare il nome della classe (UC1.3.1);
			\item L'utente può aggiungere un campo dati a una classe (UC1.3.2);
			\item L'utente può eliminare un campo dati a una classe (UC1.3.3);
			\item L'utente può modificare un campo dati a una classe (UC1.3.4);
			\item L'utente può aggiungere un' operazione a una classe (UC1.3.5);
			\item L'utente può eliminare un' operazione a una classe (UC1.3.6);
			\item L'utente può modificare un'operazione a una classe (UC1.3.7);
		\end{itemize}
		\item \textbf{Pre condizione:}L'utente ha selezionato una classe da modificare all'interno di un progetto;
		\item \textbf{Post condizione:}Le varie modifiche decise dall'utente verranno applicate alla classe nel diagramma delle classi .
	\end{itemize}
	
	\subsubsection{UC1.3.1: Impostare il nome della classe}
	\begin{itemize}
		\item \textbf{Attori:} Utente;
		\item \textbf{Scopo e descrizione: }L'utente vuole impostare il nome di una classe  !!!!culoculoculo  appena creata culoculoculo!!!!!  ;
		\item \textbf{Pre condizione: }L'utente ha creato una classe all'interno del diagramma delle classi;
		\item \textbf{Post condizione: }Il nuovo nome deciso dall'utente viene impostato come nome della classe all'interno del diagramma delle classi.
	\end{itemize}
	
	\subsubsection{UC1.3.2: Aggiungere un campo dati alla classe}
	\begin{itemize}
		\item \textbf{Attori:} Utente;
		\item \textbf{Scopo e descrizione: }L'utente vuole aggiungere un nuovo campo dati a una classe;
		\item \textbf{Pre condizione: }L'utente ha selezionato una classe all'interno del diagramma delle classi alla quale vuole aggiungere un campo dati;
		\item \textbf{Post condizione: }Il campo dati viene aggiunto alla classe con i parametri decisi dall'utente.
	\end{itemize}
	
	\subsubsection{UC1.3.3: Eliminare un campo dati alla classe}
	\begin{itemize}
		\item \textbf{Attori:} Utente;
		\item \textbf{Scopo e descrizione: }L'utente vuole eliminare un campo dati all'interno di una classe;
		\item \textbf{Pre condizione: }L'utente ha selezionato il campo dati che vuole eliminare;
		\item \textbf{Post condizione: }Il campo dati viene rimosso dalla classe dopo eventuali avvisi nel caso ci siano dipendenze da controllare .
	\end{itemize}
	
	\subsubsection{UC1.3.4: Modificare un campo dati alla classe}
	\begin{itemize}
		\item \textbf{Attori:} Utente;
		\item \textbf{Scopo e descrizione: }L'utente vuole modificare un campo dati all'interno di una classe del diagramma delle classi;
		\item \textbf{Pre condizione: }L'utente ha selezionato il campo dati che vuole modificare all'interno di una classe;
		\item \textbf{Post condizione: }Le modifiche decise dall'utente vengono applicate al campo dati all'interno della classe.
	\end{itemize}
	
	\subsubsection{UC1.3.5: Aggiungere un' operazione alla classe}
	\begin{itemize}
		\item \textbf{Attori:} Utente;
		\item \textbf{Scopo e descrizione: }L'utente vuole aggiungere un' operazione a una classe;
		\item \textbf{Pre condizione: }L'utente ha selezionato una classe all'interno del diagramma delle classi alla quale vuole aggiungere un'operazione;
		\item \textbf{Post condizione: }L'operazione viene aggiunta alla classe con i parametri decisi dall'utente..
	\end{itemize}
	
	\subsubsection{UC1.3.6: Eliminare un' operazione  alla classe}
	\begin{itemize}
		\item \textbf{Attori:} Utente;
		\item \textbf{Scopo e descrizione: }L'utente vuole eliminare un'operazione all'interno di una classe;
		\item \textbf{Pre condizione: }L'utente ha selezionato L'operazione che vuole eliminare;
		\item \textbf{Post condizione: }L'operazione viene rimossa dalla classe dopo eventuali avvisi nel caso ci siano dipendenze da controllare .
	\end{itemize}
	
	\subsubsection{UC1.3.7: Modificare un' operazione alla classe}
	\begin{itemize}
		\item \textbf{Attori:} Utente;
		\item \textbf{Scopo e descrizione: }L'utente vuole modificare un' operazione all'interno di una classe del diagramma delle classi;
		\item \textbf{Pre condizione: }L'utente ha selezionato l'operazione che vuole modificare all'interno di una classe;
		\item \textbf{Post condizione: }Le modifiche decise dall'utente vengono applicate all'operazione all'interno della classe.
	\end{itemize}
	
	%parte 2
	
	\subsubsection{Impostare visibilità classe}
	\begin{itemize}
		\item \textbf{Attori}: Utente;
		\item \textbf{Descrizione}: L'utente può impostare i parametri di visibilità di una classe;
		\item \textbf{Precondizione}: L’utente si trova nella schermata dell’editor del diagramma delle classi e identifica la classe che desidera modificare;
		\item \textbf{Postcondizione}: I parametri di visibilità richiesti sono stati impostati;
	\end{itemize}
	
	\subsubsection{Commentare classe}
	\begin{itemize}
		\item \textbf{Attori}: Utente;
		\item \textbf{Descrizione}: L'utente può commentare una classe;
		\item \textbf{Precondizione}: L’utente si trova nella schermata dell’editor del diagramma delle classi e identifica la classe che desidera modificare;
		\item \textbf{Postcondizione}: Il commento relativo alla classe viene impostato;
	\end{itemize}
	
	\subsubsection{Marchiare classe statica}
	\begin{itemize}
		\item \textbf{Attori}: Utente;
		\item \textbf{Descrizione}: L'utente può impostare come "statica" una classe;
		\item \textbf{Precondizione}: L’utente si trova nella schermata dell’editor del diagramma delle classsi e identifica la classe che desidera definire "statica";
		\item \textbf{Postcondizione}: La classe viene marchiata come "statica";
	\end{itemize}
	
	\subsubsection{Marchiare classe astratta}
	\begin{itemize}
		\item \textbf{Attori}: Utente;
		\item \textbf{Descrizione}: L'utente può impostare come "astratta" una classe;
		\item \textbf{Precondizione}: L’utente si trova nella schermata dell’editor del diagramma delle classsi e identifica la classe che desidera definire "astratta";
		\item \textbf{Postcondizione}: La classe viene marchiata come "astratta";
	\end{itemize}
	
	\subsubsection{Marchiare classe finale}
	\begin{itemize}
		\item \textbf{Attori}: Utente;
		\item \textbf{Descrizione}: L'utente può impostare come "finale" una classe;
		\item \textbf{Precondizione}: L’utente si trova nella schermata dell’editor del diagramma delle classsi e identifica la classe che desidera definire "finale";
		\item \textbf{Postcondizione}: La classe viene marchiata come "finale";
	\end{itemize}
	
	\subsubsection{Marchiare classe “frozen”}
	\begin{itemize}
		\item \textbf{Attori}: Utente;
		\item \textbf{Descrizione}: L'utente può impostare come "frozen" una classe;
		\item \textbf{Precondizione}: L’utente si trova nella schermata dell’editor del diagramma delle classsi e identifica la classe che desidera definire "frozen";
		\item \textbf{Postcondizione}: La classe viene marchiata come "frozen";
	\end{itemize}
	
	\subsubsection{Marchiare classe “readOnly”}
	\begin{itemize}
		\item \textbf{Attori}: Utente;
		\item \textbf{Descrizione}: L'utente può impostare come "readOnly" una classe;
		\item \textbf{Precondizione}: L’utente si trova nella schermata dell’editor del diagramma delle classsi e identifica la classe che desidera definire "readOnly";
		\item \textbf{Postcondizione}: La classe viene marchiata come "readOnly";
	\end{itemize}
	
	\subsubsection{Marchiare classe “enum”}
	\begin{itemize}
		\item \textbf{Attori}: Utente;
		\item \textbf{Descrizione}: L'utente può impostare come "enum" una classe;
		\item \textbf{Precondizione}: L’utente si trova nella schermata dell’editor del diagramma delle classsi e identifica la classe che desidera definire "enum";
		\item \textbf{Postcondizione}: La classe viene marchiata come "enum";
	\end{itemize}
	
	\subsubsection{Definire classe parametrica}
	\begin{itemize}
		\item \textbf{Attori}: Utente;
		\item \textbf{Descrizione}: L'utente può impostare come parametrica una classe;
		\item \textbf{Precondizione}: L’utente si trova nella schermata dell’editor del diagramma delle classsi e identifica la classe che desidera definire parametrica;
		\item \textbf{Postcondizione}: La classe viene marchiata come parametrica;
	\end{itemize}
	
	\subsubsection{Aggiungere attesa di segnale}
	\begin{itemize}
		\item \textbf{Attori}: Utente;
		\item \textbf{Descrizione}: L'utente deve poter aggiungere ad un'attività un'attesa di segnale;
		\item \textbf{Precondizione}: L'utente si trova nella schermata dell'editor del diagramma delle attività ed è presente almeno un'attività;
		\item \textbf{Postcondizione}: L'utente ha modificato l'attività desiderata aggiungendo alla stessa un'attesa di segnale vuota;
	\end{itemize}
	
	\subsubsection{Raffinare classe parametrica}
	\begin{itemize}
		\item \textbf{Attori} Utente;
		\item \textbf{Descrizione}: L'utente si trova nella schermata dell'editor del diagramma delle classi e identifica la classe parametrica che desidera raffinare;
		\item \textbf{Precondizione}: L'utente si trova nella schermata dell'editor del diagramma delle attività ed è presente almeno un'attività con associata un'attesa di segnale;
		\item \textbf{Postcondizione}: La classe parametrica viene raffinata;
	\end{itemize}	
		
		
	%parte 3	
		
		\subsubsection{Caso d'uso UC: Innestare una classe interna}
		\begin{itemize}
			\item\textbf{Attori}: Utente.
			\item\textbf{Descrizione}: L'utente vuole innestare una classe all'interno di un'altra classe.
			\item\textbf{Precondizione}: Sono presenti due classi distinte e non innestate l'una nell'altra.
			\item\textbf{Postcondizione}: Nell'editor del diagramma delle classi le due classi sono visualizzate una innestata dentro l'altra.	
		\end{itemize}
		
		\subsubsection{Caso d'uso UC: Impostare l'importanza della classe}
		\begin{itemize}
			\item\textbf{Attori}: Utente
			\item\textbf{Descrizione}: L'utente vuole impostare l'importanza di una classe.
			\item\textbf{Precondizione}: Esiste una classe e l'utente vuole modificare l'importanza.
			\item\textbf{Postcondizione}:L'importanza è settata sul valore che l'utente desidera.
		\end{itemize}
		
		\subsection{Caso d'uso UC: Eliminare una classe}
		\begin{itemize}
			\item\textbf{Attori}: Utente.
			\item\textbf{Descrizione}: L'utente vuole eliminare una classe.
			\item\textbf{Precondizione}: Esiste una classe che l'utente desidera eliminare.
			\item\textbf{Postcondizione}: La classe non è più visualizzata nell'editor del diagramma delle classi.
		\end{itemize}
		
		\subsection{Caso d'uso UC: Definire una relazione}
		\begin{itemize}
			\item\textbf{Attori}: Utente.
			\item\textbf{Descrizione}: L'utente vuole definire una relazione.
			\item\textbf{Precondizione}: Sono presenti due classi e l'utente desiderano che presentino una relazione l'una dall'altra.
			\item\textbf{Scenario principale}: 
			\begin{itemize}
				\item L'utente vuole definire la dipendenza tra due classi (UC).
				\item L'utente vuole definire l'associazione tra due classi (UC).
				\item L'utente vuole definire l'ereditarietà tra due classi (UC).
				\item L'utente vuole definire l'aggregazione tra due classi (UC).
				\item L'utente vuole definire la composizione tra due classi (UC).
			\end{itemize}
			\item\textbf{Postcondizione}: Le due classi presentano una relazione.
			
		\end{itemize}
		
		\subsection{Caso d'uso UC: Modificare una relazione}
		\begin{itemize}
			\item\textbf{Attori}: Utente.
			\item\textbf{Descrizione}: L'utente vuole modificare una relazione tra due classi.
			\item\textbf{Precondizione}: È presente una relazione che l'utente vuole modificare.
			\item\textbf{Scenario principale}: 
			\begin{itemize}
				\item L'utente vuole modificare la dipendenza tra due classi (UC).
				\item L'utente vuole modificare l'associazione tra due classi (UC).
				\item L'utente vuole modificare l'ereditarietà tra due classi (UC).
				\item L'utente vuole modificare l'aggregazione tra due classi (UC).
				\item L'utente vuole modificare la composizione tra due classi (UC).
			\end{itemize}
			\item\textbf{Postcondizione}: La relazione viene modificata.
		\end{itemize}
		
		\subsection{Caso d'uso UC: Eliminare una relazione}
		\begin{itemize}
			\item\textbf{Attori}: Utente.
			\item\textbf{Descrizione}: L'utente vuole eliminare una relazione.
			\item\textbf{Precondizione}: Esiste una relazione che l'utente desidera eliminare.
			\item\textbf{Scenario principale}: 
			\begin{itemize}
				\item L'utente vuole eliminare la dipendenza tra due classi (UC).
				\item L'utente vuole eliminare l'associazione tra due classi (UC).
				\item L'utente vuole eliminare l'ereditarietà tra due classi (UC).
				\item L'utente vuole eliminare l'aggregazione tra due classi (UC).
				\item L'utente vuole eliminare la composizione tra due classi (UC).
			\end{itemize}
			\item\textbf{Postcondizione}: La relazione viene eliminata.
		\end{itemize}
		
		\subsubsection{Caso d'uso UC: Definire dipendenza tra classi}
		\begin{itemize}
			\item\textbf{Attori}: Utente.
			\item\textbf{Descrizione}: L'utente vuole definire la dipendenza tra due classi.
			\item\textbf{Precondizione}: Sono presenti due classi e l'utente vuole evidenziarne la dipendenza.
			\item\textbf{Postcondizione}: La dipendenza tra le due classi è stata definita.
		\end{itemize}
		
		\subsubsection{Caso d'uso 5: Modificare dipendenza tra classi}
		\begin{itemize}
			\item\textbf{Attori}: Utente.
			\item\textbf{Descrizione}: L'utente vuole definire la dipendenza tra due classi.
			\item\textbf{Precondizione}: Sono presenti due classi che presentano una dipendenza l'una dall'altra.
			\item\textbf{Postcondizione}: La dipendenza tra le due classi è stata modificata nel modo che l'utente desidera.
		\end{itemize}
		
		\subsubsection{Caso d'uso UC: Eliminare dipendenza tra classi}
		\begin{itemize}
			\item\textbf{Attori}: Utente.
			\item\textbf{Descrizione}: L'utente vuole eliminare una dipendenza tra classi.
			\item\textbf{Precondizione}: Esiste una dipendenza tra classi che l'utente desidera eliminare.
			\item\textbf{Postcondizione}: La dipendenza tra classi viene eliminata.
		\end{itemize}
		
		\subsubsection{Caso d'uso UC: Definire associazione tra classi}
		\begin{itemize}
			\item\textbf{Attori}: Utente.
			\item\textbf{Descrizione}: L'utente vuole definire un'associazione tra due classi.
			\item\textbf{Precondizione}: Sono presenti due classi e l'utente vuole evidenziarne l'associazione.
			\item\textbf{Postcondizione}: L'associazione tra le due classi è stata definita.
		\end{itemize}
		
		\subsubsection{Caso d'uso UC: Modificare associazione tra classi}
		\begin{itemize}
			\item\textbf{Attori}: Utente.
			\item\textbf{Descrizione}: L'utente vuole eliminare un'associazione tra classi.
			\item\textbf{Precondizione}: Sono presenti due classi e l'utente vuole modificarne l'associazione.
			\item\textbf{Postcondizione}: L'associazione tra le due classi è stata modificata.
		\end{itemize}
		
		\subsubsection{Caso d'uso UC: Eliminare associazione tra classi}
		\begin{itemize}
			\item\textbf{Attori}: Utente.
			\item\textbf{Descrizione}: L'utente vuole eliminare un'associazione tra classi.
			\item\textbf{Precondizione}: Esiste un'associazione tra classi che l'utente desidera eliminare.
			\item\textbf{Postcondizione}: L'associazione tra classi viene eliminata.
		\end{itemize}
		
		\subsubsection{Caso d'uso UC: Definire ereditarietà tra classi}
		\begin{itemize}
			\item\textbf{Attori}: Utente.
			\item\textbf{Descrizione}: L'utente vuole definire un vincolo di ereditarietà tra due classi.
			\item\textbf{Precondizione}: Sono presenti due classi e l'utente vuole evidenziarne il vincolo di ereditarietà.
			\item\textbf{Postcondizione}: L'ereditarietà tra le due classi è stata definita.
		\end{itemize}
		
		\subsubsection{Caso d'uso UC: Modificare ereditarietà tra classi}
		\begin{itemize}
			\item\textbf{Attori}: Utente.
			\item\textbf{Descrizione}: L'utente vuole modificare un vincolo di ereditarietà tra due classi.
			\item\textbf{Precondizione}: Sono presenti due classi e l'utente vuole modificarne il vincolo di ereditarietà.
			\item\textbf{Postcondizione}: L'ereditarietà tra le due classi è stata modificata.
		\end{itemize}
		
		\subsubsection{Caso d'uso UC: Eliminare ereditarietà tra classi}
		\begin{itemize}
			\item\textbf{Attori}: Utente.
			\item\textbf{Descrizione}: L'utente vuole eliminare un'ereditarietà tra classi
			\item\textbf{Precondizione}: Esiste un'ereditarietà tra classi che l'utente desidera eliminare.
			\item\textbf{Postcondizione}: L'ereditarietà tra classi viene eliminata.
		\end{itemize}
		
		\subsubsection{Caso d'uso UC: Definire aggregazione tra classi}
		\begin{itemize}
			\item\textbf{Attori}: Utente.
			\item\textbf{Descrizione}: L'utente vuole definire una relazione di aggregazione tra due classi.
			\item\textbf{Precondizione}: Sono presenti due classi e l'utente vuole evidenziarne l'aggregazione.
			\item\textbf{Postcondizione}: La relazione di aggregazione tra le due classi è stata definita.
		\end{itemize}
		
		\subsubsection{Caso d'uso UC: Modificare aggregazione tra classi}
		\begin{itemize}
			\item\textbf{Attori}: Utente.
			\item\textbf{Descrizione}: L'utente vuole modificare una relazione di aggregazione tra due classi.
			\item\textbf{Precondizione}: Sono presenti due classi e l'utente vuole modificarne l'aggregazione.
			\item\textbf{Postcondizione}: La relazione di aggregazione tra le due classi è stata modificata.
		\end{itemize}
		
		\subsubsection{Caso d'uso UC: Eliminare aggregazione tra classi}
		\begin{itemize}
			\item\textbf{Attori}: Utente.
			\item\textbf{Descrizione}: L'utente vuole eliminare una relazione di aggregazione tra classi.
			\item\textbf{Precondizione}: Esiste una relazine di aggregazione tra classi che l'utente desidera eliminare.
			\item\textbf{Postcondizione}: La relazione di aggregazione tra classi viene eliminata.
		\end{itemize}
		
		\subsubsection{Caso d'uso UC: Definire composizione tra classi}
		\begin{itemize}
			\item\textbf{Attori}: Utente.
			\item\textbf{Descrizione}: L'utente vuole definire una composizione tra due classi.
			\item\textbf{Precondizione}: Sono presenti due classi e l'utente vuole evidenziarne la composizione.
			\item\textbf{Postcondizione}: La relazione di composizione tra le due classi è stata definita.
		\end{itemize}
		
		\subsubsection{Caso d'uso UC:  Modificare composizione tra classi}
		\begin{itemize}
			\item\textbf{Attori}: Utente.
			\item\textbf{Descrizione}: L'utente vuole modificare una relazione di composizione tra classi.
			\item\textbf{Precondizione}: Esiste una relazine di composizione tra classi che l'utente desidera modificare.
			\item\textbf{Postcondizione}: La relazione di composizione tra classi viene modificata.
		\end{itemize}
		
		\subsubsection{Caso d'uso UC: Eliminare composizione tra classi}
		\begin{itemize}
			\item\textbf{Attori}: Utente.
			\item\textbf{Descrizione}: L'utente vuole eliminare una relazione di composizione tra classi.
			\item\textbf{Precondizione}: Esiste una relazine di composizione tra classi che l'utente desidera eliminare.
			\item\textbf{Postcondizione}: La relazione di composizione tra classi viene eliminata.
		\end{itemize}
		
		%%%%%%%%%%%%%%%%%%%%%%%%%%SEZIONE INTERFACCIA%%%%%%%%%%%%%%%%%%%%%%%%
		
		\subsection{Caso d'uso UC: Creare un'interfaccia}
		\begin{itemize}
			\item\textbf{Attori}: Utente.
			\item\textbf{Descrizione}: L'utente vuole creare un'interfaccia.
			\item\textbf{Precondizione}: Il sistema è pronto alla creazione di un'interfaccia, l'utente desidera creare un'interfaccia.
			\item\textbf{Postcondizione}: Nell'editor del diagramma delle classi l'interfaccia è correttamente visualizzato il diagramma nel quale è stata creata l'interfaccia.
		\end{itemize}
			
			%parte 4
				
		\subsection{Caso d'uso UC : Modificare un'interfaccia}
			\begin{itemize}
				\item \textbf{Attori}: Utente;
				\item \textbf{Descrizione}: L'utente sceglie di modificare un'interfaccia
				all'interno dell'editor del diagramma delle classi;
				\item \textbf{Precondizione}: Nell'editor del diagramma delle classi del
				sistema è stata selezionata un'interfaccia che l'utente desidera modificare;
				\item \textbf{Scenario principale degli eventi}:
					\begin{itemize}
						\item L'utente può rinominare l'interfaccia (UC);
						\item L'utente può aggiungere un attributo (UC);
						\item L'utente può modificare un attributo (UC);
						\item L'utente può rimuovere un attributo (UC);
						\item L'utente può aggiungere un'operazione (UC);
						\item L'utente può modificare un'operazione (UC);
						\item L'utente può rimuovere un'operazione (UC).
					\end{itemize}
				\item \textbf{Scenari alternativi}: Viene annullata la modifica, il sistema
				rimane nello stato precedente al tentativo di modifica;
				\item \textbf{Postcondizione}: Nell'editor del diagramma delle classi del
				sistema è visualizzato il diagramma dove sono state apportate le modifiche
				all'interfaccia.
			\end{itemize}
		\subsubsection{Caso d'uso UC : Rinominare un'interfaccia}
			\begin{itemize}
				\item \textbf{Attori}: Utente;
				\item \textbf{Descrizione}: L'utente cambia il nome dell'interfaccia;
				\item \textbf{Precondizione}: Il sistema è in attesa che l'utente inserisca
				una stringa per rinominare l'interfaccia;
				\item \textbf{Postcondizione}: Nell'editor del diagramma delle classi del
				sistema è visualizzato il diagramma dove è stato cambiato il nome
				all'interfaccia.
			\end{itemize}
		\subsubsection{Caso d'uso UC : Aggiungere un attributo}
			\begin{itemize}
				\item \textbf{Attori}: Utente;
				\item \textbf{Descrizione}: L'utente ha scelto di aggiungere un attributo
				all'interfaccia. L'utente deve definire il nuovo attributo;
				\item \textbf{Precondizione}: L'utente desidera aggiungere un attributo
				all'interfaccia selezionata dall'editor del diagramma delle classi del
				sistema. Il sistema è pronto ad aggiungere un nuovo attributo;
				\item \textbf{Scenario principale degli eventi}:
					\begin{itemize}
						\item L'utente può impostare la visibilità dell'attributo (UC);
						\item L'utente può definire il nome dell'attributo (UC);
						\item L'utente può definire tipo dell'attributo (UC).
					\end{itemize}
				\item \textbf{Scenari alternativi}: Viene annullata la modifica, il sistema
				rimane nello stato precedente al tentativo di modifica;
				\item \textbf{Postcondizione}: Nell'editor del diagramma delle classi del
				sistema è visualizzato il diagramma dove è stato aggiunto il nuovo attributo.
			\end{itemize}
		\paragraph{Caso d'uso UC : Impostare la visibilità dell'attributo}
			\begin{itemize}
				\item \textbf{Attori}: Utente;
				\item \textbf{Descrizione}: L'utente può impostare la visibilità
				dell'attributo;
				\item \textbf{Precondizione}: Il sistema è in attesa che l'utente selezioni il
				tipo di visibilità da impostare all'attributo;
				\item \textbf{Postcondizione}: Il sistema ha impostato la visibilità
				dell'attributo.
			\end{itemize}
		\paragraph{Caso d'uso UC : Definire il nome dell'attributo}
			\begin{itemize}
				\item \textbf{Attori}: Utente;
				\item \textbf{Descrizione}: L'utente può definire il nome dell'attributo;
				\item \textbf{Precondizione}: Il sistema è in attesa che l'utente inserisca
				una stringa per rinominare l'attributo;
				\item \textbf{Postcondizione}: Il sistema ha impostato il nome
				dell'attributo.
			\end{itemize}
		\paragraph{Caso d'uso UC : Definire il tipo dell'attributo}
			\begin{itemize}
				\item \textbf{Attori}: Utente;
				\item \textbf{Descrizione}: L'utente può definire il tipo dell'attributo;
				\item \textbf{Precondizione}: Il sistema è in attesa che l'utente inserisca
				il tipo dell'attributo;
				\item \textbf{Postcondizione}: Il sistema ha impostato il tipo dell'attributo.
			\end{itemize}
		\subsubsection{Caso d'uso UC : Modificare un attributo}
			\begin{itemize}
				\item \textbf{Attori}: Utente;
				\item \textbf{Descrizione}: L'utente ha scelto di modificare un attributo
				dell'interfaccia. L'utente deve selezionare l'operazione;
				\item \textbf{Precondizione}: L'utente desidera modificare un attributo
				dell'interfaccia selezionata dall'editor del diagramma delle classi del
				sistema. È stato selezionato l'attributo da modificare;
				\item \textbf{Scenario principale degli eventi}:
					\begin{itemize}
						\item L'utente può impostare la visibilità (UC Sopra);
						\item L'utente può definire il nome dell'attributo (UC Sopra);
						\item L'utente può definire tipo dell'attributo (UC Sopra).
					\end{itemize}
				\item \textbf{Scenari alternativi}: Viene annullata la modifica, il sistema
				rimane nello stato precedente al tentativo di modifica;
				\item \textbf{Postcondizione}: Nell'editor del diagramma delle classi del
				sistema è visualizzato il diagramma dove è stato modificato l'attributo.
			\end{itemize}
		\subsubsection{Caso d'uso UC : Rimuovere un attributo}
			\begin{itemize}
				\item \textbf{Attori}: Utente;
				\item \textbf{Descrizione}: L'utente può rimuovere un attributo
				dell'interfaccia. L'utente deve selezionare l'attributo;
				\item \textbf{Precondizione}: L'utente desidera rimuovere un attributo
				dell'interfaccia selezionata dall'editor del diagramma delle classi del
				sistema. È stato selezionato l'attributo da rimuovere;
				\item \textbf{Postcondizione}: Nell'editor del diagramma delle classi del
				sistema è visualizzato il diagramma dove è stato rimosso l'attributo.
			\end{itemize}
		\subsubsection{Caso d'uso UC : Aggiungere un'operazione}
			\begin{itemize}
				\item \textbf{Attori}: Utente;
				\item \textbf{Descrizione}: L'utente ha scelto di aggiungere un'operazione
				all'interfaccia. L'utente deve definire la nuova operazione;
				\item \textbf{Precondizione}: L'utente desidera aggiungere un'operazione
				all'interfaccia selezionata dall'editor del diagramma delle classi del
				sistema. Il sistema è pronto ad aggiungere una nuova operazione;
				\item \textbf{Scenario principale degli eventi}:
					\begin{itemize}
						\item L'utente può impostare la visibilità dell'operazione (UC);
						\item L'utente può definire il nome dell'operazione (UC);
						\item L'utente può definire la lista parametri dell'operazione (UC);
						\item L'utente può definire il tipo di ritorno dell'operazione (UC);
						\item L'utente può definire proprietà aggiuntive dell'operazione (UC).
					\end{itemize}
				\item \textbf{Scenari alternativi}: Viene annullata la modifica, il sistema
				rimane nello stato precedente al tentativo di modifica;
				\item \textbf{Postcondizione}: Nell'editor del diagramma delle classi del
				sistema è visualizzato il diagramma dove è stata aggiunta la nuova operazione.
			\end{itemize}
		\paragraph{Caso d'uso UC : Impostare la visibilità dell'operazione}
			\begin{itemize}
				\item \textbf{Attori}: Utente;
				\item \textbf{Descrizione}: L'utente può impostare la visibilità
				dell'operazione;
				\item \textbf{Precondizione}: Il sistema è in attesa che l'utente selezioni il
				tipo di visibilità da impostare all'operazione;
				\item \textbf{Postcondizione}: Il sistema ha impostato la visibilità
				dell'operazione.
			\end{itemize}
		\paragraph{Caso d'uso UC : Definire il nome dell'operazione}
			\begin{itemize}
				\item \textbf{Attori}: Utente;
				\item \textbf{Descrizione}: L'utente può definire il nome dell'operazione;
				\item \textbf{Precondizione}: Il sistema è in attesa che l'utente inserisca
				una stringa per rinominare l'operazione;
				\item \textbf{Postcondizione}: Il sistema ha impostato il nome
				dell'operazione.
			\end{itemize}
		\paragraph{Caso d'uso UC : Definire la lista parametri dell'operazione}
			\begin{itemize}
				\item \textbf{Attori}: Utente;
				\item \textbf{Descrizione}: L'utente può definire la lista parametri
				dell'operazione;
				\item \textbf{Precondizione}: Il sistema è in attesa che l'utente definisca
				la lista dei parametri dell'operazione;
				\item \textbf{Scenario principale degli eventi}:
					\begin{itemize}
						\item L'utente può aggiungere un parametro (UC);
						\item L'utente può modificare un parametro (UC);
						\item L'utente può rimuovere un parametro (UC).
					\end{itemize}
				\item \textbf{Postcondizione}: Il sistema ha impostato la lista parametri
				dell'operazione.
			\end{itemize}
		\subsubsection{Caso d'uso UC : Aggiungere un parametro}
			\begin{itemize}
				\item \textbf{Attori}: Utente;
				\item \textbf{Descrizione}: L'utente può aggiungere un parametro alla lista
				parametri dell'operazione;
				\item \textbf{Precondizione}: Il sistema è pronto per definire il parametro;
				\item \textbf{Scenario principale degli eventi}:
					\begin{itemize}
						\item L'utente può definire la direzione del parametro (UC);
						\item L'utente può definire il nome del parametro (UC);
						\item L'utente può definire il tipo del parametro (UC);
						\item L'utente può definire il valore di default del parametro (UC).
					\end{itemize}
				\item \textbf{Postcondizione}: Il sistema ha aggiunto il parametro alla lista.
			\end{itemize}
		\paragraph{Caso d'uso UC : Definire la direzione del parametro}
			\begin{itemize}
				\item \textbf{Attori}: Utente;
				\item \textbf{Descrizione}: L'utente può definire la direzione di un parametro
				della lista parametri dell'operazione;
				\item \textbf{Precondizione}: Il sistema è in attesa che l'utente imposti
				la direzione del parametro;
				\item \textbf{Postcondizione}: Il sistema ha impostato la direzione al
				parametro della lista.
			\end{itemize}
		\paragraph{Caso d'uso UC : Definire il nome del parametro}
			\begin{itemize}
				\item \textbf{Attori}: Utente;
				\item \textbf{Descrizione}: L'utente può definire il nome di un parametro
				della lista parametri dell'operazione;
				\item \textbf{Precondizione}: Il sistema è in attesa che l'utente inserisca
				il nome del parametro;
				\item \textbf{Postcondizione}: Il sistema ha impostato il nome al parametro
				della lista.
			\end{itemize}
		\paragraph{Caso d'uso UC : Definire il tipo del parametro}
			\begin{itemize}
				\item \textbf{Attori}: Utente;
				\item \textbf{Descrizione}: L'utente può definire il tipo di un parametro
				della lista parametri dell'operazione;
				\item \textbf{Precondizione}: Il sistema è in attesa che l'utente inserisca
				il tipo del parametro;
				\item \textbf{Postcondizione}: Il sistema ha impostato il tipo al parametro
				della lista.
			\end{itemize}
		\paragraph{Caso d'uso UC : Definire il valore di default del parametro}
			\begin{itemize}
				\item \textbf{Attori}: Utente;
				\item \textbf{Descrizione}: L'utente può definire il valore di default di un
				parametro della lista parametri dell'operazione;
				\item \textbf{Precondizione}: Il sistema è in attesa che l'utente imposti
				il valore di default del parametro;
				\item \textbf{Postcondizione}: Il sistema ha impostato il valore di default al
				parametro della lista.
			\end{itemize}
		\paragraph{Caso d'uso UC : Modificare un parametro}
			\begin{itemize}
				\item \textbf{Attori}: Utente;
				\item \textbf{Descrizione}: L'utente può modificare un parametro della lista
				parametri dell'operazione;
				\item \textbf{Precondizione}: Il sistema è pronto per modificare il parametro;
				\item \textbf{Scenario principale degli eventi}:
					\begin{itemize}
						\item L'utente può definire la direzione del parametro (UC Sopra);
						\item L'utente può definire il nome del parametro (UC Sopra);
						\item L'utente può definire il tipo del parametro (UC Sopra);
						\item L'utente può definire il valore di default del parametro (UC Sopra).
					\end{itemize}
				\item \textbf{Postcondizione}: Il sistema ha modificato il parametro della
				lista.
			\end{itemize}
		\paragraph{Caso d'uso UC : Rimuovere un parametro}
			\begin{itemize}
				\item \textbf{Attori}: Utente;
				\item \textbf{Descrizione}: L'utente può rimuovere un parametro della lista
				parametri dell'operazione;
				\item \textbf{Precondizione}: L'utente ha selezionato il parametro da
				rimuovere;
				\item \textbf{Postcondizione}: Il sistema ha rimosso il parametro dalla lista.
			\end{itemize}
		\paragraph{Caso d'uso UC : Definire il tipo di ritorno dell'operazione}
			\begin{itemize}
				\item \textbf{Attori}: Utente;
				\item \textbf{Descrizione}: L'utente può inserire il tipo di ritorno
				dell'operazione;
				\item \textbf{Precondizione}: Il sistema è in attesa che l'utente inserisca
				il tipo di ritorno;
				\item \textbf{Postcondizione}: Il sistema ha impostato il tipo di ritorno
				all'operazione.
			\end{itemize}
		\paragraph{Caso d'uso UC : Definire proprietà aggiuntive dell'operazione}
			\begin{itemize}
				\item \textbf{Attori}: Utente;
				\item \textbf{Descrizione}: L'utente può impostare proprietà aggiuntive
				all'operazione;
				\item \textbf{Precondizione}: Il sistema è pronto per ricevere proprietà
				aggiuntive;
				\item \textbf{Postcondizione}: Il sistema ha impostato le proprietà
				aggiuntive, scritte dall'utente, all'operazione.
			\end{itemize}
		\subsubsection{Caso d'uso UC : Modificare un'operazione}
			\begin{itemize}
				\item \textbf{Attori}: Utente;
				\item \textbf{Descrizione}: L'utente ha scelto di modificare un'operazione
				dell'interfaccia. L'utente deve selezionare l'operazione;
				\item \textbf{Precondizione}: L'utente desidera modificare un'operazione
				dell'interfaccia selezionata dall'editor del diagramma delle classi del
				sistema. È stata selezionata l'operazione da modificare;
				\item \textbf{Scenario principale degli eventi}:
					\begin{itemize}
						\item L'utente può impostare la visibilità (UC Sopra);
						\item L'utente può definire il nome dell'operazione (UC Sopra);
						\item L'utente può definire la lista parametri dell'operazione (UC Sopra);
						\item L'utente può definire il tipo di ritorno dell'operazione (UC Sopra);
						\item L'utente può definire proprietà aggiuntive dell'operazione (UC Sopra).
					\end{itemize}
				\item \textbf{Scenari alternativi}: Viene annullata la modifica, il sistema
				rimane nello stato precedente al tentativo di modifica;
				\item \textbf{Postcondizione}: Nell'editor del diagramma delle classi del
				sistema è visualizzato il diagramma dove è stata modificata l'operazione.
			\end{itemize}
		\subsubsection{Caso d'uso UC : Rimuovere un'operazione}
			\begin{itemize}
				\item \textbf{Attori}: Utente;
				\item \textbf{Descrizione}: L'utente può rimuovere un'operazione
				dell'interfaccia. L'utente deve selezionare l'operazione;
				\item \textbf{Precondizione}: L'utente desidera rimuovere un'operazione
				dell'interfaccia selezionata dall'editor del diagramma delle classi del
				sistema. È stata selezionata l'operazione da rimuovere;
				\item \textbf{Postcondizione}: Nell'editor del diagramma delle classi del
				sistema è visualizzato il diagramma dove è stata rimossa l'operazione.
			\end{itemize}
		\subsubsection{Caso d'uso UC : Rimuovere un'interfaccia}
			\begin{itemize}
				\item \textbf{Attori}: Utente;
				\item \textbf{Descrizione}: L'utente può rimuovere un'interfaccia nell'editor
				del diagramma delle classi. L'utente deve selezionare l'interfaccia;
				\item \textbf{Precondizione}: L'utente desidera rimuovere un'interfaccia
				selezionata dall'editor del diagramma delle classi del sistema;
				\item \textbf{Postcondizione}: Nell'editor del diagramma delle classi del
				sistema è visualizzato il diagramma dove è stata rimossa l'interfaccia.
			\end{itemize}
		\subsubsection{Caso d'uso UC : Definire la realizzazione di un'interfaccia}
			\begin{itemize}
				\item \textbf{Attori}: Utente;
				\item \textbf{Descrizione}: L'utente sceglie di definire un'associazione di
				realizzazione tra una classe ed un'interfaccia;
				\item \textbf{Precondizione}: Il sistema è pronto a creare l'associazione di
				realizzazione che l'utente desidera definire;
				\item \textbf{Scenario principale degli eventi}:
					\begin{itemize}
						\item L'utente può selezionare un'interfaccia (UC);
						\item L'utente può selezionare una classe (UC).
					\end{itemize}
				\item \textbf{Scenari alternativi}: La definizione della realizzazione di
				un'interfaccia viene annullata, il sistema rimane nello stato precedente al
				tentativo di modifica;
				\item \textbf{Postcondizione}: Nell'editor del diagramma delle classi del
				sistema è visualizzato il diagramma dove è stata definita la realizzazione.
			\end{itemize}
		\paragraph{Caso d'uso UC : Selezionare un'interfaccia}
			\begin{itemize}
				\item \textbf{Attori}: Utente;
				\item \textbf{Descrizione}: L'utente seleziona l'interfaccia di interesse;
				\item \textbf{Precondizione}: Il sistema mostra il diagramma delle classi;
				\item \textbf{Postcondizione}: Il sistema evidenzia l'interfaccia selezionata
				dall'utente.
			\end{itemize}
		\paragraph{Caso d'uso UC : Selezionare una classe}
			\begin{itemize}
				\item \textbf{Attori}: Utente;
				\item \textbf{Descrizione}: L'utente seleziona la classe di interesse;
				\item \textbf{Precondizione}: Il sistema mostra il diagramma delle classi;
				\item \textbf{Postcondizione}: Il sistema evidenzia la classe selezionata
				dall'utente.
			\end{itemize}
		\paragraph{Caso d'uso UC : Rimuovere la realizzazione di un'interfaccia}
			\begin{itemize}
				\item \textbf{Attori}: Utente;
				\item \textbf{Descrizione}: L'utente sceglie di rimuovere un'associazione di
				realizzazione tra una classe ed un'interfaccia;
				\item \textbf{Precondizione}: Il sistema è pronto a rimuovere l'associazione
				di realizzazione che l'utente desidera;
				\item \textbf{Postcondizione}: Nell'editor del diagramma delle classi del
				sistema è visualizzato il diagramma dove è stata rimossa la realizzazione.
			\end{itemize}
			
%%%%%%%%%%%%%%%%%%%% LAYER DI VISUALIZZAZIONE %%%%%%%%%%%%%%%%%%%
		\paragraph{Caso d'uso UC : Cambiare layer di visualizzazione}
			\begin{itemize}
				\item \textbf{Attori}: Utente;
				\item \textbf{Descrizione}: L'utente sceglie di cambiare il layer degli
				oggetti visualizzabili;
				\item \textbf{Precondizione}: Nell'editor del diagramma delle classi, sono
				presenti almeno due layer, l'utente desidera cambiare il layer di
				visualizzazione;
				\item \textbf{Postcondizione}: Nell'editor del diagramma delle classi del
				sistema sono visualizzati gli oggetti del diagramma appartenenti al layer
				selezionato.
			\end{itemize}
			
%%%%%%%%%%%%%%%%%%% SEZIONE PACKAGE %%%%%%%%%%%%%%%%%%%
		\section{Caso d'uso UC : Editare il diagramma dei package}
			\begin{itemize}
				\item \textbf{Attori}: Utente;
				\item \textbf{Descrizione}: L'utente sceglie di usare l'editor per il
				diagramma dei package;
				\item \textbf{Precondizione}: Il sistema è pronto all'utilizzo dell'editor per
				il diagramma dei package, l'utente desidera utilizzarlo;
				\item \textbf{Scenario principale degli eventi}:
					\begin{itemize}
						\item L'utente può creare un package (UC);
						\item L'utente può modificare un package (UC);
						\item L'utente può rimuovere un package (UC).
					\end{itemize}
				\item \textbf{Postcondizione}: L'utente genera un diagramma dei package
				attraverso l'editor.
			\end{itemize}
		\subsection{Caso d'uso UC : Creare un package}
			\begin{itemize}
				\item \textbf{Attori}: Utente;
				\item \textbf{Descrizione}: L'utente sceglie di creare un package;
				\item \textbf{Precondizione}: Il sistema è pronto alla creazione di package,
				l'utente desidera creare un package;
				\item \textbf{Postcondizione}: Nell'editor del diagramma dei package del
				sistema è visualizzato il diagramma dove è stato aggiunto il package creato.
			\end{itemize}
		\subsection{Caso d'uso UC : Modificare un package}
			\begin{itemize}
				\item \textbf{Attori}: Utente;
				\item \textbf{Descrizione}: L'utente sceglie di modificare un package
				all'interno dell'editor del diagramma dei package;
				\item \textbf{Precondizione}: Nell'editor del diagramma dei package del
				sistema è stato selezionato un package che l'utente desidera modificare;
				\item \textbf{Scenario principale degli eventi}:
					\begin{itemize}
						\item L'utente può definire il nome del package (UC);
						\item L'utente può impostare la visibilità del package (UC);
						\item L'utente può definire una dipendenza tra package (UC);
						\item L'utente può rimuovere una dipendenza tra package (UC);
						\item L'utente può innestare una classe nel package (UC);
						\item L'utente può rimuovere una classe dal package (UC);
						\item L'utente può innestare un'interfaccia nel package (UC);
						\item L'utente può rimuovere un'interfaccia dal package (UC);
						\item L'utente può innestare un package nel package (UC);
						\item L'utente può rimuovere un package dal package (UC);
					\end{itemize}
				\item \textbf{Scenari alternativi}: Viene annullata la modifica, il sistema
				rimane nello stato precedente al tentativo di modifica;
				\item \textbf{Postcondizione}: Nell'editor del diagramma dei package del
				sistema è visualizzato il diagramma dove sono state apportate le modifiche
				al package.
			\end{itemize}
		\subsubsection{Caso d'uso UC : Definire il nome del package}
			\begin{itemize}
				\item \textbf{Attori}: Utente;
				\item \textbf{Descrizione}: L'utente può dare un nome ad un package;
				\item \textbf{Precondizione}: Il sistema è in attesa che l'utente inserisca
				una stringa per rinominare il package;
				\item \textbf{Postcondizione}: Nell'editor del diagramma dei package del
				sistema è visualizzato il diagramma dove è stato cambiato il nome
				al package.
			\end{itemize}
		\subsubsection{Caso d'uso UC : Impostare la visibilità del package}
			\begin{itemize}
				\item \textbf{Attori}: Utente;
				\item \textbf{Descrizione}: L'utente può impostare la visibilità
				del package;
				\item \textbf{Precondizione}: Il sistema è in attesa che l'utente selezioni il
				tipo di visualizzazione da impostare al package;
				\item \textbf{Postcondizione}: Il sistema ha impostato la visibilità
				del package.
			\end{itemize}
		\subsubsection{Caso d'uso UC : Definire dipendenza tra package}
			\begin{itemize}
				\item \textbf{Attori}: Utente;
				\item \textbf{Descrizione}: L'utente sceglie di definire una dipendenza tra
				due package;
				\item \textbf{Precondizione}: Il sistema è pronto a creare la dipendenza che
				l'utente desidera definire;
				\item \textbf{Scenario principale degli eventi}:
					\begin{itemize}
						\item L'utente può selezionare un package (il package con dipendenza
						uscente) (UC);
						\item L'utente può selezionare un package (il package con dipendenza
						entrante) (UC).
					\end{itemize}
				\item \textbf{Scenari alternativi}: La definizione della dipendenza viene
				annullata, il sistema rimane nello stato precedente al tentativo di modifica;
				\item \textbf{Postcondizione}: Nell'editor del diagramma dei package del
				sistema è visualizzato il diagramma dove è stata definita la dipendenza.
			\end{itemize}
		\subsubsection{Caso d'uso UC : Selezionare un package}
			\begin{itemize}
				\item \textbf{Attori}: Utente;
				\item \textbf{Descrizione}: L'utente seleziona il package di interesse;
				\item \textbf{Precondizione}: Il sistema mostra il diagramma dei package;
				\item \textbf{Postcondizione}: Il sistema evidenzia il package selezionato
				dall'utente.
			\end{itemize}
		\subsubsection{Caso d'uso UC : Rimuovere dipendenza tra package}
			\begin{itemize}
				\item \textbf{Attori}: Utente;
				\item \textbf{Descrizione}: L'utente sceglie di rimuovere una dipendenza tra
				due package;
				\item \textbf{Precondizione}: Il sistema è pronto a rimuovere la dipendenza
				che l'utente desidera;
				\item \textbf{Postcondizione}: Nell'editor del diagramma dei package del
				sistema è visualizzato il diagramma dove è stata rimossa la dipendenza.
			\end{itemize}
		\subsubsection{Caso d'uso UC : Innestare una classe nel package}
			\begin{itemize}
				\item \textbf{Attori}: Utente;
				\item \textbf{Descrizione}: L'utente sceglie di innestare una classe
				all'interno di un package;
				\item \textbf{Precondizione}: Il sistema è pronto per effettuare l'operazione;
				\item \textbf{Scenario principale degli eventi}:
					\begin{itemize}
						\item L'utente può selezionare una classe (UC Sopra);
						\item L'utente può selezionare un package (UC Sopra);
					\end{itemize}
				\item \textbf{Scenari alternativi}: L'innesto viene annullato, il sistema
				rimane nello stato precedente al tentativo di modifica;
				\item \textbf{Postcondizione}: Nell'editor del diagramma dei package del
				sistema è visualizzato il diagramma dove è stato effettuato l'innesto.
			\end{itemize}
		\subsubsection{Caso d'uso UC : Rimuovere una classe dal package}
			\begin{itemize}
				\item \textbf{Attori}: Utente;
				\item \textbf{Descrizione}: L'utente può rimuovere una classe da un package
				nell'editor del diagramma dei package. L'utente deve selezionare
				la classe;
				\item \textbf{Precondizione}: L'utente desidera rimuovere una classe
				selezionata da un package nell'editor del diagramma dei package del sistema;
				\item \textbf{Postcondizione}: Nell'editor del diagramma dei package del
				sistema è visualizzato il diagramma dove è stata rimossa la classe.
			\end{itemize}
	%
	\section{Editare il diagramma delle attività}
	%
	\subsection{Aggiungere pin}
	\begin{itemize}
		\item \textbf{Attori}: Utente;
		\item \textbf{Descrizione}: L’utente deve aggiungere un pin ad una attività;
		\item \textbf{Precondizione}: Nell'editor del diagramma delle attività è stata selezionata un'attività;
		\item \textbf{Postcondizione}: Nell'editor del diagramma delle attività è visualizzato il diagramma dove è stato aggiunto ad una classe un pin in input o in output con campi vuoti;
	\end{itemize}
	
	\subsection{Modificare pin}
	\begin{itemize}
		\item \textbf{Attori}: Utente;
		\item \textbf{Descrizione}: L’utente deve modificare i pin da lui inseriti così da impostare la variabile da essi rappresentata;
		\item \textbf{Precondizione}: Nell'editor del diagramma delle attività è stata selezionata un'attività dotata di un pin;
		\item \textbf{Postcondizione}: Nell'editor del diagramma delle attività è visualizzato il diagramma dove il pin associato alla classe ora rappresenta correttamente la variabile corrispondente;
	\end{itemize}
	
	\subsection{Rimuovere pin}
	\begin{itemize}
		\item \textbf{Attori}: Utente;
		\item \textbf{Descrizione}: L’utente deve eliminare un pin da un’attività;
		\item \textbf{Precondizione}: Nell'editor del diagramma delle attività è stata selezionata un'attività dotata di un pin;
		\item \textbf{Postcondizione}: Nell'editor del diagramma delle attività è visualizzato il diagramma dove il pin associato alla classe è stato rimosso;
	\end{itemize}

	\subsection{Aggiungere trasformazione}
	\begin{itemize}
		\item \textbf{Attori}: Utente;
		\item \textbf{Descrizione}: L’utente deve collegare tra di loro i pin di output di un’attività e i pin di input di una seconda mediante una trasformazione;
		\item \textbf{Precondizione}: Nell'editor del diagramma delle attività sono presenti almeno due attività di cui una dotata di pin in output e una dotata di pin in input, con tipo tra di loro compatibile;
		\item \textbf{Postcondizione}: Nell'editor del diagramma delle attività è visualizzato il diagramma dove il pin di output della prima classe è collegato con il pin di input della seconda mediante una trasformazione con espressione di trasformazione vuota;
	\end{itemize}

	\subsection{Modificare trasformazione}
	\begin{itemize}
		\item \textbf{Attori}: Utente;
		\item \textbf{Descrizione}: L’utente deve modificare una trasformazione, in particolare editando l’\textit{espressione di trasformazione};
		\item \textbf{Precondizione}: Nell'editor del diagramma delle attività è stata selezionata una trasformazione che collega pin di output e di input di due attività;
		\item \textbf{Postcondizione}: Nell'editor del diagramma delle attività è visualizzato il diagramma dove l’espressione di trasformazione voluta è stata modificata;
	\end{itemize}

	\subsection{Rimuovere trasformazione}
	\begin{itemize}
		\item \textbf{Attori}: Utente;
		\item \textbf{Descrizione}: L’utente deve eliminare una trasformazione che collega due pin di due attività;
		\item \textbf{Precondizione}: Nell'editor del diagramma delle attività è stata selezionata una trasformazione che collega pin di output e di input di due attività;
		\item \textbf{Postcondizione}: Nell'editor del diagramma delle attività è visualizzato il diagramma dove è stata eliminata la trasformazione che collegava i due pin;
	\end{itemize}
		
	\subsection{Aggiungere invio di segnale}
	\begin{itemize}
		\item \textbf{Attori}: Utente;
		\item \textbf{Descrizione}: L'utente deve aggiungere ad un'attività un invio di segnale;
		\item \textbf{Precondizione}: Nella schermata dell'editor del diagramma delle attività è stata selezionata un'attività;
		\item \textbf{Postcondizione}: Nella schermata dell'editor del diagramma delle attività è visualizzato il diagramma dove l'attività è stata modificata con l'aggiunta un invio di segnale vuoto;
	\end{itemize}
	
	\subsection{Modificare invio di segnale}
	\begin{itemize}
		\item \textbf{Attori}: Utente;
		\item \textbf{Descrizione}: L'utente deve modificare l'invio di segnale associato ad un'attività;
		\item \textbf{Precondizione}: Nella schermata dell'editor del diagramma delle attività è stata selezionata un'attività con associato un invio di segnale;
		\item \textbf{Postcondizione}: Nella schermata dell'editor del diagramma delle attività è visualizzato il diagramma dove sono stati modificati i parametri dell'invio di segnale associato all'attività;
	\end{itemize}
	
	\subsection{Rimuovere invio di segnale}
	\begin{itemize}
		\item \textbf{Attori}: Utente;
		\item \textbf{Descrizione}: L'utente deve rimuovere un invio di segnale associato ad un'attività;
		\item \textbf{Precondizione}: Nella schermata dell'editor del diagramma delle attività è stata selezionata un'attività con associato un invio di segnale;
		\item \textbf{Postcondizione}: Nella schermata dell'editor del diagramma delle attività è visualizzato il diagramma dove è stato eliminato l'invio di segnale associato all'attività;
	\end{itemize}
		
	\subsection{Aggiungere attesa di segnale}
	\begin{itemize}
		\item \textbf{Attori}: Utente;
		\item \textbf{Descrizione}: L'utente deve aggiungere ad un'attività un'attesa di segnale;
		\item \textbf{Precondizione}: Nella schermata dell'editor del diagramma delle attività è stata selezionata un'attività;
		\item \textbf{Postcondizione}: Nella schermata dell'editor del diagramma delle attività è visualizzato il diagramma dove l'attività è stata modificata con l'aggiunta un'attesa di segnale vuota;
	\end{itemize}
	
	\subsection{Modificare attesa di segnale}
	\begin{itemize}
		\item \textbf{Attori} Utente;
		\item \textbf{Descrizione}:  L'utente deve modificare l'attesa di segnale associata ad un'attività;
		\item \textbf{Precondizione}: Nella schermata dell'editor del diagramma delle attività è stata selezionata un'attività con associata un'attesa di segnale;
		\item \textbf{Postcondizione}: Nella schermata dell'editor del diagramma delle attività è visualizzato il diagramma dove sono stati modificati i parametri dell'attesa di segnale associata all'attività;
	\end{itemize}
	
	\subsection{Rimuovere attesa di segnale}
	\begin{itemize}
		\item \textbf{Attori}: Utente;
		\item \textbf{Descrizione}: L'utente deve rimuovere un'attesa di segnale associata ad un'attività;
		\item \textbf{Precondizione}: Nella schermata dell'editor del diagramma delle attività è stata selezionata un'attività con associato un'attesa di segnale;
		\item \textbf{Postcondizione}: Nella schermata dell'editor del diagramma delle attività è visualizzato il diagramma dove è stata eliminata l'attesa di segnale associata all'attività;
	\end{itemize}

	\subsection{Aggiungere evento temporale}
	\begin{itemize}
		\item \textbf{Attori}: Utente;
		\item \textbf{Descrizione}: L'utente deve aggiungere un evento temporale al diagramma delle attività;
		\item \textbf{Precondizione}: Nella schermata dell'editor del diagramma delle attività il sistema è pronto per l'aggiunta di un elemento;
		\item \textbf{Postcondizione}: Nella schermata dell'editor del diagramma delle attività è visualizzato il diagramma dove è stato aggiunto un evento temporale vuoto;
	\end{itemize}
	
	\subsection{Modificare evento temporale}
	\begin{itemize}
		\item \textbf{Attori}: Utente;
		\item \textbf{Descrizione}: L'utente deve modificare il tipo e i parametri di un evento temporale;
		\item \textbf{Precondizione}: Nella schermata dell'editor del diagramma delle attività è stata selezionato un evento temporale;
		\item \textbf{Postcondizione}: Nella schermata dell'editor del diagramma delle attività è visualizzato il diagramma dove sono stati modificati opportunamente il tipo ed i parametri dell'evento temporale;
	\end{itemize}
	
	\subsection{Rimuovere evento temporale}
	\begin{itemize}
		\item \textbf{Attori}: Utente;
		\item \textbf{Descrizione}: L'utente deve eliminare un evento temporale;
		\item \textbf{Precondizione}: Nella schermata dell'editor del diagramma delle attività è stata selezionato un evento temporale;
		\item \textbf{Postcondizione}: Nella schermata dell'editor del diagramma delle attività è visualizzato il diagramma dove è stato eliminato l'evento temporale;
	\end{itemize}

	\section{Editare bubble flowchart}
	\begin{itemize}
		\item \textbf{Attori}: Utente;
		\item \textbf{Descrizione}: L'utente deve editare un bubble flowchart;
		\item \textbf{Precondizione}: Nella schermata dell'editor del bubble flowchart il sistema è pronto per editare il bubble flowchart;
		\item \textbf{Scenario principale degli eventi}
		\begin{itemize}
			\item L'utente può Aggiungere una bubble(UC);
			\item L'utente può Modificare una bubble(UC);
			\item L'utente può Eliminare una bubble(UC);
			\item L'utente può Aggiungere un elemento di decisione(UC);
			\item L'utente può Modificare un elemento di decisione(UC);
			\item L'utente può Eliminare un elemento di decisione(UC);
		\end{itemize}
		\item \textbf{Postcondizione}: Nella schermata dell'editor del bubble flowchart è visualizzato il diagramma editato;
	\end{itemize}
	
	\subsection{Aggiungere una bubble}
	\begin{itemize}
		\item \textbf{Attori}: Utente;
		\item \textbf{Descrizione}: L'utente deve aggiungere una bubble di un tipo desiderato al bubble flowchart;
		\item \textbf{Precondizione}: Nella schermata dell'editor del bubble flowchart il sistema è pronto per l'aggiunta di una bubble;
		\item \textbf{Postcondizione}: Nella schermata dell'editor del bubble flowchart è visualizzato il diagramma a cui è stata aggiunta una bubble vuota del tipo voluto;
	\end{itemize}
	
	\subsection{Modificare una bubble}
	\begin{itemize}
		\item \textbf{Attori}: Utente;
		\item \textbf{Descrizione}: L'utente deve modificare i parametri di una bubble;
		\item \textbf{Precondizione}: Nella schermata dell'editor del bubble flowchart è stata selezionata una bubble;
		\item \textbf{Postcondizione}: Nella schermata dell'editor del bubble flowchart è visualizzato il diagramma in cui sono stati opportunamente modificati i parametri della bubble;
	\end{itemize}
	
	\subsection{Eliminare una bubble}
	\begin{itemize}
		\item \textbf{Attori}: Utente;
		\item \textbf{Descrizione}: L'utente deve eliminare una bubble;
		\item \textbf{Precondizione}: Nella schermata dell'editor del bubble flowchart è stata selezionata una bubble;
		\item \textbf{Postcondizione}: Nella schermata dell'editor del bubble flowchart è visualizzato il diagramma in cui è stata eliminata la bubble;
	\end{itemize}
	
	\subsection{Aggiungere un elemento di decisione}
	\begin{itemize}
		\item \textbf{Attori}: Utente;
		\item \textbf{Descrizione}: L'utente deve aggiungere un elemento di decisione al bubble flowchart;
		\item \textbf{Precondizione}: Nella schermata dell'editor del bubble flowchart il sistema è pronto per l'aggiunta di un elemento di decisione;
		\item \textbf{Postcondizione}: Nella schermata dell'editor del bubble flowchart è visualizzato il diagramma a cui è stato aggiunto un elemento di decisione vuoto;
	\end{itemize}
	
	\subsection{Modificare un elemento di decisione}
	\begin{itemize}
		\item \textbf{Attori}: Utente;
		\item \textbf{Descrizione}: L'utente deve modificare i parametri di un elemento di decisione;
		\item \textbf{Precondizione}: Nella schermata dell'editor del bubble flowchart è stato selezionato un elemento di decisione;
		\item \textbf{Postcondizione}: Nella schermata dell'editor del bubble flowchart è visualizzato il diagramma in cui sono stati opportunamente modificati i parametri dell'elemento di decisione;
	\end{itemize}
	
	\subsection{Eliminare un elemento di decisione}
	\begin{itemize}
		\item \textbf{Attori}: Utente;
		\item \textbf{Descrizione}: L'utente deve eliminare un elemento di decisione;
		\item \textbf{Precondizione}: Nella schermata dell'editor del bubble flowchart è stato selezionato un elemento di decisione;
		\item \textbf{Postcondizione}: Nella schermata dell'editor del bubble flowchart è visualizzato il diagramma in cui è stato eliminato l'elemento di decisione;
	\end{itemize}
	
	
	\section{Leggere il codice}
	\begin{itemize}
		\item \textbf{Attori}: Utente;
		\item \textbf{Descrizione}: L'utente deve leggere il codice;
		\item \textbf{Precondizione}: Nella schermata del visualizzatore del codice il sistema è pronto a mostrare il codice prodotto;
		\item \textbf{Postcondizione}: Nella schermata del visualizzatore del codice è mostrato il codice prodotto
	\end{itemize}
	
	\subsection{Esportare il codice}
	\begin{itemize}
		\item \textbf{Attori}: Utente;
		\item \textbf{Descrizione}: L'utente deve esportare il codice generato nei file sorgente appropriati per il linguaggio corrispondente;
		\item \textbf{Precondizione}: Nelle schermate degli editor messi a disposizione del programma sono stati disegnati i diagrammi che rappresentano il codice desiderato;
		\item \textbf{Postcondizione}: In una cartella a scelta dell'utente il programma ha generato tutti i file sorgenti voluti, organizzati secondo quanto specificato dall'utente tramite i diagrammi. Questi file contengono codice corretto e compilabile. Qualora il programma non avesse potuto tradurre efficacemente una parte del diagramma dell'utente, il programma ha comunicato un avvertimento all'utente e commentato opportunamente il codice nel sorgente;
	\end{itemize}
	
	\section{Salvare il progetto}
	\begin{itemize}
		\item \textbf{Attori}: Utente;
		\item \textbf{Descrizione}: L'utente deve salvare il lavoro fatto fino a quel momento;
		\item \textbf{Precondizione}: Nelle schermate degli editor messi a disposizione del programma sono stati disegnati i diagrammi che rappresentano il codice desiderato;
		\item \textbf{Postcondizione}: In una cartella a scelta dell'utente il programma ha generato un file contenente tutte le informazioni necessarie per ripristinarne lo stato attuale;
	\end{itemize}		
			
\end{document}
