\newglossaryentry{UML} {
	name=UML,
	description={Acronimo di Unified Modeling Language (linguaggio di modellizzazione
	unificato), è un linguaggio di modellizzazione e specifica basato sul paradigma
	orientato agli oggetti}
}
\newglossaryentry{Java} {
	name=Java,
	description={Linguaggio di programmazione ad alto livello orientato agli oggetti}
}
\newglossaryentry{Javascript} {
	name=Javascript,
	description={Linguaggio di scripting orientato agli oggetti e agli eventi,
	comunemente utilizzato nella programmazione web lato client}
}
\newglossaryentry{Android} {
	name=Android,
	description={Sistema operativo per dispositivi mobili sviluppato da Google Inc. e basato su kernel Linux}
}
\newglossaryentry{HTML} {
	name=HTML,
	description={Acronimo di HyperText Markup Language (linguaggio a marcatori per
	ipertesti), è un linguaggio di markup usato principalmente per creare la
	struttura di documenti ipertestuali}
}
\newglossaryentry{CSS} {
	name=CSS,
	description={Acronimo di Cascading Style Sheets (fogli di stile a cascata), è un linguaggio usato per
	definire la formattazione di documenti HTML}
}
\newglossaryentry{ISO} {
	name=ISO,
	description={Abbreviazione di International Organization fo Standardization (organizzazione internazionale per la
	normalizzazione), è la più importante organizzazione a livello mondiale per la definizione di norme tecniche}
}
\newglossaryentry{UTF-8} {
	name=UTF-8,
	description={Acronimo di Unicode Transformation Format 8 bit, è una codifica di caratteri Unicode in sequenze di
	lunghezza variabile di byte}
}
\newglossaryentry{Mailing list} {
	name=Mailing list,
	description={Lista di distribuzione o diffusione, è un servizio/strumento offribile da una rete di computer verso
	vari utenti e costituito da un sistema organizzato per la partecipazione di più persone ad una discussione asincrona
	o per la distribuzione di informazioni utili agli interessati/iscritti attraverso l'invio di e-mail ad una lista di
	indirizzi di posta elettronica di utenti iscritti}
}
