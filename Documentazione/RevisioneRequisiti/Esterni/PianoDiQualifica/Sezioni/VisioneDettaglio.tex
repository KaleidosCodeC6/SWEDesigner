\documentclass[../PianoDiQualifica.tex]{subfiles}
\begin{document}
	\section{La strategia di gestione della qualità nel dettaglio}
		\subsection{Risorse}
			\subsubsection{Necessarie}
				Per la realizzazione del prodotto sono necessarie le risorse umane e
				tecnologiche citate di seguito.
				\begin{itemize}
					\item \textbf{Risorse umane}: sono descritte dettagliatamente nel
					\pianodiprogetto. 
					\begin{itemize}
						\item \responsabilediprogetto;
						\item \amministratore;
						\item \analista;
						\item \progettista;
						\item \programmatore;
						\item \verificatore.
					\end{itemize}
					\item \textbf{Risorse software}: sono descritte dettagliatamente
					nelle \normediprogetto.
					Si tratta di software che permettano:
					\begin{itemize}
						\item la comunicazione e la condivisione del lavoro tra gli
						elementi del team;
						\item la stesura della documentazione in formato LaTeX;
						\item la creazione di diagrammi UML;
						\item la codifica nei linguaggi di programmazione scelti;
						\item la semplificazione delle attività di verifica;
						\item la gestione dei test sul codice.
					\end{itemize}
				\item \textbf{Risorse hardware}: ciascun componente del gruppo ha bisogno
				di un computer con tutti i software necessari. È necessario avere a
				disposizione almeno un luogo dove poter effettuare le riunioni del team.
				\end{itemize}
			\subsubsection{Disponibili}
				Ogni membro del team ha a disposizione uno o più computer personali
				dotati degli strumenti necessari.\\
				Le riunioni interne si svolgono presso le aule del dipartimento di
				Matematica dell'Università degli Studi di Padova.
		\subsection{Misure e metriche}
			Il processo di verifica deve essere quantificabile per fornire informazioni
			utili, bisogna quindi stabilire le metriche da adottare per le misurazioni.
			Si definiranno due intervalli di misure:
			\begin{itemize}
				\item \textbf{Range di accettazione}: intervallo di valori vincolante per
				l'accettazione del prodotto;
				\item \textbf{Range ottimale}: intervallo di valori entro cui è
				consigliabile rientri la misurazione. Il mancato rispetto di questa
				condizione non pregiudica l'accettazione del prodotto, ma richiede
				verifiche più approfondite in merito.
			\end{itemize}
			\subsubsection{Metriche per i processi}
				\paragraph{Schedule Variance\\}
					È una metrica di progetto standard, indica se si è in linea, in
					anticipo o in ritardo rispetto alla schedulazione pianificata delle
					attività di progetto. È pari alla differenza tra il valore delle
					attività pianificate e il valore delle attività svolte alla data
					corrente.
					\subparagraph{Parametri utilizzati}
						\begin{itemize}
							\item \textbf{Range di accettazione}: $\geq -(preventivo*5\%)$;
							\item \textbf{Range ottimale}: $\geq 0$.
						\end{itemize}
				\paragraph{Budget Variance\\}
					È una metrica di progetto standard, indica se si spende di più o di
					meno rispetto a quanto preventivato alla data corrente. È pari alla
					differenza tra costo pianificato e costo effettivamente sostenuto
					alla data corrente.
					\subparagraph{Parametri utilizzati}
						\begin{itemize}
							\item \textbf{Range di accettazione}: $\geq -(preventivo*10\%)$;
							\item \textbf{Range ottimale}: $\geq 0$.
						\end{itemize}
			\subsubsection{Metriche per i documenti}
				\paragraph{Indice Gulpease\\}
					Definito nel 1988 all'Università degli Studi di Roma ``La
					Sapienza'' per valutare la leggibilità di un documento redatto in
					lingua italiana, l'indice \gl{Gulpease} si basa sul calcolo del
					numero di caratteri contenuto in una parola rapportato con altri
					fattori quali il numero di parole e di frasi.
					La formula per il calcolo dell'indice Gulpease è la seguente:
					\begin{equation*}
						89+\frac{300\left(\textit{numero di frasi}\right)-10\left(\textit{numero di lettere}\right)}{\textit{numero di parole}}
					\end{equation*}
					Il risultato indica	quindi la complessità del documento con un
					valore compreso tra 0 e 100, dove 100 indica la più alta
					leggibilità. Attraverso gli studi condotti,	risulta che testi con
					un indice:
					\begin{itemize}
						\item \textbf{inferiore a 80} sono difficili da leggere per chi
						ha la licenza elementare;
						\item \textbf{inferiore a 60} sono difficili da leggere per chi
						ha licenza media;
						\item \textbf{inferiore a 40} sono difficili da leggere per chi
						ha un diploma superiore.
					\end{itemize}
					Tale indice, però, non indica la comprensibilità del testo. Il
					documento potrebbe contenere frasi incomprensibili ed avere comunque
					un alto indice Gulpease. Per la tipologia dei documenti redatti, la
					formalità nella scrittura e gli argomenti trattati risulta difficile
					adeguare la stesura del testo ad un indice Gulpease ottimale. Per
					questo motivo, ogni documento sarà valutato anche da un individuo
					che avrà il compito di valutare se parti di testo dovranno essere
					semplificate o meno. Inoltre, i limiti imposti da tale indice
					saranno sufficientemente rilassati per accettare anche frasi poco
					più complesse.
					\subparagraph{Parametri utilizzati}
						\begin{itemize}
							\item \textbf{Range di accettazione}: 40 - 100;
							\item \textbf{Range ottimale}: 50 - 100.
						\end{itemize}
			\subsubsection{Metriche per il codice}
				\paragraph{Rapporto linee di commento su linee di codice\\}
					Indica il rapporto tra linee di commento e linee di codice in un
					file (linee vuote escluse). Ritenendo importante la rapidità di
					comprensione del codice, questa metrica è utile per stimare la
					manutenibilità.
					\subparagraph{Parametri utilizzati}
						\begin{itemize}
							\item \textbf{Range di accettazione}: $\geq$ 0.25;
							\item \textbf{Range ottimale}: $\geq$ 0.30.
						\end{itemize}
				\paragraph{Numero di parametri\\}
					Indica il numero di parametri formali di un metodo. Più alto è il
					numero dei parametri formali, più aumenta la quantità di memoria
					occupata nella pila dei processi.
					\subparagraph{Parametri utilizzati}
						\begin{itemize}
							\item \textbf{Range di accettazione}: 0 - 8;
							\item \textbf{Range ottimale}: 0 - 5.
						\end{itemize}
				\paragraph{Numero di campi dati\\}
					Indica il numero di campi dati interni ad una classe. Un numero
					elevato può rendere difficile la manutenibilità del codice della
					classe, oltre ad essere indice di cattiva programmazione.\\
					È possibile ridurre il numero di campi dati attraverso
					l'incapsulamento di ulteriori classi.
					\subparagraph{Parametri utilizzati}
						\begin{itemize}
							\item \textbf{Range di accettazione}: 0 - 16;
							\item \textbf{Range ottimale}: 0 - 10.
						\end{itemize}
				\paragraph{Complessità ciclomatica\\}
					Indica il numero di cammini linearmente indipendenti attraverso il
					grafo di controllo di flusso del metodo/funzione: i nodi del grafo
					corrispondono a gruppi indivisibili di istruzioni, mentre gli archi
					connettono due nodi se il secondo gruppo può essere eseguito
					immediatamente dopo il primo.\\
					È possibile ridurre l'indice di complessità attraverso la
					suddivisione del metodo/funzione in più parti.
					\subparagraph{Parametri utilizzati}
						\begin{itemize}
							\item \textbf{Range di accettazione}: 0 - 10;
							\item \textbf{Range ottimale}: 0 - 6.
						\end{itemize}
						\par È accettato anche un valore più elevato, qualora
						dovesse influire positivamente sulla velocità di esecuzione.
\end{document}