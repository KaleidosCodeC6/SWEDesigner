\documentclass[../PianoDiQualifica.tex]{subfiles}
\begin{document}

\section{Risorse}
\subsection{Necessarie}
per la realizzazione del prodotto sono necessarie le risorse umane e tecnologiche citate di seguito.
\begin{itemize}
	\item \textbf{risorse umane}: sono descritte dettagliatamente nel \pianodiprogetto. 
	\begin{itemize}
		\item \responsabilediprogetto;
		\item \amministratore;
		\item \analista;
		\item \progettista;
		\item \programmatore;
		\item \verificatore.
	\end{itemize}
	\item \textbf{risorse software}: sono descritte dettagliatamente nelle \normediprogetto.
	Si tratta di software che permettano:
	\begin{itemize}
		\item la comunicazione e la condivisione del lavoro tra gli elementi del team;
		\item la stesura della documentazione in formato LaTeX;
		\item la creazione di diagrammi UML;
		\item la codifica nei linguaggi di programmazione scelti;
		\item la semplificazione delle attività di verifica;
		\item la gestione dei test sul codice.
	\end{itemize}
\item \textbf{risorse hardware}: ciascun componente del gruppo ha bisogno di un computer con tutti i software necessari. È necessario avere a disposizione almeno un luogo dove poter effettuare le riunioni del team.
\end{itemize}
\subsection{Disponibili}
Ogni membro del team ha a disposizione uno o più computer personali dotati degli strumenti necessari. \\
Le riunioni interne si svolgono presso le aule del dipartimento di Matematica dell'Università degli Studi di Padova.
\section{Misure e metriche}
Il processo di verifica deve essere quantificabile per fornire informazioni utili, bisogna quindi stabilire le metriche da adottare per le misurazioni. Si definiranno due intervalli di misure:
\begin{itemize}
	\item \textbf{range di accettazione}: intervallo di valori vincolante per l'accettazione del prodotto;
	\item \textbf{range ottimale}: intervallo di valori entro cui è consigliabile rientri la misurazione. Il mancato rispetto di questa condizione non pregiudica l'accettazione del prodotto, ma richiede verifiche più approfondite in merito.
\end{itemize}
\subsection{Metriche per i processi}
\subsubsection{Schedule Variance}
È una metrica di progetto standard, indica se si è in linea, in anticipo o in ritardo rispetto alla schedulazione pianificata delle attività di progetto. È pari alla differenza tra il valore delle attività pianificate e il valore delle attività svolte alla data corrente.
\paragraph{Parametri utilizzati}
\begin{itemize}
	\item \textbf{Range di accettazione}: $\geq -(preventivo*5\%)$;
	\item \textbf{Range ottimale}: $\geq 0$.
\end{itemize}
\subsubsection{Budget Variance}
È una metrica di progetto standard, indica se si spende di più o di meno rispetto a quanto preventivato alla data corrente. È pari alla differenza tra costo pianificato e costo effettivamente sostenuto alla data corrente.
\paragraph{Parametri utilizzati}
\begin{itemize}
	\item \textbf{Range di accettazione}: $\geq -(preventivo*10\%)$;
	\item \textbf{Range ottimale}: $\geq 0$.
\end{itemize}

\end{document}