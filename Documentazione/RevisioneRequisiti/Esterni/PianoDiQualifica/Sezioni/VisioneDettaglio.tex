\documentclass[../PianoDiQualifica.tex]{subfiles}
\begin{document}
	\section{La strategia di gestione della qualità nel dettaglio}
		\subsection{Risorse}
			\subsubsection{Necessarie}
				Per la realizzazione del prodotto sono necessarie le risorse
				umane e tecnologiche elencate di seguito.
				\begin{itemize}
					\item \textbf{Risorse umane}: sono descritte
					dettagliatamente nel \pianodiprogettov:
					\begin{itemize}
						\item \responsabilediprogetto;
						\item \amministratore;
						\item \analista;
						\item \progettista;
						\item \programmatore;
						\item \verificatore.
					\end{itemize}
					\item \textbf{Risorse software}: sono descritte
					dettagliatamente nelle \normediprogettov. Si tratta di
					software che permettono:
					\begin{itemize}
						\item la comunicazione e la condivisione del lavoro
						tra gli elementi del team;
						\item la stesura della documentazione in
						formato \gl{\LaTeX};
						\item la creazione di diagrammi UML;
						\item la codifica nei linguaggi di programmazione scelti;
						\item la semplificazione delle attività di verifica;
						\item la gestione dei test sul codice.
					\end{itemize}
					\item \textbf{Risorse hardware}: ciascun componente del
					gruppo deve avere un computer con tutti i software necessari
					descritti nelle \normediprogettov. È necessario avere a
					disposizione almeno un luogo dove poter effettuare le
					riunioni interne.
				\end{itemize}
			\subsubsection{Disponibili}
				Ogni membro del team ha a disposizione uno o più computer
				personali dotati degli strumenti necessari.\\
				Le riunioni interne si svolgono presso le aule del dipartimento
				di Matematica dell'Università degli Studi di Padova.
		\subsection{Misure e metriche}\label{Misure&Metriche}
			Il processo di verifica deve essere quantificabile per fornire
			informazioni utili. Bisogna quindi stabilire le metriche da adottare
			per le misurazioni durante i processi di verifica.\\
			Si definiranno due intervalli di misure (\gl{range}):
			\begin{itemize}
				\item \textbf{Range di accettazione}: intervallo di valori
				vincolante per l'accettazione del prodotto;
				\item \textbf{Range ottimale}: intervallo di valori entro cui è
				consigliabile rientri la misurazione. Il mancato rispetto di
				questa condizione non pregiudica l'accettazione del prodotto, ma
				richiede verifiche più approfondite in merito.
			\end{itemize}
			\subsubsection{Metriche per i processi}
				\paragraph{\gl{Schedule Variance}\\}
					È una metrica di progetto standard, indica se si è in linea, in
					anticipo o in ritardo rispetto alla schedulazione pianificata delle
					attività di progetto. È pari alla differenza tra il valore delle
					attività realizzate ed il valore delle attività pianificate alla
					data corrente.
					\subparagraph{Parametri utilizzati}
						\begin{itemize}
							\item \textbf{Range di accettazione}: $\geq -(preventivo*5\%)$;
							\item \textbf{Range ottimale}: $\geq 0$.
						\end{itemize}
				\paragraph{\gl{Budget Variance}\\}
					È una metrica di progetto standard, indica se si è speso di più o
					di meno rispetto a quanto preventivato alla data corrente. È pari
					alla differenza tra costo pianificato e costo effettivamente
					sostenuto alla data corrente.
					\subparagraph{Parametri utilizzati}
						\begin{itemize}
							\item \textbf{Range di accettazione}: $\geq -(preventivo*10\%)$;
							\item \textbf{Range ottimale}: $\geq 0$.
						\end{itemize}
			\subsubsection{Metriche per i documenti}
				\paragraph{Indice Gulpease\\}
					Definito nel 1988 all'Università degli Studi di Roma ``La
					Sapienza'' per valutare la leggibilità di un documento redatto in
					lingua italiana, l'indice Gulpease si basa sul calcolo del
					numero di caratteri contenuto in una parola rapportato con altri
					fattori quali il numero di parole e di frasi.
					La formula per il calcolo dell'indice Gulpease è la seguente:
					\begin{equation*}
						89+\frac{300\left(\textit{numero di frasi}\right)-10\left(\textit{numero di lettere}\right)}{\textit{numero di parole}}
					\end{equation*}
					Il risultato indica	quindi la complessità del documento con un
					valore compreso tra 0 e 100, dove 100 indica la più alta
					leggibilità. Attraverso gli studi condotti,	risulta che testi con
					un indice:
					\begin{itemize}
						\item \textbf{inferiore a 80} sono difficili da leggere per chi
						ha la licenza elementare;
						\item \textbf{inferiore a 60} sono difficili da leggere per chi
						ha licenza media;
						\item \textbf{inferiore a 40} sono difficili da leggere per chi
						ha un diploma superiore.
					\end{itemize}
					Tale indice, però, non indica la comprensibilità del testo. Il
					documento potrebbe contenere frasi incomprensibili ed avere comunque
					un alto indice Gulpease. Per la tipologia dei documenti redatti, la
					formalità nella scrittura e gli argomenti trattati risulta difficile
					adeguare la stesura del testo ad un indice Gulpease ottimale. Ogni
					documento sarà quindi controllato anche da un essere umano
					che avrà il compito di valutare se parti di testo dovranno essere
					semplificate o meno. Inoltre, i limiti imposti da tale indice
					saranno sufficientemente rilassati per accettare anche frasi poco
					più complesse.
					\subparagraph{Parametri utilizzati}
						\begin{itemize}
							\item \textbf{Range di accettazione}: 40 - 100;
							\item \textbf{Range ottimale}: 50 - 100.
						\end{itemize}
			\subsubsection{Metriche per il codice}\label{MetrichePerIlCodice}
				\paragraph{Rapporto linee di commento su linee di codice\\}
					Indica il rapporto tra linee di commento e linee di codice in un
					file (linee vuote escluse). Ritenendo importante la rapidità di
					comprensione del codice, questa metrica è utile per stimare la
					manutenibilità.
					\subparagraph{Parametri utilizzati}
						\begin{itemize}
							\item \textbf{Range di accettazione}: $\geq$ 0.25;
							\item \textbf{Range ottimale}: $\geq$ 0.30.
						\end{itemize}
				\paragraph{Numero di parametri\\}
					Indica il numero di parametri formali di un metodo. Più alto è il
					numero dei parametri formali, più aumenta la quantità di memoria
					occupata nella pila dei processi.
					\subparagraph{Parametri utilizzati}
						\begin{itemize}
							\item \textbf{Range di accettazione}: 0 - 8;
							\item \textbf{Range ottimale}: 0 - 5.
						\end{itemize}
				\paragraph{Numero di campi dati\\}
					Indica il numero di campi dati interni ad una classe. Un numero
					elevato può rendere difficile la manutenibilità del codice della
					classe oltre ad essere indice di cattiva programmazione.\\
					È possibile ridurre il numero di campi dati attraverso
					l'incapsulamento di ulteriori classi.
					\subparagraph{Parametri utilizzati}
						\begin{itemize}
							\item \textbf{Range di accettazione}: 0 - 16;
							\item \textbf{Range ottimale}: 0 - 10.
						\end{itemize}
				\paragraph{Complessità ciclomatica\\}
					Indica il numero di cammini linearmente indipendenti attraverso il
					grafo di controllo di flusso del metodo/funzione: i nodi del grafo
					corrispondono a gruppi indivisibili di istruzioni mentre gli archi
					connettono due nodi se il secondo gruppo può essere eseguito
					immediatamente dopo il primo.\\
					È possibile ridurre l'indice di complessità attraverso la
					suddivisione del metodo/funzione in più parti.
					\subparagraph{Parametri utilizzati}
						\begin{itemize}
							\item \textbf{Range di accettazione}: 0 - 10;
							\item \textbf{Range ottimale}: 0 - 6.
						\end{itemize}
						\par È accettato anche un valore più elevato, qualora
						dovesse influire positivamente sulla velocità di esecuzione.
				\paragraph{Livello di annidamento\\}
					Indica quante strutture di controllo sono inserite l'una all'interno
					dell'altra. Un alto livello di annidamento può portare ad una
					complessità maggiore del codice causando difficoltà nella verifica,
					comprensione e modifica dello stesso.
					\subparagraph{Parametri utilizzati}
						\begin{itemize}
							\item \textbf{Range di accettazione}: 0 - 6;
							\item \textbf{Range ottimale}: 0 - 4.
						\end{itemize}
				\paragraph{Grado di accoppiamento\\}
					Viene calcolato in base a due indici:
					\begin{itemize}
						\item Accoppiamento afferente: numero di classi esterne
						al \gl{package} che dipendono da sue classi interne. Un numero
						alto indica che troppe classi dipendono da tale package, quindi
						eventuali modifiche provocherebbero forti ripercussioni
						sull'esterno. Se il numero è basso il package risulta poco
						utile;
						\item Accoppiamento efferente: numero di classi interne
						al package dipendenti da classi esterne. Un numero alto può
						essere sintomo di una scarsa progettazione.
					\end{itemize}
				\paragraph{Grado di instabilità\\}
					Tale metrica viene utilizzata per misurare l'instabilità delle
					componenti di un sistema basandosi sul grado di accoppiamento sopra
					descritto. Un valore alto indica alta instabilità e quindi bassa
					libertà di modifica del codice in quanto ogni modifica ha effetti
					su più classi. Tale indice viene calcolato con la seguente formula:
					\begin{equation*}
						I=\frac{C \ped{e} }{C \ped{a}  +C \ped{e} }
					\end{equation*}
					dove:
					\begin{itemize}
						\item C\ped a: rappresenta l'accoppiamento afferente;
						\item C\ped e: rappresenta l'accoppiamento efferente.
					\end{itemize}
					\subparagraph{Parametri utilizzati}
						\begin{itemize}
							\item \textbf{Range di accettazione}: 0 - 0.8;
							\item \textbf{Range ottimale}: 0.3 - 0.7.
						\end{itemize}
				\paragraph{Chiamate innestate di metodi\\}
					Indica quante chiamate innestate di metodi sono inserite l'una
					all'interno dell'altra. Un alto valore può portare a una saturazione
					dello \gl{stack}.
					\subparagraph{Parametri utilizzati}
						\begin{itemize}
							\item \textbf{Range di accettazione}: 0 - 6;
							\item \textbf{Range ottimale}: 0 - 4.
						\end{itemize}
				\paragraph{Copertura del codice\\}
					Rappresenta la percentuale di codice eseguita durante i test.
					Maggiore è questo valore, più esaurienti saranno i test e maggiori
					sono le probabilità di individuare gli eventuali errori.
					\subparagraph{Parametri utilizzati}
						\begin{itemize}
							\item \textbf{Range di accettazione}: 80\% - 100\%;
							\item \textbf{Range ottimale}: 90\% - 100\%.
						\end{itemize}
				\paragraph{Numero di linee per metodo\\}
					Indica il numero di \gl{statement} che compongono un metodo.
					Se un metodo risulta troppo lungo il suo funzionamento risulterà più
					complicato da comprendere, quindi può essere opportuno dividerlo in
					più sotto-funzioni. Un metodo troppo lungo potrebbe addirittura essere
					sintomo di cattiva progettazione della classe.
					\subparagraph{Parametri utilizzati}
						\begin{itemize}
							\item \textbf{Range di accettazione}: $\leq$ 60;
							\item \textbf{Range ottimale}: $\leq$ 40.
						\end{itemize}
				\paragraph{Validazione W3C\\}
					L'applicativo web deve superare il test di validazione offerto da
					W3C con 0 errori gravi.
					Sono accettati avvisi ed inesattezze che non compromettano le
					funzionalità del sito fino a un massimo di 10 per pagina.
					\subparagraph{Parametri utilizzati}
						\begin{itemize}
							\item \textbf{Range di accettazione}: 0 - 10 (per pagina);
							\item \textbf{Range ottimale}: 0 - 0 (per pagina).
						\end{itemize}
			\subsubsection{Tabella riepilogativa}
				\begin{table}[H]
					\begin{tabular}{!{\VRule[1.4pt]}l!{\VRule}c!{\VRule}c!{\VRule[1.4pt]}}
						\noalign{\hrule height 1.4pt}
						\rowcolor{blue!30}\textbf{Metriche} & \textbf{Range di accettazione} & \textbf{Range ottimale} \\
						\noalign{\hrule height 1.4pt}
						\rowcolor{blue!20}\textbf{Metriche per i processi} & & \\ \hline
						Schedule Variance & $\geq -(preventivo*5\%)$ & $\geq 0$ \\ \hline
						Budget Variance & $\geq -(preventivo*10\%)$ & $\geq 0$ \\ \hline
						\noalign{\hrule height 1.4pt}
						\rowcolor{blue!20}\textbf{Metriche per i documenti} & & \\ \hline
						Indice Gulpease & 40 - 100 & 50 - 100 \\
						\noalign{\hrule height 1.4pt}
						\rowcolor{blue!20}\textbf{Metriche per il codice} &  &  \\ \hline
						Linee di commento su linee di codice & $\geq$ 0.25 & $\geq$ 0.30 \\ \hline
						Numero di parametri & 0 - 8 & 0 - 5  \\ \hline
						Numero di campi dati & 0 - 16 & 0 - 10 \\ \hline
						Complessità ciclomatica & 0 - 10 & 0 - 6  \\ \hline
						Livello di annidamento & 0 - 6 & 0 - 4 \\ \hline
						Grado di instabilità & 0 - 0.8 & 0.3 - 0.7 \\ \hline
						Chiamate innestate di metodi & 0 - 6 & 0 - 4 \\ \hline
						Copertura del codice & 80\% - 100\% & 90\% - 100\% \\ \hline
						Numero di linee per metodo & $\leq$ 60 & $\leq$ 40 \\ \hline
						Validazione W3C & 0 - 10 (per pagina) & 0 - 0 (per pagina) \\
						\noalign{\hrule height 1.4pt}
					\end{tabular}
					\caption{Riepilogo misure e metriche}
				\end{table}
\end{document}