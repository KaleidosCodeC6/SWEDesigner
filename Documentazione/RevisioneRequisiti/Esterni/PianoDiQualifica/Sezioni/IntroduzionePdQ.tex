\documentclass[../PianoDiQualifica.tex]{subfiles}
\begin{document}
	\section{Introduzione}
		\subsection{Scopo del documento}
			Questo documento definisce gli obbiettivi e le metodologie che ogni membro del gruppo
			\kaleidoscode\ adotterà per garantire un determinato livello di qualità del prodotto.
			A tal proposito ogni membro del gruppo è tenuto a leggere, perseguire e raggiungere gli obbiettivi definiti in esso.
		\subsection{Scopo del prodotto}
			Lo scopo del progetto è la realizzazione di un software di
			costruzione di diagrammi \gl{UML} con la relativa generazione
			di codice \gl{Java} e \gl{Javascript} utilizzando tecnologie
			web. Il prodotto deve essere conforme ai vincoli qualitativi richiesti dal committente.
		\subsection{Glossario}
			Al fine di evitare ogni ambiguità di linguaggio e massimizzare la
			comprensione dei documenti, i termini tecnici, di dominio, gli
			acronimi e le parole che necessitano di essere chiarite, sono
			riportate nel documento \glossariov.\\
			Ogni occorrenza di vocaboli presenti nel \textit{Glossario} è
			marcata da una ``G'' maiuscola in pedice.
		\subsection{Riferimenti utili}
			\subsubsection{Riferimenti normativi}
    			\begin{itemize}
    				\item \textbf{Capitolato d'appalto}:\\
    				\url{http://www.math.unipd.it/~tullio/IS-1/2016/Progetto/C6.pdf} (09/03/2017).
				\end{itemize}
			\subsubsection{Riferimenti informativi}	
				\begin{itemize}
					\item \textbf{Qualità del software (Slide del Corso di Ingegneria del Software)}:\\
					\url{http://www.math.unipd.it/~tullio/IS-1/2016/Dispense/L10.pdf} (09/03/2017);
					\item \textbf{Qualità di Processo (Slide del Corso di Ingegneria del Software)}:\\
					\url{http://www.math.unipd.it/~tullio/IS-1/2016/Dispense/L11.pdf} (09/03/2017);
					\item \textbf{Glossario}: \glossariov.
				\end{itemize}
\end{document}
