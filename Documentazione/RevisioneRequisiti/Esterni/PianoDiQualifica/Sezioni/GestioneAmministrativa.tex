\documentclass[../PianoDiQualifica.tex]{subfiles}
\begin{document}
	\section{Gestione amministrativa}
		\subsection{Definizione di un errore}
		Si verifica un errore quando si presenta una delle seguenti condizioni:
		\begin{itemize}
			\item Errore ortografico prodotto all'interno dei documenti;
			\item Violazione dei vincoli imposti dalle norme tipografiche nei documenti;
			\item Violazione dei vincoli imposti dalle norme tipografiche nel codice;
			\item Violazione degli indici delle misure e delle metriche prefissate;
			\item Violazione dei vincoli di prodotto.
		\end{itemize}
		\subsection{Comunicazione degli errori}
			La comunicazione degli errori viene svolta dai verificatori attraverso:
			\begin{itemize}
				\item \gl{Slack} nel caso di errori nella documentazione;
				\item La creazione di ``Issue'' su \gl{GitHub} nel caso di errori nei file di codice.
			\end{itemize}
		\subsection{Risoluzione degli errori}
			Ogni qualvolta vengono riscontrati degli errori, il team si riunisce per
			decidere come intervenire per risolverli. La risoluzione può quindi essere
			assegnata ad un solo membro del gruppo o ad un sottoinsieme ristretto
			di esso.
\end{document}