\documentclass[../PianoDiQualifica.tex]{subfiles}
\begin{document}
	\section{Visione generale della strategia}
		Per garantire la qualità dei prodotti realizzati durante lo sviluppo del
		progetto è indispensabile definire e perseguire strategie che assicurino la
		qualità dei processi adottati nonché il loro continuo miglioramento;
		inoltre, è necessario definire metriche e pianificare attività che valutino in
		modo preciso la qualità dei prodotti ottenuti e dei processi adottati. A tal
		scopo verranno adottate le seguenti strategie:
		\begin{itemize}
			\item Definizione accurata di norme che regolamentano e standardizzano i
			processi coinvolti nel progetto in termini di:
				\begin{itemize}
					\item Processi di fornitura;
					\item Processi di sviluppo;
					\item Processi di supporto;
					\item Processi organizzativi.
				\end{itemize}
			\item Descrizione dettagliata delle strategie di pianificazione adottate
			per lo sviluppo del progetto in termini di:
				\begin{itemize}
					\item Modello di sviluppo adottato;
					\item Analisi dei rischi che si possono incontrare;
					\item Pianificazione delle attività e dei tempi;
					\item Stima preventiva delle risorse che saranno impiegate;
					\item Assegnazione delle risorse al fine di portare a termine le
					attività pianificate nei tempi previsti;
					\item Consuntivo, durante lo sviluppo del progetto, delle risorse
					impiegate.
				\end{itemize}
			\item Ad ogni processo coinvolto nello sviluppo del progetto verrà applicato
			il principio \gl{PDCA} affiancato dal modello \gl{CMM}. Essi permettono il
			controllo, la valutazione e il miglioramento continuo dei processi nonché
			la determinazione del livello di maturità dell'organizzazione nel gestirli.
		\end{itemize}
		\subsection{Obiettivi di qualità}
			Prendendo come riferimento lo standard [ISO/IEC 9126] e lo
			standard [ISO/IEC 12207], il gruppo \kaleidoscode\ si impegna a
			garantire che il prodotto \progetto\ abbia le seguenti qualità:
			\subsubsection{Funzionalità}
				Si garantisce che il sistema prodotto abbia tutte le funzionalità
				indicate nel documento \analisideirequisitiv. L'implementazione di
				ogni requisito deve essere quanto più completa ed economica.
				\begin{itemize}
					\item \textbf{Misura}: l'unità di misura utilizzata sarà la
					quantità di requisiti mappati in componenti del sistema create
					e funzionanti.
					\item \textbf{Metrica}: la sufficienza è raggiunta quando
					vengono soddisfatti tutti i requisiti obbligatori.
					\item \textbf{Strumenti}: il sistema deve superare tutti i
					test che saranno previsti in fase di progettazione.
				\end{itemize}
			\subsubsection{Affidabilità}
				Il sistema deve essere quanto più possibile robusto. Nel caso di
				eventuali errori deve essere di facile ripristino.
				\begin{itemize}
					\item \textbf{Misura}: l'unità di misura utilizzata sarà la
					quantità di esecuzioni che vanno a buon fine.
					\item \textbf{Metrica}: poiché non è possibile valutare ancora
					tutte le casistiche di utilizzo, le esecuzioni dovranno
					coprire al meglio la gamma di possibilità. Risulta quindi
					impossibile stabilire oggettivamente una soglia di
					sufficienza.
					\item \textbf{Strumenti}: da definire.
				\end{itemize}
			\subsubsection{Usabilità}
				Il sistema deve coniugare la facilità di apprendimento e
				utilizzo con il soddisfacimento di tutte le necessità dell'utente.
				\begin{itemize}
					\item \textbf{Misura}: vista la mancanza di una metrica
					oggettiva adatta, l'unità di misura utilizzata sarà una
					valutazione soggettiva dell'usabilità.
					\item \textbf{Metrica}: non esistendo una metrica oggettiva
					adatta, è impossibile stabilire una soglia di sufficienza.
					I membri del gruppo si impegneranno comunque a garantire
					un'elevata esperienza d'uso.
					\item \textbf{Strumenti}: ci si affiderà ad un piccolo
					campione di utilizzatori esterni per avere un riscontro
					quantomeno realistico dell'usabilità. Per maggiori
					informazioni si vedano \normediprogettov.
				\end{itemize}
			\subsubsection{Efficienza}
				Il sistema deve ridurre al minimo l'utilizzo delle risorse
				impiegate e deve fornire le funzionalità richieste nel minor tempo
				possibile.
				\begin{itemize}
					\item \textbf{Misura}: il tempo di latenza dell'editor per
					eseguire un comando.
					\item \textbf{Metrica}: la sufficienza viene definita come un
					tempo di latenza inferiore ai 2 secondi.
					\item \textbf{Strumenti}: da definire.
				\end{itemize}
			\subsubsection{Manutenibilità}
				Il sistema deve essere comprensibile ed estensibile.
				\begin{itemize}
					\item \textbf{Misura}: le unità di misura utilizzate saranno
					quelle descritte nella sezione \ref{MetrichePerIlCodice}.
					\item \textbf{Metrica}: il prodotto deve ottenere la
					sufficienza in tutte le metriche descritte nella sezione
					\ref{MetrichePerIlCodice}.
					\item \textbf{Strumenti}: consultare sezione ``Norme di codifica''
					in \normediprogettov.
				\end{itemize}
			\subsubsection{Portabilità}
				Il sistema deve essere più portabile possibile. Il \gl{front-end}
				deve essere utilizzabile dal maggior numero di \gl{browser}
				possibili.
				\begin{itemize}
					\item \textbf{Misura}: il front-end deve rispettare gli
					standard del \gl{W3C}.
					\item \textbf{Metrica}: il software deve avere le
					caratteristiche di portabilità descritte. È necessario che il
					prodotto raggiunga la sufficienza in tutte le metriche
					descritte nella sezione \ref{MetrichePerIlCodice}.
					\item \textbf{Strumenti}: consultare \normediprogettov.	
				\end{itemize}
			\subsubsection{Altre qualità}
				Saranno importanti per il prodotto anche le seguenti proprietà:
				\begin{itemize}
					\item \textbf{Incapsulamento}: aumenta il riuso e la
					manutenibilità del codice. Saranno utilizzate interfacce ove
					possibile;
					\item \textbf{Coesione}: le funzionalità che hanno gli stessi
					obiettivi devono risiedere nello stesso componente in modo da
					favorire semplicità e manutenibilità, oltre a ridurre l'indice
					di dipendenza.
				\end{itemize}		
		\subsection{Organizzazione}
			La gestione della strategia di verifica si basa sull'attuazione delle
			relative attività descritte nelle \normediprogettov. Tali attività vengono
			eseguite per ogni processo attuato allo scopo di verifica della qualità del
			processo stesso e dell'eventuale prodotto ottenuto facendo riferimento
			anche alle metriche definite nella sezione \ref{Misure&Metriche}.
			Ogni documento prevede un diario delle modifiche che permette di concentrare
			l'attività di verifica solo nelle parti modificate dopo l'ultima eseguita.
			Data la diversa natura dei prodotti ottenuti dalle fasi del progetto si
			applicherà, per ognuno di essi, una diversa procedura di verifica:
			\begin{itemize}
				\item \textbf{Analisi e Analisi di dettaglio}: in questa fase si effettuerà
				una prima stesura dei documenti illustrati nel \pianodiprogettov.
				\begin{itemize}
					\item Verrà controllata la correttezza ortografica con \textit{LanguageTool 3.6},
					opportunamente integrato in \textit{TexStudio};
					\item Verrà controllata la correttezza lessicale con un'attenta ed accurata
					rilettura affiancata dal controllo di \textit{LanguageTool 3.6};
					\item Verrà controllata la correttezza dei contenuti rispetto alle aspettative
					del documento;
					\item Verrà controllata la corrispondenza di ciascun caso d'uso con un requisito,
					mediante l'utilizzo dell'applicativo web creato appositamente;
					\item Verrà controllato il rispetto delle \normediprogettov\ da parte di ciascun
					documento;
					\item Verranno controllate le rappresentazioni grafiche, figure e tabelle
					assicurandosi che per ciascuna di esse sia presente un'opportuna didascalia;
				\end{itemize}
				\item \textbf{Progettazione architetturale}: verrà controllato che tutti i requisiti
				corrispondano ad un componente individuato in questa fase e se ne assicurerà la
				tracciabilità;
				\item \textbf{Progettazione di dettaglio e Codifica}: durante ciascuna delle
				iterazioni di questa fase si svolgeranno i test per la verifica del codice. Tali
				attività avverranno nel modo più automatizzato possibile. I \verificatori\
				supervisioneranno questa fase controllando la presenza di eventuali errori;
				\item \textbf{Validazione e verifica}: in questa fase verrà effettuato il collaudo
				del prodotto, in modo da assicurare il suo corretto funzionamento al momento
				della consegna.
			\end{itemize} 
		\subsection{Scadenze temporali}
			Dato l'obiettivo di rispettare le scadenze fissate nel \pianodiprogettov, è
			indispensabile pianificare l'attività di verifica della documentazione e del
			codice prodotto in modo che risulti sistematica e organizzata. Grazie
			all'applicazione di tale strategia l'individuazione e la correzione degli
			errori avverrà il prima possibile, impedendo la loro rapida diffusione e
			mitigando la possibilità che gli stessi si ripresentino in futuro,
			diminuendo così il rischio di ritardi. Tale pianificazione è documentata nel
			\pianodiprogettov\ il quale contiene, nella sottosezione 1.4, anche le
			scadenze temporali che il gruppo \kaleidoscode\ si impegna a rispettare.
		\subsection{Responsabilità}
			I ruoli responsabili delle attività di verifica sono il
			\responsabilediprogetto\ e il \verificatore. I loro compiti e
			responsabilità, descritti nelle \normediprogettov, permettono alle attività
			di verifica di essere efficienti e sistematiche.
			Ogni componente del team è responsabile del materiale da lui prodotto.
\end{document}
