\documentclass[../PianoDiProgetto.tex]{subfiles}
\begin{document}
	\section{Introduzione}
		\subsection{Scopo del documento}
			Tale documento ha lo scopo di presentare le strategie di pianificazione adottate
			dal gruppo \kaleidoscode\  per sviluppare il progetto 	\progetto, in modo
			da garantire un'avanzamento controllato e mostrare le risorse utilizzate.
			Gli aspetti presi in considerazione sono:
			\begin{itemize}
				\item Modello di sviluppo adottato;
				\item Pianificazione dei tempi e delle attività;
				\item Stima preventiva delle risorse che saranno impiegate;
				\item Consultivo delle risorse impiegate, durante l'avanzamento del progetto;
			\end{itemize}
		\subsection{Scopo del prodotto}
			Lo scopo del progetto è la realizzazione di un software di
			costruzione di diagrammi \gl{UML} con la relativa generazione
			di codice \gl{Java} e \gl{Javascript} utilizzando tecnologie
			web.
		\subsection{Modello di sviluppo}
			Per lo sviluppo del progetto \progetto si è scelto di adottare il Modello
			incrementale; in quanto si è ritenuto adatto per via delle seguenti caratteristiche:
			\begin{itemize}
				\item Permettette di suddividere il progetto in più macrofasi, ognuna delle quali
				può essere suddivisa in diverse sottofasi, fino al grado di profondità ritenuto necessario; a questo consegue che è possibile applicare PDCA con accuratezza a diversi livelli di dettaglio, rendendo più 					affidabili i prodotti sviluppati;
				\item Sono previste 4 revisioni con il committente, il quale rilascia un feedback
				per i prodotti intermedi specifici della particolare revisione; il modello
				incrementale si adatta bene a questo tipo di situazione, in quanto se il
				feed è positivo si può incrementare ulteriolmente come pianificato; in caso
				contrario, la flessibilità del modello permette di correggere agevolmente il
				problema riscontrato;
				\item I requisiti vengono classificati in base alla loro priorità; il modello adottato
				permette di implementare prima quelli ritenuti di maggiore priorità e, una
				volta verificati e validati, il sistema viene incrementato con quelli di minore
				importanza;
				\item Tale modello garantisce una maggiore affidabilità del processo di sviluppo
				del progetto, riducendo il rischio di fallimento o ritardi di consegna; in
				quanto i cicli di incremento sono soggetti a PDCA;
			\end{itemize}
		\subsection{Scadenze}
				Di seguito viene riportata la tabella delle scadenze che il gruppo \kaleidoscode
				ha deciso di rispettare, in merito allo sviluppo del progetto \progetto.
				
				\begin{table}[H]
				\center
				\begin{tabular}{|cc|}
					\noalign{\hrule height 1.5pt}
					\textbf{Nome Revisione} & \textbf{Data} \\  [0.8cm]
					\hline
					Revisione dei requisiti (RR) & 18/04/2017\\ [0.8cm]
					\hline
					Revisione di progettazione (RP) & 15/05/2017 \\ [0.8cm]
					\hline
					Revisione di qualifica (RQ) & 27/06/2017 \\ [0.8cm]
					\hline
					Revisione di accettazione (RA) & 13/07/2017 \\ [0.8cm]
					\noalign{\hrule height 1.5pt}
			\end{tabular}
			\caption{Scadenze \label{tab:table_label}}
		\end{table}
		\subsection{Riferimenti}
			\subsubsection{Riferimenti normativi}
    			\begin{itemize}
    			\item \textbf{Software Engineering - Ian Sommerville - 9 th Edition 2010:} \\
    			- Part 4: Software managment; \\
			\end{itemize}
			\subsubsection{Riferimenti informativi}	
				\begin{itemize}
					\item \textbf{Organigramma e offerta tecnico-econonmica}:\\
					\url{http://www.math.unipd.it/~tullio/IS-1/2016/Progetto/PD01b.html};
					\item \textbf{Capitolato d'appalto C6: \progetto - Editor di diagrammi UML con 							generazione di codice:}:\\
					\url{http://www.math.unipd.it/~tullio/IS-1/2016/Progetto/C6.pdf} (03/03/2017);
					\item \textbf{Norme di Progetto}: \normediprogettov;
				\end{itemize}
\end{document}
