\documentclass[../PianoDiProgetto.tex]{subfiles}
\begin{document}
	\section{Organigramma}
		\subsection{Redazione}
			\begin{table}[H]
				\center
				\begin{tabularx}{\textwidth}{|X|X|X|}
					\noalign{\hrule height 1.5pt}
					\textbf{Nome} & \textbf{Data} & \textbf{Firma}     \\
					\hline
					  &  &\\
					\hline
					  &   &\\
					\noalign{\hrule height 1.5pt}
			\end{tabularx}
			\caption{Redazione.  \label{tab:table_label}}
		\end{table}
		
		\subsection{Approvazione}
			\begin{table}[H]
				\center
				\begin{tabularx}{\textwidth}{|X|X|X|}
					\noalign{\hrule height 1.5pt}
					\textbf{Nome} & \textbf{Data} & \textbf{Firma}     \\
					\hline
					  &  &\\
					\noalign{\hrule height 1.5pt}
			\end{tabularx}
			\caption{Approvazione.  \label{tab:table_label}}
		\end{table}
		
		\subsection{Accettazione componenti}
			\begin{table}[H]
				\center
				\begin{tabularx}{\textwidth}{|X|X|X|}
					\noalign{\hrule height 1.5pt}
					\textbf{Nome} & \textbf{Data} & \textbf{Firma}     \\
					\hline
					Bonato Marco  & 01/04/2017 &\\
					\hline
					Bonolo Enrico  & 01/04/2017 &\\
					\hline
					Pace Giulio  & 01/04/2017 & \\
					\hline
					Pezzuto Francesco  & 01/04/2017 &\\
					  \hline
					 Sanna Giovanni & 01/04/2017 &\\
					  \hline
					 Sovilla Matteo & 01/04/2017 &\\
					\noalign{\hrule height 1.5pt}
			\end{tabularx}
			\caption{Accettazione componenti.  \label{tab:table_label}}
		\end{table}
		
		\subsection{Componenti}
			\begin{table}[H]
				\center
				\begin{tabularx}{\textwidth}{|X|X|X|}
					\noalign{\hrule height 1.5pt}
					\textbf{Nome} & \textbf{Matricola} & \textbf{Posta elettronica}     \\
					\hline
					Bonato Enrico  & 1096071 & enrico.bonato.5@studenti.unipd.it \\
					\hline
					Bonolo Marco  & 1102360 & marco.bonolo@studenti.unipd.it\\
					\hline
					Pace Giulio  & 1102974 & giulio.pace@studenti.unipd.it\\
					\hline
					Pezzuto Francesco  & 1116523 & francesco.pezzuto@studenti.unipd.it\\
					\hline
					Sanna Giovanni & 1029744  & giovannibruno.sanna@studenti.unipd.it\\
					\hline
					Sovilla Matteo & 1124500 & matteo.sovilla@studenti.unipd.it\\
					\noalign{\hrule height 1.5pt}
			\end{tabularx}
			\caption{Accettazione componenti.  \label{tab:table_label}}
		\end{table}
		
		\subsection{Definizione ruoli}
			I ruoli e le responsabilità, indispensabili al corretto sviluppo del progetto \progetto ,  verranno ripartiti tra i componenti del gruppo in modo da rispettare le seguenti regole:
			\begin{itemize}
			\item Ogni singolo componente del gruppo potrà ricoprire più ruoli, sia contemporaneamente che in distinte fasi del progetto, in ogni caso sempre garantendo assenza di conflitto di interessi tra i ruoli assunti;
			\item Nella pianificazione, in cui si assegnano attività a risorse umane, è ammessa la duplicazione di ruoli, i quali devono però essere ricoperti da persone distinte;
			\item Il carico di lavoro individuale dovrà essere ripartito equamente tra i componenti del gruppo;
			\item Ogni componente, durante lo sviluppo del progetto, dovrà ricoprire almeno una volta ogni ruolo;
			\item L'impegno totale di ore rendicontabili presentate a consuntivo da ogni componente di un gruppo dovrà situarsi fra un minimo di 85 e un massimo di 105 ore produttive. Le ore rendicontabili non includono le attività di auto-formazione.
			\end{itemize}
			I compiti e le responsabilità di ogni ruolo sono indicati nel documento \normediprogettov.
		
\end{document}