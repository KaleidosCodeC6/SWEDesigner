\documentclass[../NormeDiProgetto.tex]{subfiles}
\begin{document}
	\subsection{Sviluppo}
		\subsubsection{Scopo}
			In questa fase vengono incluse tutte le attività volte a creare il prodotto.
		\subsubsection{Aspettative}
			I risultati ottenuti in seguito ad una corretta attuazione del
			\textit{Processo di sviluppo} sono:
			\begin{itemize}
				\item Realizzare un prodotto e/o servizio che soddisfi i requisiti concordati;
				\item Realizzare un prodotto e/o servizio che soddisfi le attività di validazione
				e verifica;
				\item Determinare eventuali vincoli tecnologici e requisiti;
				\item Determinare gli obiettivi di sviluppo.
			\end{itemize}
		\subsubsection{Descrizione}
			Prendendo come riferimento lo standard [ISO/IEC 12207], si devono svolgere le seguenti
			attività:
			\begin{itemize}
				\item Analisi dei requisiti;
				\item Progettazione;
				\item Codifica.
			\end{itemize}
		\subsubsection{Analisi dei requisiti}
			\paragraph{Scopo\\}
				Determinare tutti i requisiti del progetto. Il risultato di questa attività è un
				documento in cui vengono elencati i requisiti e i relativi casi d'uso.
			\paragraph{Aspettative\\}
				Produrre il documento \analisideirequisiti\ in conformità ai requisiti richiesti
				dal Proponente.
			\paragraph{Descrizione\\}
				Vengono analizzati e tracciati tutti i requisiti attraverso l'analisi della
				specifica del capitolato e la convocazione di riunioni con il proponente
				\proponente\ volte al chiarimento di eventuali dubbi o all'approfondimento di
				requisiti già noti.
			\paragraph{Classificazione dei requisiti\\}
				Viene stilata una lista dei requisiti individuati nel capitolato e nelle
				riunioni avvenute con il Proponente. La classificazione
				degli stessi deve avvenire secondo la seguente codifica:
				\begin{center}
					R[Importanza][Tipo][Codice]
				\end{center}
				dove:
				\begin{itemize}
					\item \textbf{Importanza} può assumere i seguenti valori:
					\begin{itemize}
						\item \textbf{0} se il requisito è obbligatorio;
						\item \textbf{1} se il requisito è desiderabile;
						\item \textbf{2} se il requisito è opzionale.
					\end{itemize}
					\item \textbf{Tipo} può assumere i seguenti valori:
					\begin{itemize}
						\item \textbf{F} se il requisito è funzionale;
						\item \textbf{Q} se il requisito è di qualità;
						\item \textbf{P} se il requisito è prestazionale;
						\item \textbf{V} se il requisito è di vincolo.
					\end{itemize}
					\item \textbf{Codice} identifica univocamente
					ciascun requisito in modo gerarchico.
				\end{itemize}
				Inoltre, per ciascun requisito deve essere indicata:
				\begin{itemize}
					\item \textbf{Fonte} dell'individuazione del requisito che può essere:
					\begin{itemize}
						\item \textbf{Capitolato};
						\item \textbf{Caso d'uso};
						\item \textbf{Interno} (discussioni del gruppo);
					\end{itemize}
					\item \textbf{Descrizione} breve e chiara.
				\end{itemize}
				Tutti i requisiti devono essere inseriti nell'apposito sistema di tracciamento ``SWEgo''
				realizzato all'interno del gruppo, che provvederà alla generazione
				dei codici identificativi univoci.
			\paragraph{Classificazione dei casi d'uso\\}
				L'analisi e l'identificazione dei casi d'uso, o use case (UC), deve
				procedere dal generale al particolare.\\
				Ciascun caso d'uso sarà classificato gerarchicamente con la seguente dicitura:
				\begin{center}
					UC[Codice del padre].[Codice identificativo]
				\end{center}
				dove:
				\begin{itemize}
					\item \textbf{Codice del padre} rappresenta il codice univoco
					del relativo caso d'uso padre qualora esistesse, altrimenti è omesso;
					\item \textbf{Codice identificativo} rappresenta il codice
					univoco e progressivo del corrispondente caso d'uso. Il codice
					può includere diversi livelli di gerarchia che devono essere
					separati da un punto.
				\end{itemize}
				Inoltre, per ciascun caso d'uso deve essere indicato:
				\begin{itemize}
					\item \textbf{Nome} del caso d'uso;
					\item \textbf{Attori} coinvolti;
					\item \textbf{Descrizione} chiara e sufficientemente
					dettagliata;
					\item \textbf{Precondizione};
					\item \textbf{Scenario principale degli eventi} che descrive il principale flusso di svolgimento e l'eventuale sequenza dei casi d'uso figli;
					\item \textbf{Scenari alternativi} che descrivono la sequenza
					di eventuali casi d'uso non appartenenti allo scenario
					principale;
					\item \textbf{Requisiti} ricavati dal caso d'uso;
					\item Eventuali \textbf{Inclusioni};
					\item Eventuali \textbf{Estensioni};
					\item Eventuali \textbf{Generalizzazioni};
					\item \textbf{Postcondizione}.
				\end{itemize}
				Infine, tutti i principali casi d'uso devono essere rappresentati da un diagramma UML.\\
				Tutti i casi d'uso devono essere inseriti nell'apposito sistema di tracciamento ``SWEgo''
				realizzato all'interno del gruppo, che provvederà alla generazione
				dei codici identificativi univoci.
		\subsubsection{Progettazione}
			\paragraph{Scopo\\}
				In questa fase viene descritta una soluzione del problema soddisfacente per tutti
				gli stakeholder.
			\paragraph{Aspettative\\}
				I risultati ottenuti in seguito ad una corretta attuazione di tale fase sono:
				\begin{itemize}
					\item Definire l'architettura logica di sistema che identifica gli elementi
					del sistema;
					\item Definire l'architettura logica di sistema in conformità ai requisiti
					definiti;
					\item Tracciare, verificare, mantenere coerenti e sincronizzati i requisiti di sistema con
					l'architettura logica di sistema;
					\item Garantire la qualità attraverso la correttezza per costruzione.
				\end{itemize}
			\paragraph{Descrizione\\}
				Partendo dai requisiti sviluppati nella fase precedente, i \progettisti\ devono
				sviluppare l'architettura logica del sistema seguendo le sottostanti linee guida:
				\begin{itemize}
					\item Utilizzare componenti con specifiche chiare e coese;
					\item Realizzare l'architettura rispettando le risorse e i costi prefissati;
					\item Adottare strutture che si adattino al cambiamento;
					\item Utilizzare componenti riusabili;
					\item Suddividere il sistema fino a quando ogni componente ha una
					complessità trattabile;
					\item Riconoscere le componenti terminali.
				\end{itemize}
			\paragraph{Diagrammi UML\\}
				Per definire l'architettura logica del sistema nella fase di progettazione,
				verranno utilizzati	diversi tipi di diagrammi UML 2.0:
				\begin{itemize}
					\item Diagrammi delle attività;
					\item Diagrammi delle classi;
					\item Diagrammi dei package;
					\item Diagrammi di sequenza.
				\end{itemize}
			\paragraph{Design pattern\\}
				I \progettisti\ devono descrivere i design pattern utilizzati per realizzare l'architettura.
				È necessario fornire una breve descrizione ed un diagramma semplificato per ognuno di essi.
		\subsubsection{Codifica}
			\paragraph{Scopo\\}
				Lo scopo della fase di codifica è quello di implementare il prodotto costruendo
				unità software eseguibili che riflettano la struttura definita in fase di
				progettazione.
			\paragraph{Aspettative\\}
				I risultati ottenuti in seguito ad una corretta attuazione di tale fase sono:
				\begin{itemize}
					\item Realizzare le unità software in conformità ai requisiti;
					\item Tracciare le unità software e le relativi componenti dell'architettura
					logica di sistema;
					\item Definire criteri di verifica delle unità software.
				\end{itemize}
			\paragraph{Descrizione\\}
				Partendo dall'architettura logica di sistema definita nella fase precedente, i
				\programmatori\ devono implementare le unità software definendone anche i criteri di
				verifica. Nello svolgimento di tali attività devono attenersi ai metodi e
				agli standard di codifica utilizzati.
			\paragraph{Standard di codifica\\}
				Ulteriori norme di codifica potrebbero essere aggiunte nelle fasi successive.
				\subparagraph{Intestazione dei file\\}
					Per i file contenenti codice si deve:
					\begin{itemize}
						\item Usare le seguente intestazione all'inizio di ogni file:\\
						\textcolor{purple}{	\\/*\\
							* File-Name: Nome del file\\
							* File-Author: Cognome Nome dell'autore\\
							* File-Date: Data di creazione\\
							* File-Summary: Breve descrizione del file\\
							* File-Description: Descrizione dettagliata del file\\
							**/}
						\item Prima di ogni classe scrivere un commento con la seguente struttura:\\
						\textcolor{purple}{	\\/*\\
							Class-Name: Nome della classe\\
							Class-Summary: Breve descrizione della classe\\
							**/}
						\item Per ogni metodo si deve scrivere un commento così strutturato:\\
						\textcolor{purple}{	\\/*\\
							Method-Name: Nome della classe\\
							Method-Summary: Breve descrizione della classe\\
							Method-Input: breve descrizione dei parametri della funzione nel caso
							ci siano\\
							Method-Output: breve descrizione dei valori di ritorno nel caso ci siano\\ 
							**/ }
					\end{itemize}
				\subparagraph{Codifica e convenzioni\\}
					Tutti i file devono seguire la convenzione UTF-8 per la codifica dei caratteri e
					LF (U+000A) per andare a capo.\\
					Per indicare il nome di variabili, classi, metodi e funzioni si deve usare la lingua
					inglese ed la notazione ``CamelCase''. È preferibile evitare di usare nomi ambigui che
					potrebbero generare confusione.\\
					Lo stile di indentazione del codice da seguire è una variante del ``K\&R'' dove lo stile
					di indentazione delle parentesi del corpo di una funzione è uguale a quello usato per
					i blocchi. L'indentazione deve essere effettuata utilizzando il tasto di tabulazione.\\
					È preferibile evitare lo sviluppo di metodi e funzioni ricorsive. Qualora fossero
					comunque usate si deve dimostrarne la corretta terminazione e valutarne l'impatto sulle
					performace del prodotto in termini di memoria utilizzata.\\
\end{document}
