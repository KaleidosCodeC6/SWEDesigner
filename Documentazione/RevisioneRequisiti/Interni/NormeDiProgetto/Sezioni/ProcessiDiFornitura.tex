\documentclass[../NormeDiProgetto.tex]{subfiles}
\begin{document}
	\section{Processo di Fornitura}
		\subsection{Scopo}
			Lo scopo del Processo di Fornitura è quello di consegnare un prodotto e/o un servizio che soddisfi i requisiti concordati.
		\subsection{Risultati}
			I risultati ottenuti in seguito ad una corretta attuazione del Processo di Fornitura sono:
			\begin{itemize}
				\item stabilire un accordo tra fornitore e proponente in merito allo sviluppo, al mantenimento, al funzionamento, alla consegna e all'installazione del prodotto e/o servizio
				\item realizzare un prodotto e/o servizio che soddisfi i requisiti concordati
				\item consegnare il prodotto al proponente in conformità con i requisiti concordati
				\item installare il prodotto in conformità con i requisiti concordati
			\end{itemize}
		\subsection{Descrizione}
			Rispettando lo standard [ISO/IEC 12207], il fornitore deve svolgere le seguenti attività:
			\begin{itemize}
				\item identificazione opportunità (\studiodifattibilita\)
				\item accordo contrattuale
				\item esecuzione del contratto
				\item consegna e supporto del prodotto e/o servizio
				\item chiusura
			\end{itemize}
		\subsection{Identificazione opportunità}
			Successivamente alla pubblicazione dei capitolati d'appalto, il
			\responsabilediprogetto\ ha il compito di convocare il numero
			di riunioni necessarie al confronto tra i membri del gruppo sui
			capitolati proposti. In questo modo, gli \analisti\ avranno modo
			di ottenere una base riguardante le conoscenze e preferenze di ogni
			membro del gruppo. Sulla base delle decisioni prese, gli
			\analisti\ devono redigere uno \studiodifattibilita\
			dei capitolati secondo:
			\begin{itemize}
				\item \textbf{Dominio tecnologico}: conoscenze sulle
				tecnologie impiegate nello sviluppo del progetto in questione;
				\item \textbf{Dominio applicativo}: conoscenze sul dominio di
				applicazione del prodotto;
				\item \textbf{Individuazione di rischi e criticità}: punti
				critici ed eventuali rischi percorribili durante lo sviluppo.
			\end{itemize}
			Nello \studiodifattibilita\ sono racchiuse le motivazioni che hanno spinto il nostro gruppo a candidarsi come fornitore per il proponente \proponente\
		\subsection{Accordo contrattuale}
			Il fornitore deve accordarsi con il proponente \proponente\ per chiarire, definire e accettare le richieste presenti nel documento di presentazione del capitolato fornito dal proponente.
		\subsection{Esecuzione del contratto}
			Il fornitore è tenuto a collaborare con il proponente \proponente\ per tutta la durata del progetto al fine di raggiungere i seguenti obbiettivi:
			\begin{itemize}
				\item Chiarire ogni dubbio riguardante i vincoli sui requisiti
				\item Chiarire ogni dubbio riguardante i vincoli di progetto
			\end{itemize}
			Il fornitore è tenuto a procurare al proponente \proponente\ e ai committenti \vardanega\ e \cardin\ i seguenti documenti:
			\begin{itemize}
				\item \pianodiprogetto\
				\item \analisideirequisiti\
				\item \pianodiqualifica\
			\end{itemize}
		\subsection{Consegna e supporto del prodotto e/o servizio}
			Dopo aver terminato le fasi di sviluppo, verifica e validazione, il fornitore è tenuto a consegnare al proponente \proponente\ il prodotto realizzato in conformità con i requisiti richiesti. Dovrà quindi consegnare:
			\begin{itemize}
				\item CD ROM contenente il prodotto realizzato
				\item Manuale di installazione
				\item Manuale d'uso
			\end{itemize}
			Il fornitore, dopo la consegna del prodotto, non si occuperà della fase di manutenzione del prodotto.
		\subsection{Chiusura}
			La chiusura dell'accordo contrattuale tra fornitore e proponente è sancita dalla consegna del prodotto realizzato.
\end{document}