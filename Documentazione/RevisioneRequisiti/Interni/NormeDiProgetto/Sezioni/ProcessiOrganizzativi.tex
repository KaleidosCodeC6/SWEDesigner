\documentclass[../NormeDiProgetto.tex]{subfiles}
\begin{document}
	\section{Processi organizzativi}
		\subsection{Comunicazioni esterne}
			Per le comunicazioni esterne è stata creata la casella di posta
			elettronica:
			\begin{center}
				\mailkaleidoscode
			\end{center}
			Tale indirizzo deve essere l'unico canale di comunicazione tra il
			gruppo di lavoro e l'esterno.
			Il \responsabilediprogetto\ è l'unico ad accedere
			all'indirizzo ed è quindi l'unico a poter comunicare con il
			committente del progetto. È compito del \responsabilediprogetto\ informare
			i membri del gruppo delle discussioni avvenute e,
			qualora fosse necessario, inoltrargli il messaggio attraverso
			una \gl{mailing list}.
		\subsection{Comunicazioni interne}
			Per le comunicazioni interne viene utilizzato il sistema di
			comunicazione offerto in Asana.\\
			Tale sistema deve essere utilizzato dai membri del gruppo
			per comunicare tra loro. Tutte le conversazioni devono avere
			come destinatario l'indirizzo ``Kaleidos Code''.
			In questo modo, ogni componente è costantemente informato sullo
			scambio di informazioni interne.
			Qualora fosse necessario l'uso di e-mail, come ad esempio nel caso di
			un inoltro di messaggio da parte del \responsabilediprogetto, è stata creata una \gl{mailing list}:
			\begin{center}
				\mailinglist
			\end{center}
			Per facilitare le comunicazioni tra i membri del gruppo, viene
			utilizzato anche il sistema di messaggistica e videoconferenza
			Google Hangout.
			L'uso di quest'ultimo, nel caso in cui
			vengano prese decisioni	o emergano contenuti utili allo
			sviluppo del progetto, comporta l'obbligo di redigere un verbale
			da parte di un membro del gruppo, che pubblicherà attraverso il sistema
			di comunicazione di Asana e ne salverà una copia in Google Drive una volta
			terminata la conversazione. La verbalizzazione ha l'obiettivo di tenere
			traccia di ogni argomento discusso, in
			quanto una comunicazione verbale non documentata non è
			accettabile per il corretto svolgimento del progetto.\\
			Per una comunicazione istantanea è utilizzato anche il sistema
			di messaggistica Telegram. Si richiede che la conversazione
			venga documentata come sopra descritto.
		\subsection{Composizione e-mail e conversazioni}
			In questo paragrafo viene descritta la struttura che deve avere
			un messaggio sia per una comunicazione esterna che per una
			conversazione interna attraverso il servizio offerto in Asana e
			\gl{mailing list}.
			\subsubsection{Destinatario}
				\begin{itemize}
					\item \textbf{Interno - Asana}: l'unico indirizzo utilizzabile è
					il nome del gruppo: Kaleidos Code;
					\item \textbf{Interno - e-mail}: l'unico indirizzo utilizzabile è
					\mailkaleidoscode;
					\item \textbf{Esterno}: può variare a seconda che ci si debba
					riferire  al proponente, al \vardanega\ o al \cardin.
				\end{itemize}
			\subsubsection{Mittente}
				\begin{itemize}
					\item \textbf{Interno - Asana}: è rappresentato automaticamente
					dallo username del creatore della conversazione;
					\item \textbf{Interno - e-mail}: l'indirizzo di chi scrive
					il messaggio;
					\item \textbf{Esterno}: l'unico indirizzo utilizzabile è
					\mailkaleidoscode\ e deve essere usato solamente dal
					\responsabilediprogetto.
				\end{itemize}
			\subsubsection{Oggetto}
				L'oggetto deve essere chiaro ed esaustivo, possibilmente non
				confondibile con altri preesistenti.\\
				L'oggetto di una comunicazione, una volta avviata, non deve mai essere cambiato.\\
				Solamente per le e-mail, nel caso si debba
				comporre un messaggio di risposta vi è l'obbligo di aggiungere la
				particella ``Re:'' all'inizio dell'oggetto per poter distinguere i
				livelli di risposta; se si dovesse trattare di un inoltro, si deve
				usare invece la particella ``I:''.
			\subsubsection{Corpo}
				Il corpo di un messaggio deve contenere tutte le informazioni
				necessarie alla piena comprensione della comunicazione.\\
				Nel caso di e-mail, se alcune parti del messaggio hanno uno o più destinatari specifici,
				sarà necessario aggiungere il loro nome	prima del relativo paragrafo
				attraverso la segnatura	\textit{@destinatario};
				in Asana invece, si dovrà menzionare lo specifico destinatario
				attraverso la determinata funzionalità alla creazione del messaggio.\\
				Solamente per le e-mail, in caso di risposta od inoltro del
				messaggio, il contenuto aggiunto deve essere sempre messo in testa.
				Si consiglia di non cancellare il resto del messaggio,
				per consentire una visione completa della discussione.
			\subsubsection{Allegati}
				Qualora fosse necessario, è permesso l'uso di allegati. Possono
				essere usati ad esempio per inviare i verbali di una riunione dei
				membri del gruppo oppure di un incontro con il proponente o con
				il committente.
				
				
				\subsection{Riunioni}
				\subsubsection{Frequenza}
				Le riunioni interne del gruppo di lavoro avranno una cadenza settimanale.
				\subsubsection{Convocazione riunione}
				\paragraph{Interna\\}
				Il \responsabilediprogetto\ ha il compito di convocare le riunioni generali interne, ossia
				le riunioni a cui sono convocati tutti i membri del gruppo.
				Se un componente qualsiasi ritiene necessaria la convocazione di una riunione, deve
				inoltrare la richiesta al responsabile il quale decide se respingere o accettare
				tale richiesta.
				Il \responsabilediprogetto\ deve convocare l'assemblea con almeno quattro giorni di preavviso
				attraverso l'invio di una richiesta nella sezione Comunicazioni di Asana, il cui
				corpo è formato da:
				\begin{itemize}
					\item Oggetto: Convocazione riunione n. X (dove X indica il numero progressivo
					di riunioni effettuate)
					\item Corpo:
					\begin{itemize}
						\item Data e ora prevista
						\item Luogo previsto
						\item Ordine del giorno
					\end{itemize}
				\end{itemize}
				Ogni componente deve rispondere al messaggio nel minor tempo possibile confermando
				la presenza o in caso contrario motivando l'eventuale assenza. In caso di manca risposta
				il \responsabilediprogetto\ è obbligato a contattare direttamente il componente che non
				ha fornito una risposta. Una volta ricevute tutte le risposte, il \responsabilediprogetto\
				può decidere se confermare, rinviare o annullare la riunione, basandosi sul numero di
				presenze e/o assenze, per garantire il massimo numero di presenti. La conferma, il rinvio,
				l'annullamento devono essere effettuati tramite messaggio nella sezione Comunicazioni di cui
				sopra.
				\paragraph{Esterna\\}
				Il \responsabilediprogetto\ ha il compito di convocare le riunioni generali esterne, ossia
				le riunioni a cui sono convocati tutti i membri del gruppo e il proponente e/o committente.
				Se un componente qualsiasi ritiene necessaria la convocazione di una riunione, deve
				inoltrare la richiesta al \responsabilediprogetto\ il quale decide se respingere o accettare
				tale richiesta.
				Il \responsabilediprogetto\ deve prima di tutto concordare con il proponente e/o committente delle date e dei luoghi
				possibili in cui svolgere la riunione attraverso l'invio di un'email contenente:
				\begin{itemize}
					\item Oggetto: Richiesta riunione
					\item Corpo
					\begin{itemize}
						\item Motivazione
						\item Eventuali date e/o luoghi possibili
					\end{itemize}
				\end{itemize}
				Dopo aver concordato delle date possibili, il responsabile deve inviare un messaggio a
				i membri del gruppo nella sezione Comunicazioni in Asana in cui sono specificate:
				\begin{itemize}
					\item Oggetto: Richiesta riunione col proponente
					\item Corpo
					\begin{itemize}
						\item Date e/o luoghi possibili
					\end{itemize}
				\end{itemize}
				Ogni membro del gruppo deve rispondere alla comunicazione confermando la presenza o in caso contrario motivando l'eventuale essenza. In caso di mancata risposta, il \responsabilediprogetto\ deve contattare direttamente il componente che non ha fornito una risposta. Una volta ricevute tutte le risposte, il responsabile può decidere se confermare, rinviare, annullare la riunione con il proponente basandosi sul numero di presenze e /o assenze.
				Il \responsabilediprogetto\ deve poi, in caso di:
				\begin{itemize}
					\item conferma: comunicare a tutti i membri e al proponente e/o committente la data e il luogo definitivi;
					\item rinvio: comunicare a tutti i membri la decisione presa e ripetere il procedimento dall'inizio;
					\item annullamento: comunicare a tutti i membri la decisione presa e ripetere il procedimento dall'inizio.
				\end{itemize}
				\subsubsection{Svolgimento riunione}
				\paragraph{Esterna\\}
				All'inizio di ogni riunione, verificata la presenza dei membri previsti, viene scelto un segretario che avrà il compito di annotare gli argomenti trattati e di redigere il verbale della riunione, che dovrà poi essere inviato ai restanti membri del gruppo. 
				Tutti i partecipanti devono tenere un comportamento consono al miglior svolgimento dell'assemblea e al raggiungimento degli obbiettivi della stessa. Il segretario deve inoltre assicurarsi che venga seguito l'ordine del giorno in modo da affrontare ogni argomento previsto.
				\subsubsection{Verbale}
				\paragraph{Riunione interna\\}
				Il verbale di Riunione interna è un documento informale che permette di tenere traccia degli argomenti discussi in ogni riunione. Il segretario, scelto a rotazione, ha il compito di redigere questo documento. Il verbale prodotto deve poi essere condiviso con tutti i membri del gruppo con Google Drive, in modo da rendere sempre disponibile e accessibile il contenuto dello stesso.
				\paragraph{Riunione esterna\\}
				In caso di riunione esterna con il proponente e/o committente, il verbale è un documento ufficiale che può avere valore normativo e quindi deve essere redatto seguendo criteri specifici.
				Per agevolare la scrittura di tale documento è stato preparato un template \LaTeX\ che ne definisce la struttura e ne organizza i contenuti. Vi è quindi l'obbligo di seguire il sopraccitato schema per creare il verbale, di inviare il verbale prodotto come allegato al proponente e/o committente in risposta all'email della riunione e di condividere il 
				verbale prodotto con tutti i membri del gruppo con Google Drive.

\end{document}
