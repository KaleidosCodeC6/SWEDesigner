\documentclass[../NormeDiProgetto.tex]{subfiles}
\begin{document}
	\section{Processo di sviluppo}
		\subsection{Scopo}
			In questa fase vengono incluse tutte le attività volte a creare il prodotto.
		\subsection{Aspettative}
			I risultati ottenuti in seguito ad una corretta attuazione del Processo di Sviluppo sono:
			\begin{itemize}
				\item realizzare un prodotto e/o servizio che soddisfi i requisiti concordati
				\item realizzare un prodotto e/o servizio che soddisfi le attività di validazione e verifica
				\item determinare eventuali vincoli tecnologici
				\item determinare gli obbiettivi di sviluppo
			\end{itemize}
		\subsection{Descrizione}
			Rispettando lo standard [ISO/IEC 12207], si devono svolgere le seguenti attività:
			\begin{itemize}
				\item \analisideirequisiti\
				\item Progettazione
				\item Codifica
				\item Qualificazione
			\end{itemize}
		\subsection{Analisi dei requisiti}
			\subsubsection{Scopo}
				Determinare tutti i requisiti del progetto. Il risultato di questa attività è un documento in cui vengono elencati i requisiti e i relativi casi d'uso.
			\subsubsection{Aspettative}
				Produrre il documento \analisideirequisiti\ in conformità ai requisiti richiesti dal proponente.
			\subsubsection{Descrizione}
				Vengono analizzati e tracciati tutti i requisiti  attraverso l'analisi della specifica del capitolato e attraverso delle riunioni con il proponente \proponente\ volte al chiarimento di eventuali dubbi o all'approfondimento di requisiti già noti.
			\subsubsection{Classificazione dei requisiti}
				Una lista dei requisiti individuati dal capitolato stesso e dalle
				eventuali riunioni avvenute con il proponente. La classificazione
				degli stessi, deve avvenire secondo la seguente codifica:
				\begin{center}
					R[Importanza][Tipo][Codice]
				\end{center}
				dove:
				\begin{itemize}
					\item \textbf{Importanza} può assumere i seguenti valori:
					\begin{itemize}
						\item \textbf{0} se il requisito è obbligatorio;
						\item \textbf{1} se il requisito è desiderabile;
						\item \textbf{2} se il requisito è opzionale.
					\end{itemize}
					\item \textbf{Tipo} può assumere i seguenti valori:
					\begin{itemize}
						\item \textbf{F} se il requisito è funzionale;
						\item \textbf{Q} se il requisito è di qualità;
						\item \textbf{P} se il requisito è prestazionale;
						\item \textbf{V} se il requisito è di vincolo.
					\end{itemize}
					\item \textbf{Codice} è un codice che identifica univocamente
					ciascun requisito in modo gerarchico.
				\end{itemize}
				Inoltre, per ciascun requisito deve essere indicata:
				\begin{itemize}
					\item \textbf{Fonte} dell'individuazione del requisito che può essere:
					\begin{itemize}
						\item \textbf{Capitolato};
						\item \textbf{Caso d'uso};
						\item \textbf{Interno} (discussioni del gruppo);
					\end{itemize}
					\item \textbf{Descrizione} breve e chiara.
				\end{itemize}
			\subsubsection{Classificazione dei casi d'uso}
				L'analisi e l'identificazione dei casi d'uso, o use case (UC), deve
				procedere dal generale al particolare.\\
				Ciascun caso d'uso sarà classificato gerarchicamente con la seguente dicitura:
				\begin{center}
					UC[Codice del padre].[Codice identificativo]
				\end{center}
				dove:
				\begin{itemize}
					\item \textbf{Codice del padre} rappresenta il codice univoco
					del relativo caso d'uso padre qualora esistesse, altrimenti è omesso;
					\item \textbf{Codice identificativo} rappresenta il codice
					univoco e progressivo del corrispondente caso d'uso. Il codice
					può includere diversi livelli di gerarchia che devono essere
					separati da un punto.
				\end{itemize}
				Inoltre, per ciascun caso d'uso deve essere indicato:
				\begin{itemize}
					\item \textbf{Nome} del caso d'uso;
					\item \textbf{Attori} coinvolti;
					\item \textbf{Descrizione} chiara e sufficientemente
					dettagliata;
					\item \textbf{Precondizione};
					\item \textbf{Postcondizione};
					\item \textbf{Scenario principale degli eventi} che descrive
					la sequenza dei casi d'uso figli;
					\item \textbf{Scenari alternativi} che descrivono la sequenza
					di eventuali casi d'uso non appartenenti allo scenario
					principale;
					\item \textbf{Requisiti} ricavati dal caso d'uso;
					\item Eventuali \textbf{Inclusioni};
					\item Eventuali \textbf{Estensioni};
					\item Eventuali \textbf{Generalizzazioni}.
				\end{itemize}
				Infine, ogni caso d'uso deve essere rappresentato da un grafico \gl{UML}.
		\subsection{Progettazione}
			\subsubsection{Scopo}
			\subsubsection{Aspettative}
			\subsubsection{Descrizione}
		\subsection{Codifica}
			\subsubsection{Scopo}
			\subsubsection{Aspettative}
			\subsubsection{Descrizione}
		\subsection{Qualificazione}
			\subsubsection{Scopo}
			\subsubsection{Aspettative}
			\subsubsection{Descrizione}
\end{document}