\documentclass[../NormeDiProgetto.tex]{subfiles}
\begin{document}
	\section{Processi di sviluppo}
		\subsection{Analisi dei requisiti}
			\subsubsection{Studio di fattibilità e analisi dei rischi}
				Successivamente alla pubblicazione dei capitolati d'appalto, il
				\responsabilediprogetto\ ha il compito di convocare il numero
				di riunioni necessarie al confronto tra i membri del gruppo sui
				capitolati proposti. In questo modo, gli \analisti\ avranno modo
				di ottenere una base riguardante le conoscenze e preferenze di ogni
				membro del gruppo. Sulla base delle decisioni prese, gli
				\analisti\ devono redigere uno \studiodifattibilita\
				dei capitolati secondo:
				\begin{itemize}
					\item \textbf{Dominio tecnologico}: conoscenze sulle
					tecnologie impiegate nello sviluppo del progetto in questione;
					\item \textbf{Dominio applicativo}: conoscenze sul dominio di
					applicazione del prodotto;
					\item \textbf{Individuazione di rischi e criticità}: punti
					critici ed eventuali rischi percorribili durante lo sviluppo.
				\end{itemize}
			\subsubsection{Analisi dei requisiti}
				Una volta completato lo \studiodifattibilita\ gli analisti
				hanno il compito di redigere l'\analisideirequisiti\
				strutturata nei paragrafi seguenti.
				\paragraph{Classificazione dei requisiti\\}
					Una lista dei requisiti individuati dal capitolato stesso e dalle
					eventuali riunioni avvenute con il proponente. La classificazione
					degli stessi, deve avvenire secondo la seguente codifica:
					\begin{center}
						R[Importanza][Tipo][Codice]
					\end{center}
					dove:
					\begin{itemize}
						\item \textbf{Importanza} può assumere i seguenti valori:
						\begin{itemize}
							\item \textbf{0} se il requisito è obbligatorio;
							\item \textbf{1} se il requisito è desiderabile;
							\item \textbf{2} se il requisito è opzionale.
						\end{itemize}
						\item \textbf{Tipo} può assumere i seguenti valori:
						\begin{itemize}
							\item \textbf{F} se il requisito è funzionale;
							\item \textbf{Q} se il requisito è di qualità;
							\item \textbf{P} se il requisito è prestazionale;
							\item \textbf{V} se il requisito è di vincolo.
						\end{itemize}
						\item \textbf{Codice} è un codice che identifica univocamente
						ciascun requisito in modo gerarchico.
					\end{itemize}
					Inoltre, per ciascun requisito deve essere indicata:
					\begin{itemize}
						\item \textbf{Fonte} dell'individuazione del requisito che può essere:
						\begin{itemize}
							\item \textbf{Capitolato};
							\item \textbf{Caso d'uso};
							\item \textbf{Interno} (discussioni del gruppo);
						\end{itemize}
						\item \textbf{Descrizione} breve e chiara.
					\end{itemize}
				\paragraph{Classificazione dei casi d'uso\\}
					L'analisi e l'identificazione dei casi d'uso, o use case (UC), deve
					procedere dal generale al particolare.\\
					Ciascun caso d'uso sarà classificato gerarchicamente con la seguente dicitura:
					\begin{center}
						UC[Codice del padre].[Codice identificativo]
					\end{center}
					dove:
					\begin{itemize}
						\item \textbf{Codice del padre} rappresenta il codice univoco
						del relativo caso d'uso padre qualora esistesse, altrimenti è omesso;
						\item \textbf{Codice identificativo} rappresenta il codice
						univoco e progressivo del corrispondente caso d'uso. Il codice
						può includere diversi livelli di gerarchia che devono essere
						separati da un punto.
					\end{itemize}
					Inoltre, per ciascun caso d'uso deve essere indicato:
					\begin{itemize}
						\item \textbf{Nome} del caso d'uso;
						\item \textbf{Attori} coinvolti;
						\item \textbf{Descrizione} chiara e sufficientemente
						dettagliata;
						\item \textbf{Precondizione};
						\item \textbf{Postcondizione};
						\item \textbf{Scenario principale degli eventi} che descrive
						la sequenza dei casi d'uso figli;
						\item \textbf{Scenari alternativi} che descrivono la sequenza
						di eventuali casi d'uso non appartenenti allo scenario
						principale;
						\item \textbf{Requisiti} ricavati dal caso d'uso;
						\item Eventuali \textbf{Inclusioni};
						\item Eventuali \textbf{Estensioni};
						\item Eventuali \textbf{Generalizzazioni}.
					\end{itemize}
					Infine, ogni caso d'uso deve essere rappresentato da un grafico \gl{UML}.
	
\end{document}