\documentclass[../StudioDiFattibilita.tex]{subfiles}
\begin{document}
	\section{Capitolato C3}
		\subsection{Descrizione}
			Il progetto prevede la realizzazione di un applicativo web che disegni e descriva gli scenari di danno, con particolare focus sulle catastrofi naturali, che posso colpire un'azienda.
		\subsection{Dominio applicativo}
			L'applicazione è rivolta a ogni azienda, in quanto chiunque è potenzialmente interessato a conoscere i possibili rischi legati alla loro attività.
		\subsection{Dominio tecnologico}
			Il proponente non richiede uno stack tecnologico particolare. Suggerisce però le seguenti tecnologie:
			\begin{itemize}
				\item \textbf{Amazon Web Services} per l'archiviazione dati;
				\item \textbf{\gl{Asana}} per la gestione dei processi;
				\item \textbf{\gl{Bootstrap}} e \textbf{Javascript} per la realizzazione dell'applicazione;
				\item \textbf{Slack} per la comunicazione.
			\end{itemize}
		\subsection{Valutazione}
			\subsubsection{Aspetti positivi}
				Gli aspetti ritenuti positivi di questo progetto sono:
				\begin{itemize}
					\item Lo stack tecnologico consigliato dal proponente sembra ragionevolmente semplice da utilizzare;
					\item Il progetto nel complesso riguarda una tematica interessante per alcuni membri del gruppo.
				\end{itemize}					
			\subsubsection{Fattori di rischio}
			I fattori di rischio che sono stati individuati sono:
			\begin{itemize}
				\item Azienda con sede all'estero e quindi con possibilità limitate per incontri;.
			\end{itemize}
			\subsection{Conclusioni}
			Il progetto sembra interessante per lo stack tecnologico consigliato e per la tematica affrontata. Il fatto che l'azienda proponente si trovi all'estero, però, è un fattore di rischio molto alto in quanto il gruppo teme che la comunicazione possa essere difficile e frammentaria. Per questo motivo il gruppo \kaleidoscode\ ha preferito scartare questo capitolato.
\end{document}
