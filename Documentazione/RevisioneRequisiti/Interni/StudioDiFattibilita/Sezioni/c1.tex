\documentclass[../StudioDiFattibilita.tex]{subfiles}
\begin{document}
	\section{Capitolato C1}
		\subsection{Descrizione}
		Il capitolato richiede lo sviluppo di un API market che consiste in un sito web dove gli utente possono condividere e monitorare l'uso di \gl{microservizi} che hanno creato.
		\subsection{Dominio applicativo}
		L'applicativo è rivolto a chiunque, sia azienda che privato, voglia condividere o usare un microservizio.
		\subsection{Dominio tecnologico} 
		I vincoli tecnologici imposti dal capitolato sono:
		\begin{itemize}
			\item I microservizi che si vogliono condividere devono essere scritti utilizzando il linguaggio \gl{Jolie};
			\item Il sito web deve essere creato tramite Javascript, HTML e CSS3.
		\end{itemize}
		Per le altri componenti viene lasciata libertà di scelta.
		\subsection{Valutazione}
			\subsubsection{Aspetti positivi}
			I principali aspetti ritenuti positivi dal gruppo sono:
				\begin{itemize}
				\item Le architetture a microservizi sono una tecnologia ritenuta interessante e innovativa dai membri del gruppo; 
				\item I linguaggi per la parte web, con i quali i membri del gruppo hanno una certa confidenza.
			\end{itemize}
			\subsubsection{Fattori di rischio}
				\begin{itemize}
				\item I microservizi sono un ambito poco conosciuto da parte del gruppo e potrebbero rivelarsi insidiosi;
				\item L'azienda proponente potrebbe avere più difficoltà e meno flessibilità negli incontri essendo un'azienda con poco personale.
			\end{itemize}
			\subsection{Conclusioni}
			Data la mancata esperienza in ambito di microservizi il gruppo \kaleidoscode\ ha preferito scartare questo capitolato.
\end{document}