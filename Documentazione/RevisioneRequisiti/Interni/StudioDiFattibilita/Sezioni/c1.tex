\documentclass[../StudioDiFattibilita.tex]{subfiles}
\begin{document}
	\section{Capitolato C1}
		\subsection{Descrizione}
		Il \gl{capitolato} richiede lo sviluppo di un API marchet che consiste in un sito web dove gli utente possono condividere e monitorare l'uso di \gl{microsevizi} che hanno creato.
		\subsection{Dominio applicativo}
		L'applicativo è rivolto a chiunque, sia azienda che privato, voglia condividere o usare un \gl{microservizio}.
		\subsection{Dominio tecnologico} 
		I vincoli tecnologici imposti dal \gl{capitolato} sono:
		\begin{itemize}
			\item I \gl{microservizzi} che si vogliono condividere devono essere scritti in \gl{Jolie};
			\item	Il sito web deve essere creato tramite \gl{javascript}, \gl{HTML} e \gl{CSS}3.
		\end{itemize}
		Per le altri componenti viene lasciata libertà di scelta.
		\subsection{Valutazione}
			\subsubsection{Aspetti positivi}
			I principali aspetti ritenuti positivi dal gruppo sono:
				\begin{itemize}
				\item Le architetture a microservizi sono una tecnologia ritenuta interessante e innovativa dai membri del gruppo; 
				\item I linguaggi per la parte web linguaggia con i quali i membri del gruppo hanno una certa confidenza.
			\end{itemize}
			\subsubsection{Fattori di rischio}
				\begin{itemize}
				\item I \gl{microservizi} sono un'ambito poco conosciuto da parte del gruppo e potrebbero rivelarsi insidiosi;
				\item L'azienda \textit{proponente} potrebbe avere più difficolta e meno flessibiliatà negli incontri essendo un'azienda con meno personale.
			\end{itemize}
			\subsection{Conclusioni}
			Data la mancata esperianza in ambito di \gl{microservizi} il gruppo ha preferito scartare questo \gl{capitolato}.
\end{document}