\documentclass[../StudiodiFattibilita.tex]{subfiles}
\begin{document}
	\section{Introduzione}
		\subsection{Scopo del documento}
			L'obiettivo del documento è quello di esprimere le motivazioni che hanno portato il gruppo a scegliere il \gl{capitolato} C6 - \progetto: editor di diagrammi \gl{UML} con generazione di codice Platform.
			Inoltre vengono riportati le motivazioni che hanno fatto scartare gli altri capitolati.
		\subsection{Glossario}
			Al fine di evitare ogni ambiguità di linguaggio e massimizzare la
			comprensione dei documenti, i termini tecnici, di dominio, gli
			acronimi e le parole che necessitano di essere chiarite, sono
			riportate nel documento \glossariov.\\
			Ogni occorrenza di vocaboli presenti nel \textit{Glossario} è
			marcata da una ``G'' maiuscola in pedice.
		\subsection{Riferimenti}
			\subsubsection{Normativi}
			\begin{itemize}
				\item \textbf{Norme di progetto}: \normediprogettov.
			\end{itemize}
			\subsubsection{Informativi}
			\begin{itemize}
				\item \textbf{Capitolato d'appalto C1} - APIM: An \gl{API} Market Platform\\
				\url{http://www.math.unipd.it/~tullio/IS-1/2016/Progetto/C1.pdf} (08/03/2017);
				\item \textbf{Capitolato d'appalto C2} - AtAVi: Accoglienza tramite Assistente Virtuale\\
				\url{http://www.math.unipd.it/~tullio/IS-1/2016/Progetto/C2.pdf} (08/03/2017);
				\item \textbf{Capitolato d'appalto C3} - DeGeOP: A Designer and Geo-ocalizer Web App for Organizational Plants\\ 
				\url{http://www.math.unipd.it/~tullio/IS-1/2016/Progetto/C3.pdf} (08/03/2017);
				\item \textbf{Capitolato d'appalto C4} - eBread: applicazione di lettura per dislessici \\
				\url{http://www.math.unipd.it/~tullio/IS-1/2016/Progetto/C4.pdf} (08/03/2017);
				\item \textbf{Capitolato d'appalto C5} - Monolith: an interactive bubble provider\\
				\url{http://www.math.unipd.it/~tullio/IS-1/2016/Progetto/C5.pdf} (08/03/2017);
				\item \textbf{Capitolato d'appalto C6} - SWEDesigner: editor di diagrammi UML con generazione di codice Platform\\
				\url{http://www.math.unipd.it/~tullio/IS-1/2016/Progetto/C6.pdf} (08/03/2017);
				\item \textbf{Glossario}: \glossariov.
			\end{itemize}
\end{document}