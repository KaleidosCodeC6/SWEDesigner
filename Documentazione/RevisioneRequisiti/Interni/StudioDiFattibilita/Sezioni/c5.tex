\documentclass[../StudioDiFattibilita.tex]{subfiles}
\begin{document}
	\section{Capitolato C5}
		\subsection{Descrizione}
		L'obiettivo di questo capitolato è quello di creare un \gl{framework} chiamato Monolith che permetterà di creare "interactive bubbles" facilmente (per bubbles sono intese tutte quelle features che possono essere aggiunte a un'applicazione di messaggistica come mini-giochi, sondaggi, ecc...)   
		\subsection{Dominio applicativo}
		Il framework potrà essere usato per sviluppare facilmente interactive bubbles che potranno essere usati all'interno di \gl{Rocket.Chat} (sito web di \gl{chat} open-source).
		\subsection{Dominio tecnologico}
		Per sviluppare il framework il proponente ha chiesto di utilizzare:
			\begin{itemize}
			\item La sesta edizione di Javascript come linguaggio principale; 
			\item \gl{GitHub} o \gl{Bitbucket} per il versionamento e la pubblicazione;
			\item CSS preferibilmente per la parte di stile;
			\item \gl{Heroku} come piattaforma cloud per l'esecuzione.
			\end{itemize}
		\subsection{Valutazione}
			\subsubsection{Aspetti positivi}
			Gli aspetti ritenuti positivi dal gruppo sono:
				\begin{itemize}
				\item Tematica generale del progetto interessante e attuale;
				\item Specifiche del progetto chiare e ben definite.
				\end{itemize}
			Le criticità riscontrate del gruppo sono:	
			\subsubsection{Fattori di rischio}
				\begin{itemize}
				\item Azienda con sede all'estero e quindi con possibilità limitate per incontri;
				\item Tecnologie da usare con molti vincoli e poco conosciute dal gruppo.
			\end{itemize}
			
			\subsection{Conclusioni}
				Il progetto sembra interessante per lo stack tecnologico consigliato e per la tematica affrontata. Il fatto che l'azienda proponente si trovi all'estero, però, è un fattore di rischio molto alto in quanto il gruppo teme che la comunicazione possa essere difficile e frammentaria. Per questo motivo il gruppo \kaleidoscode\ ha preferito scartare questo capitolato.
\end{document}