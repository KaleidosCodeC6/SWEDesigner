\documentclass[../StudioDiFattibilita.tex]{subfiles}
\begin{document}
	\section{Capitolato C2}
		\subsection{Descrizione}
		Si vuole creare un applicativo web che permetta un ospite in visita all'ufficio del proponente di essere accolto da un assistente virtuale. L'assistente deve annunciarne la presenza e intrattenerlo con varie attività mentre l'interessato del suo arrivo viene avvisato sul sistema di comunicazione aziendale (\gl{Slack}).
		\subsection{Dominio applicativo}
			L'applicazione è prodotta \textit{ad hoc} per uso e consumo del proponente. Pertanto il suo dominio applicativo risulta essere l'accoglienza di clienti nel contesto dell'azienda del proponente stesso.
		\subsection{Dominio tecnologico}
			Per realizzare il progetto il proponente richiede il seguente \gl{stack tecnologico}:
		\begin{itemize}
			\item \textbf{\gl{Amazon Web Services}}; 
			\item \textbf{\gl{NoSQL}}, \textbf{\gl{DynamoDB}} o \textbf{\gl{MongoDB}};
			\item \textbf{\gl{HTML}5}, \textbf{\gl{CSS}3} e \textbf{\gl{Javascript}} per l'interfaccia con l'utente;
			\item \textbf{\gl{Slack}} per il sistema di comunicazione;
			\item \textbf{\gl{Node.js}} \textbf{\gl{Swift}} come linguaggio di programmazione lato server;
			\item \textbf{\gl{SDK Alexa}} \textbf{\gl{Siri}} come assistente virtuale.
		\end{itemize}
		\subsection{Valutazione}
			\subsubsection{Aspetti positivi}
			Sono stati individuati i seguenti aspetti positivi:
			\begin{itemize}
				\item il progetto risulta stimolante e interessante a tutti i membri del gruppo;
				\item Il proponente ha fatto una buona impressione durante la presentazione dei capitolati e sembra particolarmente disponibile.
			\end{itemize}
			\subsubsection{Fattori di rischio}
			I fattori di rischio che sono stati individuati sono:
			\begin{itemize}
				\item Lo \gl{stack tecnologico} richiesto è vasto e impegnativo da padroneggiare. Inoltre la scelta di una tecnologia rispetto a un'altra potrebbe cambiare radicalmente il risultato finale;
				\item L'argomento è molto vasto e potrebbe nascondere insidie che emergerebbero durante le fasi successive di lavoro.
			\end{itemize}
			\subsection{Conclusioni}
				Nel complesso il \gl{capitolato} in questione risulta molto interessante al gruppo. La vastità dell'argomento e dello stack tecnologico da usare però hanno fatto propendere la scelta verso altri capitolati.
\end{document}