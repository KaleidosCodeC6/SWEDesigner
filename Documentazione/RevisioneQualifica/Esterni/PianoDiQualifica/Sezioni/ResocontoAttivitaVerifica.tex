\documentclass[../PianoDiQualifica.tex]{subfiles}
\begin{document}
	\section{Resoconto delle attività di verifica}\label{sez:resocontoAttivitaVerifica}
		\subsection{Periodo di Analisi e Analisi di dettaglio}
			\subsubsection{Processi}
				\begin{table}[H]
				\center
					\begin{tabular}{|l|c|c|}
						\hline
						\rowcolor{blue!30}\textbf{Documento} & \textbf{Schedule variance} & \textbf{Budget variance} \\ \hline
						\analisideirequisitiRR & 0\% & N.D. \\ \hline
						\glossarioRR & 0\% & N.D. \\ \hline
						\normediprogettoRR & 0\% & N.D. \\ \hline
						\pianodiprogettoRR & 0\% & N.D. \\ \hline
						\pianodiqualificaRR & 0\% & N.D. \\ \hline
						\studiodifattibilitaRR & 0\% & N.D. \\ \hline
					\end{tabular}
				\caption{RR - Schedule e budget variance}
				\end{table}
			\subsubsection{Indici di Gulpease}
				\begin{table}[H]
				\center
					\begin{tabular}{|l|c|c|}
						\hline
						\rowcolor{blue!30}\textbf{Documento} & \textbf{Valutazione} & \textbf{Esito} \\ \hline
						\analisideirequisitiRR & 41.08 & Accettabile \\ \hline
						\glossarioRR & 45.95 & Accettabile \\ \hline
						\normediprogettoRR & 46.95 & Accettabile \\ \hline
						\pianodiprogettoRR & 48.23 & Accettabile \\ \hline
						\pianodiqualificaRR & 53.99 & Ottimale \\ \hline
						\studiodifattibilitaRR & 46.73 & Accettabile \\ \hline
					\end{tabular}
				\caption{RR - Indici di Gulpease calcolati sulla documentazione prodotta}
				\end{table}
		\subsection{Periodo di Progettazione architetturale}
			\subsubsection{Processi}
				\begin{table}[H]
				\center
					\begin{tabular}{|l|c|c|}
						\hline
						\rowcolor{blue!30}\textbf{Documento} & \textbf{Schedule variance} & \textbf{Budget variance} \\ \hline
						\analisideirequisitiRP & 7\% & 0\% \\ \hline
						\glossarioRP & 0\% & 0\% \\ \hline
						\normediprogettoRP & 0\% & 0\% \\ \hline
						\pianodiprogettoRP & 0\% & 0\% \\ \hline
						\pianodiqualificaRP & -10\% & 0\% \\ \hline
						\specificatecnicaRP & 8\% & -5\% \\ \hline
					\end{tabular}
				\caption{RP - Schedule e budget variance}
				\end{table}
			\subsubsection{Indici di Gulpease}
				\begin{table}[H]
				\center
					\begin{tabular}{|l|c|c|}
						\hline
						\rowcolor{blue!30}\textbf{Documento} & \textbf{Valutazione} & \textbf{Esito} \\ \hline
						\analisideirequisitiRP & 44.03 & Accettabile \\ \hline
						\glossarioRP & 44.98 & Accettabile \\ \hline
						\normediprogettoRP & 46.69 & Accettabile \\ \hline
						\pianodiprogettoRP & 50.42 &  Ottimale\\ \hline
						\pianodiqualificaRP & 52.64 & Ottimale \\ \hline
						\specificatecnicaRP & 42.84 & Accettabile \\ \hline
					\end{tabular}
				\caption{RP - Indici di Gulpease calcolati sulla documentazione prodotta}
				\end{table}
			\subsubsection{Progettazione}
				% GRADO DI ACCOPPIAMENTO AFFERENTE
				\begin{longtable}{|r|c|}
					\hline
					\rowcolor{blue!30}\textbf{Componente} & \textbf{Accoppiamento afferente} \\
					\hline
					\endhead
					Client::View::MainView & 0 \\ \hline
					Client::View::TitleBarView & 1 \\ \hline
					Client::View::ToolBarView & 1 \\ \hline
					Client::View::AddressView & 1 \\ \hline
					Client::View::EditPanelView & 1 \\ \hline
					Client::View::Paper & 1 \\ \hline
					Client::Model::Command & 3\\ \hline
					Client::Model::ConcreteCommand & 0 \\ \hline
					Client::Model::State & 1 \\ \hline
					Client::Model::DAO & 1 \\ \hline
					Client::Model::MainModel &4 \\ \hline
					Client::Model::TitleBarModel & 1 \\ \hline
					Client::Model::ToolBarModel & 5 \\ \hline
					Client::Model::PackageToolbar & 0 \\ \hline
					Client::Model::ClassToolbar & 0 \\ \hline
					Client::Model::ActivityToolbar & 0 \\ \hline
					Client::Model::BubbleToolbar & 0 \\ \hline
					Client::Model::AddressModel & 1 \\ \hline
					Client::Model::EditPanelModel & 2 \\ \hline
					Client::Model::ItemPanel & 0 \\ \hline
					Client::Model::DiagramTree & 1 \\ \hline
					Client::Model::Diagram & 5 \\ \hline
					Client::Model::PackageDiagram & 0 \\ \hline
					Client::Model::ClassDiagram & 0 \\ \hline
					Client::Model::ActivityDiagram & 0 \\ \hline
					Client::Model::BubbleDiagram & 0 \\ \hline
					Client::Model::RequestHandler::Sender & 1 \\ \hline
					Client::Model::RequestHandler::Receiver & 1 \\ \hline
					Server::RequestHandler::Sender & 1 \\ \hline
					Server::RequestHandler::Receiver & 1 \\ \hline
					Server::CodeGenerator::CodeGenerator & 1 \\ \hline
					Server::CodeGenerator::Parser::Parser & 2 \\ \hline
					Server::CodeGenerator::Coder::Coder & 2 \\ \hline
					Server::CodeGenerator::Coder::JavaCoder & 0 \\ \hline
					Server::CodeGenerator::Coder::JavaScriptCoder & 0 \\ \hline
					Server::CodeGenerator::Coder::CoderClass & 0 \\ \hline
					Server::CodeGenerator::Coder::CoderOperation & 0 \\ \hline
					Server::CodeGenerator::Coder::CoderParameter & 0 \\ \hline
					Server::CodeGenerator::Coder::CoderAttribute & 0 \\ \hline
					Server::CodeGenerator::Coder::CoderActivity & 0 \\ \hline
					Server::CodeGenerator::Coder::CodedProg & 1 \\ \hline
					Server::CodeGenerator::Coder::CoderElement & 6 \\ \hline
					Server::CodeGenerator::Builder::Builder & 1 \\ \hline
					Server::CodeGenerator::Zipper::Zipper & 1 \\ \hline
					Server::DAO & 1 \\ \hline
				\caption{RP - Grado di accoppiamento afferente}
				\end{longtable}
				Tutti i componenti sono almeno sul range accettabile.
				% GRADO DI ACCOPPIAMENTO EFFERENTE
				\begin{longtable}{|r|c|}
					\hline
					\rowcolor{blue!30}\textbf{Componente} & \textbf{Accoppiamento efferente} \\
					\hline
					\endhead
					Client::View::MainView & 9 \\ \hline
					Client::View::TitleBarView & 0 \\ \hline
					Client::View::ToolBarView & 0 \\ \hline
					Client::View::AddressView & 0 \\ \hline
					Client::View::EditPanelView & 0 \\ \hline
					Client::View::Paper & 1 \\ \hline
					Client::Model::Command & 0\\ \hline
					Client::Model::ConcreteCommand & 2 \\ \hline
					Client::Model::State & 1 \\ \hline
					Client::Model::DAO & 1 \\ \hline
					Client::Model::MainModel & 1 \\ \hline
					Client::Model::TitleBarModel & 0 \\ \hline
					Client::Model::ToolBarModel & 0 \\ \hline
					Client::Model::PackageToolbar & 1 \\ \hline
					Client::Model::ClassToolbar & 1 \\ \hline
					Client::Model::ActivityToolbar & 1 \\ \hline
					Client::Model::BubbleToolbar & 1 \\ \hline
					Client::Model::AddressModel & 0 \\ \hline
					Client::Model::EditPanelModel & 0 \\ \hline
					Client::Model::ItemPanel & 1 \\ \hline
					Client::Model::DiagramTree & 1 \\ \hline
					Client::Model::Diagram & 1 \\ \hline
					Client::Model::PackageDiagram & 1 \\ \hline
					Client::Model::ClassDiagram & 1 \\ \hline
					Client::Model::ActivityDiagram & 1 \\ \hline
					Client::Model::BubbleDiagram & 1 \\ \hline
					Client::Model::RequestHandler::Sender & 1 \\ \hline
					Client::Model::RequestHandler::Receiver & 1 \\ \hline
					Server::RequestHandler::Sender & 1 \\ \hline
					Server::RequestHandler::Receiver & 1 \\ \hline
					Server::CodeGenerator::CodeGenerator & 3 \\ \hline
					Server::CodeGenerator::Parser::Parser & 0 \\ \hline
					Server::CodeGenerator::Coder::Coder & 2 \\ \hline
					Server::CodeGenerator::Coder::JavaCoder & 1\\ \hline
					Server::CodeGenerator::Coder::JavaScriptCoder & 1\\ \hline
					Server::CodeGenerator::Coder::CoderClass & 1\\ \hline
					Server::CodeGenerator::Coder::CoderOperation & 1\\ \hline
					Server::CodeGenerator::Coder::CoderParameter & 1\\ \hline
					Server::CodeGenerator::Coder::CoderAttribute & 1\\ \hline
					Server::CodeGenerator::Coder::CoderActivity & 2\\ \hline
					Server::CodeGenerator::Coder::CodedProg & 0\\ \hline
					Server::CodeGenerator::Coder::CoderElement & 0\\ \hline
					Server::CodeGenerator::Builder::Builder & 1 \\ \hline
					Server::CodeGenerator::Zipper::Zipper & 1 \\ \hline
					Server::DAO & 0 \\ \hline
				\caption{RP - Grado di accoppiamento efferente}
				\end{longtable}
				Tutti i componenti sono almeno sul range accettabile.
			\subsection{Periodo di Progettazione di dettaglio e Codifica}
				\subsubsection{Processi}
					\begin{table}[H]
					\center
						\begin{tabular}{|l|c|c|}
							\hline
							\rowcolor{blue!30}\textbf{Documento} & \textbf{Schedule variance} & \textbf{Budget variance} \\ \hline
							\analisideirequisitiRQ & 0\% & 0\% \\ \hline
							\glossarioRQ & 0\% & 0\% \\ \hline
							\normediprogettoRQ & 0\% & 0\% \\ \hline
							\pianodiprogettoRQ & 0\% & 0\% \\ \hline
							\pianodiqualificaRQ & -5\% & 0\% \\ \hline
							\specificatecnicaRQ & 5\% & 0\% \\ \hline
							\definizionediprodottoRQ & -3\% & 8\% \\ \hline
							Codifica e Debug & -10\% & -15\% \\ \hline
						\end{tabular}
					\caption{RQ - Schedule e budget variance}
					\end{table}
					Si individua una forte anomalia nella budget variance per la codifica e debug. L'avanzo sulle attività di progettazione è sufficiente a mantenere in bilancio il preventivo, ma questo valore è indice di ingenuità significative durante la pianificazione.
				\subsubsection{Indici di Gulpease}
					\begin{table}[H]
					\center
						\begin{tabular}{|l|c|c|}
							\hline
							\rowcolor{blue!30}\textbf{Documento} & \textbf{Valutazione} & \textbf{Esito} \\ \hline
							\analisideirequisitiRQ &  &  \\ \hline
							\glossarioRQ &  &  \\ \hline
							\normediprogettoRQ &  &  \\ \hline
							\pianodiprogettoRQ &  &  \\ \hline
							\pianodiqualificaRQ &  &  \\ \hline
							\specificatecnicaRQ &  &  \\ \hline
							\definizionediprodottoRQ &  &  \\ \hline
						\end{tabular}
					\caption{RQ - Indici di Gulpease calcolati sulla documentazione prodotta}
					\end{table}
				\subsubsection{Progettazione}
					\begin{table}[H]
					\center
						\begin{tabular}{|l|c|c|}
							\hline
							\rowcolor{blue!30}\textbf{Metrica} & \textbf{Valutazione} & \textbf{Esito} \\ \hline
							Numero di violazioni delle norme di progettazione &  &  \\ \hline
						\end{tabular}
					\caption{RQ - Numero di violazioni delle norme di progettazione}
					\end{table}
					% GRADO DI ACCOPPIAMENTO AFFERENTE
					\begin{longtable}{|r|c|}
						\hline
						\rowcolor{blue!30}\textbf{Componente} & \textbf{Accoppiamento afferente} \\
						\hline
						\endhead
						Client::View::MainView & 0 \\ \hline
						Client::View::TitleBarView & 1 \\ \hline
						Client::View::ToolBarView & 1 \\ \hline
						Client::View::AddressView & 1 \\ \hline
						Client::View::EditPanelView & 1 \\ \hline
						Client::View::Paper & 1 \\ \hline
						Client::Model::Command & 3\\ \hline
						Client::Model::ConcreteCommand & 0 \\ \hline
						Client::Model::State & 1 \\ \hline
						Client::Model::DAO & 1 \\ \hline
						Client::Model::MainModel &4 \\ \hline
						Client::Model::TitleBarModel & 1 \\ \hline
						Client::Model::ToolBarModel & 5 \\ \hline
						Client::Model::PackageToolbar & 0 \\ \hline
						Client::Model::ClassToolbar & 0 \\ \hline
						Client::Model::ActivityToolbar & 0 \\ \hline
						Client::Model::BubbleToolbar & 0 \\ \hline
						Client::Model::AddressModel & 1 \\ \hline
						Client::Model::EditPanelModel & 2 \\ \hline
						Client::Model::ItemPanel & 0 \\ \hline
						Client::Model::DiagramTree & 1 \\ \hline
						Client::Model::Diagram & 5 \\ \hline
						Client::Model::PackageDiagram & 0 \\ \hline
						Client::Model::ClassDiagram & 0 \\ \hline
						Client::Model::ActivityDiagram & 0 \\ \hline
						Client::Model::BubbleDiagram & 0 \\ \hline
						Client::Model::RequestHandler::Sender & 1 \\ \hline
						Client::Model::RequestHandler::Receiver & 1 \\ \hline
						Server::RequestHandler::Sender & 1 \\ \hline
						Server::RequestHandler::Receiver & 1 \\ \hline
						Server::CodeGenerator::CodeGenerator & 1 \\ \hline
						Server::CodeGenerator::Parser::Parser & 2 \\ \hline
						Server::CodeGenerator::Coder::Coder & 2 \\ \hline
						Server::CodeGenerator::Coder::JavaCoder & 0 \\ \hline
						Server::CodeGenerator::Coder::JavaScriptCoder & 0 \\ \hline
						Server::CodeGenerator::Coder::CoderClass & 0 \\ \hline
						Server::CodeGenerator::Coder::CoderOperation & 0 \\ \hline
						Server::CodeGenerator::Coder::CoderParameter & 0 \\ \hline
						Server::CodeGenerator::Coder::CoderAttribute & 0 \\ \hline
						Server::CodeGenerator::Coder::CoderActivity & 0 \\ \hline
						Server::CodeGenerator::Coder::CodedProg & 1 \\ \hline
						Server::CodeGenerator::Coder::CoderElement & 6 \\ \hline
						Server::CodeGenerator::Builder::Builder & 1 \\ \hline
						Server::CodeGenerator::Zipper::Zipper & 1 \\ \hline
						Server::DAO & 1 \\ \hline
					\caption{RQ - Grado di accoppiamento afferente}
					\end{longtable}
					Tutti i componenti sono almeno sul range accettabile.
					% GRADO DI ACCOPPIAMENTO EFFERENTE
					\begin{longtable}{|r|c|}
						\hline
						\rowcolor{blue!30}\textbf{Componente} & \textbf{Accoppiamento efferente} \\
						\hline
						\endhead
						Client::View::MainView & 9 \\ \hline
						Client::View::TitleBarView & 0 \\ \hline
						Client::View::ToolBarView & 0 \\ \hline
						Client::View::AddressView & 0 \\ \hline
						Client::View::EditPanelView & 0 \\ \hline
						Client::View::Paper & 1 \\ \hline
						Client::Model::Command & 0\\ \hline
						Client::Model::ConcreteCommand & 2 \\ \hline
						Client::Model::State & 1 \\ \hline
						Client::Model::DAO & 1 \\ \hline
						Client::Model::MainModel & 1 \\ \hline
						Client::Model::TitleBarModel & 0 \\ \hline
						Client::Model::ToolBarModel & 0 \\ \hline
						Client::Model::PackageToolbar & 1 \\ \hline
						Client::Model::ClassToolbar & 1 \\ \hline
						Client::Model::ActivityToolbar & 1 \\ \hline
						Client::Model::BubbleToolbar & 1 \\ \hline
						Client::Model::AddressModel & 0 \\ \hline
						Client::Model::EditPanelModel & 0 \\ \hline
						Client::Model::ItemPanel & 1 \\ \hline
						Client::Model::DiagramTree & 1 \\ \hline
						Client::Model::Diagram & 1 \\ \hline
						Client::Model::PackageDiagram & 1 \\ \hline
						Client::Model::ClassDiagram & 1 \\ \hline
						Client::Model::ActivityDiagram & 1 \\ \hline
						Client::Model::BubbleDiagram & 1 \\ \hline
						Client::Model::RequestHandler::Sender & 1 \\ \hline
						Client::Model::RequestHandler::Receiver & 1 \\ \hline
						Server::RequestHandler::Sender & 1 \\ \hline
						Server::RequestHandler::Receiver & 1 \\ \hline
						Server::CodeGenerator::CodeGenerator & 3 \\ \hline
						Server::CodeGenerator::Parser::Parser & 0 \\ \hline
						Server::CodeGenerator::Coder::Coder & 2 \\ \hline
						Server::CodeGenerator::Coder::JavaCoder & 1\\ \hline
						Server::CodeGenerator::Coder::JavaScriptCoder & 1\\ \hline
						Server::CodeGenerator::Coder::CoderClass & 1\\ \hline
						Server::CodeGenerator::Coder::CoderOperation & 1\\ \hline
						Server::CodeGenerator::Coder::CoderParameter & 1\\ \hline
						Server::CodeGenerator::Coder::CoderAttribute & 1\\ \hline
						Server::CodeGenerator::Coder::CoderActivity & 2\\ \hline
						Server::CodeGenerator::Coder::CodedProg & 0\\ \hline
						Server::CodeGenerator::Coder::CoderElement & 0\\ \hline
						Server::CodeGenerator::Builder::Builder & 1 \\ \hline
						Server::CodeGenerator::Zipper::Zipper & 1 \\ \hline
						Server::DAO & 0 \\ \hline
					\caption{RQ - Grado di accoppiamento efferente}
					\end{longtable}
					Tutti i componenti sono almeno sul range accettabile.
					% METRICHE PER LA PROGETTAZIONE
					\paragraph{Metriche per la progettazione\\}
						Per evitare la creazione di tabelle troppo prolisse e complesse da consultare,
						sono stati riportati solo i valori minimi e massimi rilevati nelle misurazioni.
						%fan in fan out
						\begin{table}[H]
							\center
							\begin{tabular}{|l|c|c|c|c|}
								\hline
								\rowcolor{blue!30}\textbf{Metrica} & \textbf{Range Accettazione} & \textbf{Range Ottimale}&\textbf{04/07/2017}&\textbf{Esito} \\ \hline
								Fan In & & & min 0 - max 6 & Ottimo \\ \hline
								Fan Out & & & min 0 - max 5	& Ottimo \\ \hline
							\end{tabular}
							\caption{RQ - Fan In e Fan Out}
						\end{table}
						%fine fan
				\subsubsection{Codifica}	\begin{table}[H]
					\center
					\begin{tabular}{|>{\centering}p{6cm}|c|c|c|c|}
						\hline
						\rowcolor{blue!30}\textbf{Metrica} & \textbf{Accettazione} & \textbf{Ottimale}&\textbf{04/07/2017}&\textbf{Esito} \\ \hline
						Numero di violazioni delle norme di codifica & 0-10 & 0-5 & 4 & Ottimale\\ \hline
						\end{tabular}
					\caption{RQ - Numero di violazioni delle norme di codifica}
					\end{table}
					% METRICHE PER IL CODICE
					\paragraph{Metriche per il codice\\}
						Per evitare la creazione di tabelle troppo prolisse e complesse da consultare,
						sono stati riportati solo i valori minimi e massimi rilevati nelle misurazioni.
							\begin{table}[H]
							\center
							\begin{tabular}{|>{\centering}p{5cm}|c|c|c|c|}
								\hline
								\rowcolor{blue!30}\textbf{Metrica} & \textbf{Accettazione} & \textbf{Ottimale}&\textbf{04/07/2017}&\textbf{Esito} \\ \hline
								Percentuale totale di test superati &$80\%$ - $100\%$&$90\%$ - $100\%$&96\%&Ottimale\\ \hline
								Grado di accoppiamento afferente&$0$ - $7$&$0$ - $5$&
								min 0 - max 6 &Accettabile\\ \hline
								Grado di accoppiamento efferente &$0$ - $7$&$0$ - $5$&min 0 - max 9 & Non Acc. \\ \hline
								Linee di commento su linee di codice &$\geq 0.25$& $\geq 0.30$ &min 26\%-max 47\% &Accettabile\\ \hline
								Numero di parametri &$0$ - $8$&$0$ - $5$&min 0- max 5 & Ottimale\\ \hline
								Numero di campi dati &$0$ - $16$&$0$ - $10$&min 0 - max6&Ottimale\\ \hline
								Complessità ciclomatica &$0$ - $10$&$0$ - $6$ &&\\\hline
								Livello di annidamento &$0$ - $6$&$0$ - $4$&&\\ \hline
								Chiamate innestate di metodi &$0$ - $6$&$0$ - $4$ &&\\ \hline
								Copertura del codice & $80\%$ - $100\%$&$90\%$ - $100\%$ &&\\ \hline
								Numero di linee per metodo & $\leq 60$ & $\leq 40$ &min 3 - max 52 &Accettabile\\ \hline
								Validazione W3C & $0$ - $10$ (pagina) & $0$ - $0$ ( pagina)& min 0 - max 4&Accettabile \\ \hline
							\end{tabular}
							\caption{Metriche per qualità di prodotto}
						\end{table}
			%	\subsubsection{Test}
			%			\begin{table}[H]
			%				\center
			%				\begin{tabular}{|>{\cen/*tering}p{6cm}|c|c|c|c|}
			%					\hline
			%					\rowcolor{blue!30}\textbf{Metrica} & %\textbf{Accettazione} & %\textbf{Ottimale}&\textbf{04/07/2017}&\textbf{Esito} \\ \hline
			%					Percentuale di test di unità effettuati && &100\% &  \\ \hline
			%			\end{tabular}
			%		\caption{RQ - Verifica test}
			%		\end{table}
\end{document}
