\documentclass[../PianoDiProgetto.tex]{subfiles}
\begin{document}
	\section{Analisi dei Rischi}
	E' stata eseguita un'analisi dei principali rischi che il gruppo può incontrare
	durante lo sviluppo del progetto al fine di ottenere una migliore qualità di tale
	processo di sviluppo. Per ogni rischio, inoltre, viene determinato un metodo
	da seguire per prevenirlo e/o mitigarlo. Ciascun rischio verrà monitorato e ne
	verrà descritto l'effettivo riscontro durante l'avanzamento del progetto.
		\subsection{Livello tecnologico}
			\begin{table}[H]
				\center
				\begin{tabularx}{\textwidth}{X X}
					\noalign{\hrule height 1.5pt}
					\textbf{Nome Rischio} & Poca familiarità con tecnologie e strumenti adottati \\
					\hline
					\textbf{Descrizione}  & Alcuni componenti del gruppo non conoscono
					sufficientemente le tecnologie e/o gli strumenti
					di supporto che verranno utilizzati durante lo
					sviluppo del progetto. Inoltre non tutti hanno
					sostenuto i relativi esami.  \\
					\hline
					\textbf{Probabilità di occorrenza}  & Bassa  \\
					\hline
					\textbf{Effetto}  & Serio  \\
					\hline
					\textbf{Prevenzione}  & Tutti i componenti del gruppo si impegnano a
					colmare le proprie lacune in merito alle
					tecnologie e agli strumenti adottati per lo
					svolgimento del progetto.  \\
					\hline
					\textbf{Mitigazione}  & Qualora un componente non capisse in fondo una
tecnologia o uno strumento adottato, egli si impegna a documentarsi più approfonditamente;
inoltre, se possibile e inevitabile, può richiedere
l'aiuto di un componente più preparato. \\
					\hline
					\textbf{Riscontro} & Lo studio autonomo svolto da ogni componente del gruppo riguardo le tecnologie e strumenti adottati è stato finora adeguato a svolgere le attività richieste, e non ha portato a ritardi rispetto a quanto pianificato. Inoltre la comunicazione tra i componenti è stata gestita in maniera efficiente ed ha permesso, a chi ne avesse bisogno, di ricevere chiarimenti su aspetti non compresi appieno. \\
					\noalign{\hrule height 1.5pt}
			\end{tabularx}
			\caption{Poca familiarità con tecnologie e strumenti adottati. \label{tab:table_label}}
		\end{table}
		
		\begin{table}[H]
				\center
				\begin{tabularx}{\textwidth}{X X}
					\noalign{\hrule height 1.5pt}
					\textbf{Nome Rischio} & Malfunzionamenti hardware o software  \\
					\hline
					\textbf{Descrizione}  & Ogni componente del gruppo dispone di un personal computer con cui svolge il proprio lavoro
in merito al progetto \progetto; tali dispositivi sono di tipo commerciale e non professionale,
quindi è da tenere in considerazione la rottura
degli strumenti di lavoro. Il gruppo, per versionare i prodotti delle varie attività, utilizza un \gl{repository} remoto, il quale potrebbe avere malfunzionamenti che non permetterebbero di accedere
al proprio lavoro. \\
					\hline
					\textbf{Probabilità di occorrenza}  & Moderata \\
					\hline
					\textbf{Effetto}  & Tollerabile \\
					\hline
					\textbf{Prevenzione}  &  Ogni componente del gruppo avrà cura dei propri strumenti di lavoro. Il \responsabilediprogetto\ dovrà salvare, almeno una volta al giorno, il
contenuto del repository remoto in una personale periferica esterna di memorizzazione, nonchè
condividerla in un'apposita cartella in \gl{Google
Drive}. \\
					\hline
					\textbf{Mitigazione}  & Il gruppo possiede computer di riserva, in caso
di rotture di quelli in uso; in alternativa, sono disponibili i computer del laboratorio. Se si
dovesse verificare un malfunzionamento del repository remoto, sarà disponibile l'ultima copia
aggiornata del repository \\
					\hline
					\textbf{Riscontro} & Finora tale rischio non si è presentato; ogni componente ha avuto cura dei propri strumenti di lavoro e non sono state necessarie le contromisure di mitigazione stabilite come sopra.  \\
					\noalign{\hrule height 1.5pt}
			\end{tabularx}
			\caption{Malfunzionamenti hardware o software.  \label{tab:table_label}}
		\end{table}
		
		\subsection{Livello personale}
			\begin{table}[H]
				\center
				\begin{tabularx}{\textwidth}{X X}
					\noalign{\hrule height 1.5pt}
					\textbf{Nome Rischio} & Impegni personali dei membri del gruppo  \\
					\hline
					\textbf{Descrizione}  & Con molta probabilità i componenti del gruppo avranno impegni sporadici che non permetteranno loro di lavorare sul progetto secondo la
pianificazione predeterminata.  \\
					\hline
					\textbf{Probabilità di occorrenza}  &  Alto \\
					\hline
					\textbf{Effetto}  & Tollerabile \\
					\hline
					\textbf{Prevenzione}  & Il componenti del gruppo devono comunicare
tempestivamente al \responsabilediprogetto\
qualora avessero impegni o indisponibilità. \gl{Asana} è dotato di un calendario che permette di tenere traccia delle indisponibilità dei membri del gruppo. \\
					\hline
					\textbf{Mitigazione}  & Il \responsabilediprogetto\ deve prontamente riorganizzare le attività legate al membro indisponibile; oppure, qualora necessario,
ridistribuirle agli altri membri del gruppo.  \\
					\hline
					\textbf{Riscontro} & Tale rischio non è stato riscontrato fino a questo momento. \\
					\noalign{\hrule height 1.5pt}
			\end{tabularx}
			\caption{Impegni personali dei membri del gruppo. \label{tab:table_label}}
		\end{table}
		
		\begin{table}[H]
				\center
				\begin{tabularx}{\textwidth}{X X}
					\noalign{\hrule height 1.5pt}
					\textbf{Nome Rischio} &  Dissidi tra membri del gruppo \\
					\hline
					\textbf{Descrizione}  &  Il gruppo è composto da individui perlopiù sconosciuti tra loro, con caratteri e opinioni potenzialmente eterogenei; tra essi possono insorgere
incomprensioni o dissidi che danneggiano il morale all'interno del gruppo rendendo l'ambiente
di lavoro meno produttivo. \\
					\hline
					\textbf{Probabilità di occorrenza}  & Basso \\
					\hline
					\textbf{Effetto}  &  Serio \\
					\hline
					\textbf{Prevenzione}  & Il \responsabilediprogetto\ deve costantemente
monitorare i rapporti tra i componenti del gruppo, chiarendo sul nascere eventuali dissidi. Inoltre si impegna a mantenere un clima sereno tra i membri.  \\
					\hline
					\textbf{Mitigazione}  & In caso si verifichi questo scenario, il \responsabilediprogetto\ dovrà prendere in mano la situazione e mediare l'incontro tra i componenti
in contrasto, cercando di arrivare ad un accordo
comune. Nel caso questo non risultasse possibile, dovrà riorganizzare le attività in modo da
minimizzare la collaborazione tra i componenti in contrasto.  \\
					\hline
					\textbf{Riscontro} & Tale rischio non è stato riscontrato fino a questo momento; ogni componente si è impegnato a mantenere un clima sereno all'interno del gruppo e non è stata necessaria nessuna attività di mitigazione. \\
					\noalign{\hrule height 1.5pt}
			\end{tabularx}
			\caption{Dissidi tra membri del gruppo.  \label{tab:table_label}}
		\end{table}
		
		\begin{table}[H]
				\center
				\begin{tabularx}{\textwidth}{X X}
					\noalign{\hrule height 1.5pt}
					\textbf{Nome Rischio} & Inesperienza dei membri del gruppo \\
					\hline
					\textbf{Descrizione}  & Nessun componente del gruppo ha avuto esperienze riguardo lo sviluppo di progetti software
di grandi dimensioni; inoltre, nessuno ha mai lavorato in un \gl{team} così numeroso. Per ottenere
prodotti di qualità, è necessario conoscere a fondo le metodologie di creazione e gestione di grandi
progetti software; aspetto finora mai affrontato dai componenti del gruppo. \\
					\hline
					\textbf{Probabilità di occorrenza}  & Alto \\
					\hline
					\textbf{Effetto}  & Serio \\
					\hline
					\textbf{Prevenzione}  & Ogni componente del gruppo deve studiare gli
argomenti necessari a svolgere al meglio il progetto; inoltre, ognuno si impegna ad avere un
atteggiamento collaborativo all'interno del team e volto a massimizzare la qualità dei processi svolti. \\
					\hline
					\textbf{Mitigazione}  & Il \responsabilediprogetto\ pianifica attività di
studio per permettere, a chi ne avesse bisogno, di aggiornarsi sulle conoscenze necessarie
all'avanzamento del progetto. \\
					\hline
					\textbf{Riscontro} & Soprattutto nel primo periodo, l'approccio alla progettazione di sistemi software complessi è stato difficoltoso e non subito compreso fino in fondo. Il \responsabilediprogetto ha preventivamente pianificato ore aggiuntive ad ogni componente al fine di una migliore auto formazione; mitigando tale rischio.  \\
					\noalign{\hrule height 1.5pt}
			\end{tabularx}
			\caption{Inesperienza dei membri del gruppo \label{tab:table_label}}
		\end{table}
		
		\subsection{Livello organizzativo}
			\begin{table}[H]
				\center
				\begin{tabularx}{\textwidth}{X X}
					\noalign{\hrule height 1.5pt}
					\textbf{Nome Rischio} & Stima errata di costi e/o tempi delle attività \\
					\hline
					\textbf{Descrizione}  & Durante la pianificazione, è probabile che vengano fatte stime sbagliate sui tempi necessari ad eseguire alcune attività; questo comporterebbe
un potenziale ritardo nella consegna e aumento dei costi. \\
					\hline
					\textbf{Probabilità di occorrenza}  & Moderato \\
					\hline
					\textbf{Effetto}  & Serio \\
					\hline
					\textbf{Prevenzione}  & Il \responsabilediprogetto\ deve monitorare il
progresso delle attività, in modo da individuare il prima possibile una sottostima dei tempi. Ogni componente del gruppo, qualora riscontrasse una sottostima dei tempi per una
delle attività a lui assegnate, deve comunicarlo tempestivamente al \responsabilediprogetto. \\
					\hline
					\textbf{Mitigazione}  & Nel stendere il piano delle attività il \responsabilediprogetto\ prevede, per ognuna di esse, un tempo di slack sufficientemente grande da permettere
che eventuali sottostime non provochino ritardi inaccettabili. \\
					\hline
					\textbf{Riscontro} & L'inesperienza dei membri del gruppo ha portato ad un lieve allungamento dei tempi previsti; tuttavia, tale rischio è stato riscontrato solo inizialmente e le attività di mitigazione adottate hanno permesso di non subire significativi ritardi rispetto a quanto pianificato. \\
					
					\noalign{\hrule height 1.5pt}
			\end{tabularx}
			\caption{Stima errata di costi e/o tempi delle attività. \label{tab:table_label}}
		\end{table}
		
		\subsection{Livello dei requisiti}
				\begin{table}[H]
				\center
				\begin{tabularx}{\textwidth}{X X}
					\noalign{\hrule height 1.5pt}
					\textbf{Nome Rischio} & Comprensione dei requisiti \\
					\hline
					\textbf{Descrizione}  & Durante l'acquisizione e analisi dei requisiti, alcuni di essi potrebbero essere fraintesi o compresi solo in parte. Inoltre il dominio del problema
potrebbe essere non capito fino in fondo. Questo può provocare divergenze fra ciò che si aspetta il committente e ciò che viene progettato dal fornitore. \\
					\hline
					\textbf{Probabilità di occorrenza}  & Moderato \\
					\hline
					\textbf{Effetto}  & Serio \\
					\hline
					\textbf{Prevenzione}  & Il \responsabilediprogetto\ organizzerà un numero sufficiente di incontri con il proponente, al fine di acquisire, raffinare e/o chiarire i requisiti necessari alla corretta progettazione del prodotto commissionato. Ad ogni revisione i documenti prodotti verranno fatti esaminare dal proponente che verificherà la piena comprensione e corretta interpretazione dei requisiti necessari. \\
					\hline
					\textbf{Mitigazione}  & Il \responsabilediprogetto\  organizzerà, il più tempestivamente possibile, un incontro con il committente al fine di risolvere il problema riscontrato con i requisiti. \\
					\noalign{\hrule height 1.5pt}
					\hline
					\textbf{Riscontro} &  \\
			\end{tabularx}
			\caption{Comprensione dei requisiti. \label{tab:table_label}}
		\end{table}

\end{document}