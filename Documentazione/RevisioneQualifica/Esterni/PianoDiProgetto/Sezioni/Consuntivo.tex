\documentclass[../PianoDiProgetto.tex]{subfiles}
\newcolumntype{X}{>{\centering\arraybackslash}X}
\begin{document}
	\section{Consuntivo}
	In questa sezione viene riportato, per ogni periodo svolto, il prospetto economico contenente le ore per ruolo impiegate per svolgere le attività pianificate, insieme alle spese effettivamente sostenute. \\
	Ogni tabella contiene le ore impiegate e le spese sostenute suddivise per ruolo, insieme alla differenza, riportata tra parentesi, rispetto a quanto si era preventivato; viene inoltre riportata la differenza dei totali, intesa come differenza tra preventivo e consuntivo nello specifico periodo.
	Tale bilancio potrà risultare:
	\begin{itemize}
		\item \textbf{Positivo}: il consuntivo è superiore al preventivo;
		\item \textbf{Pari}: preventivo e consuntivo coincidono;
		\item \textbf{Negativo}: il consuntivo è inferiore al preventivo.
	\end{itemize}

	\subsection{Analisi dei Requisiti}
	La seguente tabella riporta le ore effettivamente impiegate e le spese sostenute nel periodo di Analisi dei requisiti; inoltre viene riportata, fra parentesi, la differenza di ore tra consuntivo e preventivo. \\
	Il lavoro svolto in questo periodo è da considerarsi come studio personale, quindi questi dati riguardano le ore e spese non rendicontate.

		\subsubsection{Consuntivo di periodo}
		\begin{table}[H]
			\center
			\begin{tabularx}{\textwidth}{XXX}
				\noalign{\hrule height 1.5pt}
				\textbf{Ruolo} & \textbf{Ore} & \textbf{Costo(\euro)} \\
				\noalign{\hrule height 1.5pt}
				Responsabile & 12 (0) & 360,00 (0,00)\\
				Amministratore & 8 (-1) & 160,00 (-20,00)\\
				Analista & 51 (+7) & 1225,00 (+175,00) \\
				Progettista & 0 & 0,00 \\
				Programmatore & 0 & 0,00 \\
				Verificatore & 22 (-2) & 330,00 (-30,00) \\			
				\noalign{\hrule height 1.5pt}
				\textbf{Tot. consuntivo} & \textbf{93} & \textbf{2125,00}\\
				\textbf{Tot. preventivo} & \textbf{89} & \textbf{2000,00}\\
				\textbf{Differenza totali} & \textbf{+4} & \textbf{+125,00}\\
				\noalign{\hrule height 1.5pt}
			\end{tabularx}
			\caption{Consuntivo analisi dei requisiti. \label{tab:table_label}}
		\end{table}
	
		\subsubsection{Conclusioni}
		Per questo periodo sono state necessarie, per le attività degli Analisti, 7 ore in più rispetto a quanto si è preventivato; questo è dovuto al fatto che è si è reso necessario uno studio più approfondito del problema e dei requisiti. \\ 
		L'attuazione sistematica delle attività di verifica normate, ha permesso di risparmiare 2 ore per le attività di verifica. \\
		Il bilancio complessivo è positivo di \euro 125,00.
	
	\subsection{Analisi di dettaglio}
	La seguente tabella riporta le ore effettivamente impiegate e le spese sostenute nel periodo di Analisi di dettaglio; inoltre viene riportata, fra parentesi, la differenza di ore tra preventivo e consuntivo.\\
	Il lavoro svolto in questo periodo è da considerarsi come studio personale, quindi questi dati riguardano le ore e spese non rendicontate.
		
		\subsubsection{Consuntivo di periodo}
		\begin{table}[H]
			\center
			\begin{tabularx}{\textwidth}{XXX}
				\noalign{\hrule height 1.5pt}
				\textbf{Ruolo} & \textbf{Ore} & \textbf{Costo(\euro)} \\
				\noalign{\hrule height 1.5pt}
				Responsabile & 5 (0) & 150,00 (0,00) \\
				Amministratore & 5 (0) & 100,00 (0,00) \\
				Analista & 25 (+6) & 625,00 (+150,00) \\
				Progettista & 0 & 0,00 \\
				Programmatore & 0 & 0,00 \\
				Verificatore & 5 (0) & 75,00 (0,00) \\			
				\noalign{\hrule height 1.5pt}
				\textbf{Tot. consuntivo} & \textbf{40} & \textbf{950,00} \\
				\textbf{Tot. preventivo} & \textbf{34} & \textbf{800,00}\\
				\textbf{Differenza totali} & \textbf{+6} & \textbf{+150,00} \\
				\noalign{\hrule height 1.5pt}
			\end{tabularx}
			\caption{Consuntivo analisi di dettaglio. \label{tab:table_label}}
		\end{table}
	
		\subsubsection{Conclusioni}
		Per questo periodo sono state necessarie, per l'attività degli Analisti, 6 ore in più rispetto a quanto si è preventivato; questo è dovuto alla difficoltà degli Analisti di identificare con accuratezza e precisione i requisiti dettagliati del prodotto.  \\
		Il bilancio complessivo è positivo di \euro 150,00.
		
	\subsection{Progettazione Architetturale}
	La seguente tabella riporta le ore effettivamente impiegate e le spese sostenute nel periodo di Progettazione Architetturale; inoltre viene riportata, fra parentesi, la differenza di ore tra consuntivo e preventivo.\\
	
	\subsubsection{Consuntivo di periodo}
	\begin{table}[H]
		\center
		\begin{tabularx}{\textwidth}{XXX}
			\noalign{\hrule height 1.5pt}
			\textbf{Ruolo} & \textbf{Ore} & \textbf{Costo(\euro)} \\
			\noalign{\hrule height 1.5pt}
			Responsabile & 12 (0) & 360,00 (0,00) \\
			Amministratore & 23 (0) & 460,00 (0,00) \\
			Analista & 61 (-4) & 1525,00 (-100,00) \\
			Progettista & 52 (-4) & 1144,00 (-88,00)  \\
			Programmatore & 0 & 0,00 \\
			Verificatore & 40 (+4) & 720,00 (+60,00) \\			
			\noalign{\hrule height 1.5pt}
			\textbf{Tot. consuntivo} & \textbf{196} & \textbf{4209,00} \\
			\textbf{Tot. preventivo} & \textbf{200} & \textbf{4337,00}\\
			\textbf{Differenza totali} & \textbf{-4} & \textbf{-128,00} \\
			\noalign{\hrule height 1.5pt}
		\end{tabularx}
		\caption{Consuntivo progettazione architetturale. \label{tab:table_label}}
	\end{table}
	
	\subsubsection{Conclusioni}
	Per questo periodo sono state risparmiate 4 ore per le attività degli Analisti, dovuto ad una Analisi più dettagliata del previsto svolta nei periodi precedenti; sono state risparmiate 4 ore per le attività del Progettista, dovuto a una sovrastima nel preventivo; infine sono state necessarie 4 ore in più del preventivato per le attività del Verificatore, dovuto a una verifica più approfondita del \pianodiqualifica.
	Il bilancio complessivo è negativo di \euro 128,00.
	
	\subsubsection{Preventivo a finire}
	Da quanto riportato nel precedente consuntivo di periodo si evince la possibilità di impiegare il budget risparmiato per i periodi successivi. \\
	I \euro 128,00 risparmiati verranno utilizzati per aumentare le ore del verificatore nel periodo di Progettazione di dettaglio e codifica, al fine di migliorare la qualità dei prodotti in ingresso alla \revisionediqualifica.
	
	\subsection{Progettazione di dettaglio e Codifica}

	\subsubsection{Consuntivo di periodo}
	\begin{table}[H]
		\center
		\begin{tabularx}{\textwidth}{XXX}
			\noalign{\hrule height 1.5pt}
			\textbf{Ruolo} & \textbf{Ore} & \textbf{Costo(\euro)} \\
			\noalign{\hrule height 1.5pt}
			Responsabile &  0(0) & 0,00 (0,00) \\
			Amministratore &  0(0) & 0,00 (0,00) \\
			Analista &  0(0) & 0,00 (0,00) \\
			Progettista &  0(0) & 0,00 (0,00)  \\
			Programmatore & 0(0) & 0,00(0,00) \\
			Verificatore & 0(0) & 0,00 (0,00) \\			
			\noalign{\hrule height 1.5pt}
			\textbf{Tot. consuntivo} & \textbf{0} & \textbf{0,00} \\
			\textbf{Tot. preventivo} & \textbf{0} & \textbf{0,00}\\
			\textbf{Differenza totali} & \textbf{0} & \textbf{0,00} \\
			\noalign{\hrule height 1.5pt}
		\end{tabularx}
		\caption{Consuntivo progettazione di dettaglio e codifica. \label{tab:table_label}}
	\end{table}
	
	\subsubsection{Conclusioni}
	
	\subsubsection{Preventivo a finire}
		
\end{document}
