\documentclass[../AnalisiDeiRequisiti.tex]{subfiles}
\begin{document}
	\section*{Ordine del giorno}
		\begin{enumerate}
			\item Rinegoziazione di alcuni requisiti secondo quanto scritto nel verbale interno del 27/06/2017;
			\item Presentazione di una demo del prodotto e panoramica completa delle funzionalità già implementate.
		\end{enumerate}
		
		\section*{Riassunto dei contenuti}
		Come prima cosa il gruppo ha esposto le motivazioni secondo le quali si ritenesse opportuno eliminare alcuni requisiti. Il proponente è stato completamente d'accordo sia per quanto riguarda la fusione tra activity diagram e bubble diagram che per l'eliminazione della possibilità di creare classi parametriche.\\
		Il gruppo ha mostrato al proponente una demo del prodotto. Il proponente si è mostrato in linea generale soddisfatto dell'applicazione ma ha espresso alcune perplessità.\\
		Le prime critiche sono state rivolte alla grafica, ritenuta un po' retrò. Inoltre per come è stata progettata l'interfaccia utente ci sono delle zone di schermo inutilmente occupate da cornici o bottoni troppo grandi che rubano spazio alla canvas. Inoltre non c'è un'indicazione visiva durante alcune azioni come la selezione di un elemento nella canvas. Questo va a incidere sull'usabilità del prodotto.\\
		Il proponente ha quindi chiesto al gruppo quali funzionalità ritenessero più importanti da sviluppare entro la revisione di qualifica e quali invece venissero delegate alla revisione di accettazione.\\
		Le funzionalità ancora da sviluppare, vista l'imminente scadenza, sono tutte delegate alla RA e sono le seguenti:
		\begin{itemize}
			\item implementazione della base di dati per le bubble personalizzate;
			\item command pattern per le funzionalità undo-redo;
			\item generazione di un report degli errori in compilazione durante la generazione dello stesso;
			\item visualizzazione dei layer per grado di importanza;
			\item alla creazione di un nuovo progetto viene chiesto di salvare il corrente.
		\end{itemize}
		
		Per quanto riguarda il report degli errori e il salvataggio del progetto corrente il proponente non ha avuto nulla da aggiungere. \\
		Riguardo il database il proponente ha espresso alcune perplessità sulla scelta di MySQL e ha consigliato l'utilizzo di Arango o di Postgress. 
		Il proponente ha inoltre dichiarato di non essere interessato al command pattern per la funzionalità di undo-redo in quanto preferisce che il tempo che il gruppo avrebbe dedicato allo sviluppo di questa funzionalità (giudicata relativamente complessa) venga speso nel miglioramento delle funzionalità già presenti.
		Inoltre ha consigliato di utilizzare colori diversi per la visualizzazione dei layer.\\
	
	
	\section*{Conclusioni}
	
		Visto il riscontro positivo il gruppo ha deciso di eliminare definitivamente i requisiti in questione dal documento \analisideirequisiti.
		Per quanto riguarda le critiche ricevute il gruppo si impegna a miglorare l'interfaccia utente per renderla più conforme ai desideri del proponente.
		Come concordato vengono inoltre eliminati dall'\analisideirequisiti anche i requisiti riguardanti la funzionalità di undo-redo. 
		Infine verrà presa in considerazione la possibilità di cambiare database.
\end{document}