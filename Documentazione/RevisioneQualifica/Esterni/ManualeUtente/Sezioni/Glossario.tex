\documentclass[../ManualeUtente.tex]{subfiles}
\begin{document}
	\section{Glossario}\label{sez:Glossario}
		\begin{description}
			\item \textbf{\underline{A}}
				\item \textbf{Applicazione web} - Applicazione risiedente in uno o più server web, accessibile via
				browser.
			\item \textbf{\underline{D}}
				\item \textbf{Diagramma dei package} - Diagramma in linguaggio UML che serve a descrivere
				l'architettura di un sistema orientato agli oggetti al livello di package, definendone proprietà e
				relazioni.
				\item \textbf{Diagramma delle classi} - Diagramma in linguaggio UML che serve a descrivere
				l'architettura di un sistema orientato agli oggetti al livello di classi, definendone comportamento,
				proprietà e relazioni.
			\item \textbf{\underline{U}}
				\item \textbf{UML} - Acronimo di Unified Modeling Language (linguaggio di modellazione unificato),
				è un linguaggio di modellazione e specifica basato sul paradigma orientato agli oggetti.
		\end{description}
\end{document}
