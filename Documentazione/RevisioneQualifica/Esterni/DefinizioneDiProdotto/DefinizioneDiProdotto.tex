% Document-Author: KaleidosCode
% Document-Date: 15/06/2017 aggiornare la data
% Document-Description: Documento contenente la Definizione di prodotto per SWEDesigner

\documentclass[a4paper,12pt]{article}
\usepackage{../../../Templates/kaleidos}
\renewcommand{\abstractname}{Tabella contenuti}
\newcommand\VRule[1][\arrayrulewidth]{\vrule width #1}

\author{KaleidosCode}
\date{15/06/2017}	% aggiornare la data
\intestazioni{Definizione di Prodotto}
\pagestyle{myheadings}
    
\begin{document}
	\begin{titlepage}
		\centering Università degli Studi di Padova
		\line(1,0){350}\\
		\vspace{0.4cm}
		{\bfseries\scshape\LARGE KaleidosCode\\}
		\vspace{0.4cm}
		{\bfseries\scshape\LARGE SWEDesigner\\}
		{\scshape\Large Software per diagrammi UML\\}
		\vspace{1cm}
		{\scshape\Large \definizionediprodottoi\ \\}		% cambiare il titolo per ogni documento
		\vspace{1.4cm}
		\logo
		\vspace{1.2cm}
		\centering{\bfseries Informazioni sul documento\\}
		\line(1,0){240}\\
		% compilare i campi per ogni documento	
		\begin{tabular}{r|l}
			{\hfill \textbf{Versione}} 			& 1.0.0\\
			{\hfill \textbf{Data Redazione}} 	& 15/06/2017\\	% aggiornare la data
			{\hfill \textbf{Redazione}} 		& Bonolo Marco\\ & Pace Giulio\\ & Pezzuto Francesco\\ & Sovilla Matteo\\
			{\hfill \textbf{Verifica}} 			& Sanna Giovanni\\
			{\hfill \textbf{Approvazione}} 		& Bonato Enrico\\
			{\hfill \textbf{Uso}} 				& Esterno\\
			{\hfill \textbf{Distribuzione}} 	& \vardanega \\ & \cardin \\ & \proponente\\
		\end{tabular}\\
		\vspace{2cm}
		\texttt{kaleidos.codec6@gmail.com}
	\end{titlepage}

	\pagestyle{myfront}
	\newpage
		\subfile{DiarioModificheDDP}
	\newpage
		\tableofcontents
	\newpage
		\listoftables
	\newpage
		\listoffigures
		
	\newpage
	\pagestyle{mymain}
		\subfile{Sezioni/IntroduzioneDDP}
	\newpage
		\subfile{Sezioni/StandardDiProgetto}
	\newpage
		\subfile{Sezioni/ArchitetturaDellApplicazione}
	\newpage
		\subfile{Sezioni/SpecificaDelleComponenti}
	\newpage
		\subfile{Sezioni/DiagrammiDiSequenza}
	\label{LastPage}
\end{document}
