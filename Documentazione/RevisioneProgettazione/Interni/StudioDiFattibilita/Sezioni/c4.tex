\documentclass[../StudioDiFattibilita.tex]{subfiles}
\begin{document}
	\section{Capitolato C4}
		\subsection{Descrizione}
			L'obiettivo di questo capitolato è realizzare un'applicazione in ambiente \gl{Android} che agevoli la
			lettura alle persone affette da dislessia, grazie all'aiuto di tecnologie, fra cui
			sintesi vocale e stili di testo particolari.\\
			Per permettere la riusabilità delle componenti sviluppate e consentire quindi la realizzazione di
			altre applicazioni, viene raccomandato di dividere progettazione e implementazione in almeno due
			componenti distinte:
			\begin{itemize}
				\item Libreria delle funzionalità di sintesi vocale e informazioni per la sincronizzazione a
				partire dal testo;
				\item Applicazione che farà uso delle librerie e informazioni del punto precedente.
			\end{itemize}
			Le tipologie di applicazione da sviluppare proposte sono un lettore di e-book oppure un
			client di messaggistica.\\
			È obbligatoriamente richiesto che l'applicazione sviluppata utilizzi la sintesi vocale e
			l'evidenziamento del testo sincronizzato con la riproduzione dell'audio, con supporto per almeno un
			testo sorgente tra e-book, pdf, HTML, messaggi o semplice testo.
		\subsection{Dominio applicativo}
			Il prodotto è rivolto principalmente a persone affette da dislessia con difficoltà nella lettura.
		\subsection{Dominio tecnologico}
			Viene lasciata grande libertà sulle tecnologie da impiegare nella realizzazione, purché adeguate allo
			scopo. È obbligatorio realizzare un'applicazione per dispositivi mobili con l'utilizzo di un servizio
			di sintesi vocale. Viene invece raccomandato:
			\begin{itemize}
				\item L'utilizzo del motore di sintesi ``Flexible and Adaptive Text-To-Speech'';
				\item La realizzazione dell'applicazione per piattaforma Android.
			\end{itemize}
		\subsection{Valutazione}
			\subsubsection{Aspetti positivi}
				Gli aspetti positivi individuati dal gruppo sono:
					\begin{itemize}
						\item Il dominio applicativo è sicuramente apprezzabile visto lo scopo di aiutare persone
						in difficoltà;
						\item Lo sviluppo dell'applicazione sarebbe formativo per il team, in quanto prevede l'uso di tecnologie mai affrontate da nessuno dei componenti.
					\end{itemize}
			\subsubsection{Fattori di rischio}
				I fattori di rischio individuati dal gruppo sono:
				\begin{itemize}
					\item Il dominio tecnologico risulta quasi totalmente sconosciuto e comunque non molto
					interessante.
				\end{itemize}
			\subsection{Conclusioni}
				Vista la scarsa conoscenza del dominio tecnologico ed il vasto ambiente di sviluppo quale é
				Android, il gruppo \kaleidoscode\ ha preferito scartare questo capitolato.
\end{document}
