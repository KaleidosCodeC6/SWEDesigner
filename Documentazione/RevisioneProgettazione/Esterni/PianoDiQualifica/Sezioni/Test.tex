\documentclass[../PianoDiQualifica.tex]{subfiles} 
\begin{document} 
	\section{Test}\label{Test} 
	Sono state individuate quattro tipologie di test: 
	Sono state individuate quattro tipologie di test: 
	\begin{itemize} 
		\item \textbf{Test di unità}: servono alla verifica della correttezza degli algoritmi; 
		\item \textbf{Test di integrazione}: servono alla verifica della correttezza delle 
		componenti individuate; 
		\item \textbf{Test di sistema}: servono alla verifica del corretto funzionamento 
		dell'architettura e della soddisfazione dei requisiti descritti nell'\analisideirequisiti; 
		\item \textbf{Test di validazione}: servono per accertarsi che il prodotto sia conforme 
		con quanto concordato con il Proponente. 
	\end{itemize} 
	La classificazione ed il tracciamento dei test è definito nelle \normediprogettov. 
	\subsection{Test di validazione} 
	I test di validazione vengono effettuati con il Proponente e servono per accertarsi che il prodotto realizzato sia conforme alle attese.\\ 
	Per ogni test è descritta 
	una serie di passi che l'utente deve seguire in modo tale da effettuarlo correttamente.  
	
	
	%%%%%test progetto 
	
	
	\subsubsection{Test TV1}  
	L'utente vuole verificare che si possa creare un nuovo progetto. 
	All'utente è richiesto di: 
	\begin{itemize} 
		\item Premere sul pulsante per la creazione di un progetto; 
		\item Inserire un nome per il nuovo progetto; 
		\item Confermare la creazione del progetto.
	\end{itemize} 
	
	\subsubsection{Test TV2} 
	L'utente vuole verificare che si possa caricare un progetto precedentemente creato. 
	All'utente è richiesto di: 
	\begin{itemize} 
		\item Premere sul pulsante per l'apertura di un progetto; 
		\item Selezionare il progetto che intende caricare; 
		\item Confermare il caricamento. 
	\end{itemize}     
	
	\subsubsection{Test TV3} 
	L'utente vuole verificare che si possa salvare un progetto precedentemente creato. 
	All'utente è richiesto di: 
	\begin{itemize} 
		\item Premere sul pulsante per l'apertura di un progetto; 
		\item Effettuare alcune modifiche al progetto; 
		\item Premere il bottone per salvare il progetto. 
	\end{itemize}     
	% da decidere/implementare in adr % 
	\subsubsection{Test TV4} 
	L'utente vuole verificare che si possa salvare con nome un progetto precedentemente creato. 
	All'utente è richiesto di: 
	\begin{itemize} 
		\item Premere sul pulsante per l'apertura di un progetto; 
		\item Effettuare almeno una modifica al progetto; 
		\item Premere il pulsante per salvare con nome il progetto; 
		\item Inserire il percorso dove salvare in progetto; 
		\item Inserire il nome con cui si intende salvare il progetto; 
		\item Confermare il percorso e il nome con il bottone salva. %troppo specifico?? CULO CULO CULO !!!!!!!!!!!!!!!!!!!!!!! 
	\end{itemize}     
	
	
	%test generali 
	
	
	\subsubsection{Test TV5} 
	L'utente vuole verificare la possibilità riposizionare un elemento all'interno di un diagramma. 
	All'utente è richiesto di: 
	\begin{itemize} 
		\item Aprire un progetto precedentemente creato che contiene almeno un elemento o creare un nuovo progetto e aggiungere un nuovo elemento; 
		\item Tenere premuto col puntatore su un elemento e trascinarlo. 
	\end{itemize} 
	
	
	\subsubsection{Test TV6}
	L'utente vuole verificare la possibilità di poter annullare un'azione appena eseguita. 
	All'utente è richiesto di: 
	\begin{itemize} 
		\item Aprire un progetto precedentemente creato o crearne uno nuovo; 
		\item Effettuare almeno una modifica al progetto; 
		\item Annullare la modifica appena eseguita con l'apposito bottone o tramite la scorciatoia da tastiera Ctrl+z. 
	\end{itemize}    
	
	\subsubsection{Test TV7} 
	L'utente vuole verificare la possibilità di poter ripristinare un'azione appena annullata. 
	All'utente è richiesto di: 
	\begin{itemize} 
		\item Aprire un progetto precedentemente creato o crearne uno nuovo; 
		\item Effettuare almeno una modifica al progetto; 
		\item Annullare l'ultima modifica eseguita con l'apposito bottone o tramite la scorciatoia da tastiera Ctrl+z; 
		\item Ripristinare la modifica annullata con l'apposito bottone o tramite la scorciatoia da tastiera Ctrl+y. 
	\end{itemize}    
	
	%%%%%% CULO CULO CULO aggiungere o creare un package CULO CULO CULO %%%%%% 
	
	
	%test package 
	
	\subsubsection{Test TV8} 
	L'utente vuole verificare le funzionalità del diagramma dei package aggiungendo un nuovo package. 
	All'utente è richiesto di: 
	\begin{itemize} 
		\item Aprire un progetto precedentemente creato o crearne uno nuovo; 
		\item Aggiungere un nuovo package al diagramma dei package; 
		\item Inserire il nome e eventuali proprietà per il package nel menù sul lato; %%%proprietà termine esatto?
		\item Confermare la creazione del package. 
	\end{itemize} 
	
	\subsubsection{Test TV9} 
	L'utente vuole verificare le funzionalità del diagramma dei package modificando un package. 
	All'utente è richiesto di: 
	\begin{itemize} 
		\item Aprire un progetto precedentemente creato che contiene almeno un package o crearne uno nuovo e aggiungerci un package; 
		\item Selezionare un package dal diagramma dei package; 
		\item Modificare uno o più attributi del package selezionato tramite il menù sul lato; %%%%CULO CULO attributi? CULO CULO 
		\item Confermare la modifica del package. 
	\end{itemize} 
	
	\subsubsection{Test TV10} 
	L'utente vuole verificare le funzionalità del diagramma dei package eliminando un package. 
	All'utente è richiesto di: 
	\begin{itemize} 
		\item Aprire un progetto precedentemente creato che contiene almeno un package  o crearne uno nuovo e aggiungere un package; 
		\item Selezionare un package dal diagramma dei package; 
		\item Cancellare il package premendo il bottone apposito o con il tasto Canc; 
		\item Confermare la cancellazione del package; 
		\item Verificare che vengano eliminate anche le relazioni associate al package. 
	\end{itemize}  %%%eliminate in automatico o deve confermare CULO CULO CULO 
	
	
	\subsubsection{Test TV11} 
	L'utente vuole verificare le funzionalità del diagramma dei package aggiungendo una nuova relazione tra package. 
	All'utente è richiesto di: 
	\begin{itemize} 
		\item Aprire un progetto precedentemente creato che contenga almeno 2 package o crearne uno nuovo e aggiungere 2 package; 
		\item Aggiungere un nuova nuova relazione tra due package al diagramma dei package; 
		
		Per fare ciò all'utente è richiesto di: 
		\begin{itemize}  
			\item Selezionare un primo package; 
			\item Selezionare un altro package; 
			\item Selezionare la tipologia di relazione; 
			\item Inserire molteplicità o eventuali parametri. 
		\end{itemize} 
		\item Confermare la creazione della relazione. 
	\end{itemize} 
	
	\subsubsection{Test TV12} %%%%CULO CULO CULO serve o implementiamo solo crea,elimina 
	L'utente vuole verificare le funzionalità del diagramma dei package apportando una modifica ad una relazione. 
	All'utente è richiesto di: 
	\begin{itemize} 
		\item Aprire un progetto precedentemente creato che contenga almeno una relazione tra package o crearne uno nuovo e aggiungere 2 package e una relazione; 
		\item Selezionare una relazione dal diagramma dei package; 
		\item Apportare almeno una modifica alla relazione tramite l'apposito menù sul lato; 
		\item Confermare le modifiche alla relazione. 
	\end{itemize} %%%%%alla va bene?
	
	
	\subsubsection{Test TV13} 
	L'utente vuole verificare le funzionalità del diagramma dei package eliminando una relazione. 
	All'utente è richiesto di: 
	\begin{itemize} 
		\item Aprire un progetto precedentemente creato che contenga almeno una relazione tra package o crearne uno nuovo e aggiungere 2 package e una relazione; 
		\item Selezionare una relazione dal diagramma dei package; 
		\item Eliminare la relazione selezionata tramite l'apposito bottone o con il tasto Canc; 
		\item Confermare la cancellazione della relazione. 
	\end{itemize} 
	
	%%%%%test classi 
	
	\subsubsection{Test TV14} 
	L'utente vuole verificare le funzionalità del diagramma delle classi aggiungendo una nuova classe. 
	All'utente è richiesto di: 
	\begin{itemize} 
		\item Aprire un progetto precedentemente creato che contenga almeno un package o crearne uno nuovo e aggiungere un package;
		\item Selezionare un package e visualizzare il relativo diagramma delle classi; 
		\item Aggiungere una nuova classe al diagramma delle classi; 
		\item Inserire il nome per la classe appena creata; 
		\item Confermare la creazione della classe. 
	\end{itemize} 
	
	\subsubsection{Test TV15} 
	L'utente vuole verificare le funzionalità del diagramma delle classi apportando una modifica ad una classe. 
	All'utente è richiesto di: 
	\begin{itemize} 
		\item Aprire un progetto precedentemente creato che contenga almeno una classe o crearne uno nuovo e aggiungere una classe;
		\item Selezionare un package e visualizzare il relativo diagramma delle classi; 
		\item Selezionare una classe dal diagramma delle classi; 
		\item Apportare modifiche alla classe selezionata tramite l'apposito menù sul lato; 
		\item Confermare le modifiche apportate alla classe. 
	\end{itemize} 
	
	
	\subsubsection{Test TV16} 
	L'utente vuole verificare le funzionalità del diagramma delle classi eliminando una classe. 
	All'utente è richiesto di: 
	\begin{itemize} 
		\item Aprire un progetto precedentemente creato che contenga almeno una classe o crearne uno nuovo e aggiungere una classe;
		\item Selezionare un package al diagramma dei package; 
		\item Selezionare una classe dal diagramma delle classi; 
		\item Cancellare la classe premendo il bottone apposito o con il tasto Canc; 
		\item Confermare la cancellazione della classe; 
		\item Verificare che vengano eliminate anche le relazioni associate alla classe. %% CULO CULO CULO come prima 
	\end{itemize} 
	
	
	
	\subsubsection{Test TV17} 
	L'utente vuole verificare le funzionalità del diagramma delle classi aggiungendo una nuova interfaccia. 
	All'utente è richiesto di: 
	\begin{itemize} 
		\item Aprire un progetto precedentemente creato che contenga almeno un package o crearne uno nuovo e aggiungere un package;
		\item Selezionare un package e visualizzare il relativo diagramma delle classi; 
		\item Aggiungere una nuova interfaccia al diagramma delle classi; 
		\item Inserire il nome per l'interfaccia appena creata; 
		\item Confermare la creazione dell'interfaccia. 
	\end{itemize} 
	
	
	\subsubsection{Test TV18} 
	L'utente vuole verificare le funzionalità del diagramma delle classi apportando una modifica ad un' interfaccia. 
	All'utente è richiesto di: 
	\begin{itemize} 
		\item Aprire un progetto precedentemente creato che contenga almeno un'interfaccia o crearne uno nuovo e aggiungere un'interfaccia;
		\item Selezionare un package e visualizzare il relativo diagramma delle classi; 
		\item Selezionare un'interfaccia dal diagramma delle classi; 
		\item Apportare modifiche all'interfaccia selezionata tramite l'apposito menù sul lato; 
		\item Confermare le modifiche all'interfaccia. 
	\end{itemize} 
	
	
	\subsubsection{Test TV19} 
	L'utente vuole verificare le funzionalità del diagramma delle classi eliminando un'interfaccia. 
	All'utente è richiesto di: 
	\begin{itemize} 
		\item Aprire un progetto precedentemente creato che contenga almeno un'interfaccia o crearne uno nuovo e aggiungere un'interfaccia;
		\item Selezionare un package al diagramma dei package; 
		\item Selezionare un'interfaccia al diagramma delle classi; 
		\item Cancellare l'interfaccia premendo il bottone apposito o con il tasto Canc; 
		\item Confermare la cancellazione dell'interfaccia 
		\item Verificare che vengano eliminate anche le relazioni associate all'interfaccia. 
	\end{itemize} 
	
	
	\subsubsection{Test TV20} 
	L'utente vuole verificare le funzionalità del diagramma delle classi aggiungendo una nuova relazione tra classi o interfacce. 
	All'utente è richiesto di: 
	\begin{itemize} 
		\item Aprire un progetto precedentemente creato che contenga almeno 2 tra classi e interfacce nello stesso package o crearne uno nuovo e aggiungere 2 elementi tra classi e interfacce;
		\item Selezionare un package e visualizzare il relativo diagramma delle classi; 
		\item Aggiungere un nuova nuova relazione tra classi o interfacce;  
		All'utente è richiesto di: 
		\begin{itemize} 
			\item Selezionare una prima classe o interfaccia; 
			\item Selezionare un'altra classe o interfaccia; 
			\item Selezionare la tipologia di relazione; 
			\item Inserire molteplicità o eventuali parametri. 
		\end{itemize} 
		\item Confermare la creazione della relazione. 
	\end{itemize} 
	
	
	\subsubsection{Test TV21} 
	L'utente vuole verificare le funzionalità del diagramma delle classi apportando una modifica ad una relazione. 
	All'utente è richiesto di: 
	\begin{itemize}  
		\item Aprire un progetto precedentemente creato che contenga almeno 2 connessi con una relazione o crearne uno nuovo e aggiungere 2 elementi tra classi e interfacce e una relazione;
		\item Selezionare un package e visualizzare il relativo diagramma delle classi; 
		\item Selezionare una relazione dal diagramma delle classi; 
		\item Apportare modifiche alla relazione tramite l'apposito menù sul lato destro; 
		\item Confermare le modifiche alla relazione. 
	\end{itemize} 
	
	\subsubsection{Test TV22} 
	L'utente vuole verificare le funzionalità del diagramma delle classi eliminando una relazione. 
	All'utente è richiesto di: 
	\begin{itemize} 
		\item Aprire un progetto precedentemente creato che contenga almeno 2 connessi con una relazione o crearne uno nuovo e aggiungere 2 elementi tra classi e interfacce e una relazione; 
		\item Selezionare un package e visualizzare il relativo diagramma delle classi; 
		\item Selezionare una relazione dal diagramma delle classi; 
		\item Cancellare la relazione mediante l'apposito bottone o il tasto canc; 
		\item Confermare la cancellazione della relazione. %%%%forse inutile
	\end{itemize} 
	
	%%%%%%%%%%%%%%%%%%%% Diagramma attivita 
	
	
	\subsubsection{Test TV23} 
	L'utente vuole verificare le funzionalità del diagramma delle attività aggiungendo una nuova attività. 
	All'utente è richiesto di: 
	\begin{itemize} 
		\item Aprire un progetto precedentemente creato che contenga una classe con un metodo o crearne uno nuovo e aggiungere una classe che abbia un metodo;
		\item Selezionare un package e visualizzare il relativo diagramma delle classi; 
		\item Selezionare una classe e visualizzare il diagramma delle attività relativo a un suo metodo; %%%CULO metodo, funzione o cosa? 
		\item Aggiungere una nuova attività al diagramma delle attività; 
		\item Inserire il nome per l'attività appena creata; %%%serve il nome? o qualche altro parametro 
		\item Confermare la creazione dell'attività. 
	\end{itemize} 
	
	\subsubsection{Test TV24} 
	L'utente vuole verificare le funzionalità del diagramma delle attività apportando una modifica ad un'attività. 
	All'utente è richiesto di: 
	\begin{itemize}  
		\item Aprire un progetto precedentemente creato che contenga un'attività o crearne uno nuovo e aggiungere un'attività;
		\item Selezionare un package e visualizzare il relativo diagramma delle classi; 
		\item Selezionare una classe e visualizzare il diagramma delle attività relativo a un suo metodo; 
		\item Selezionare un'attività del diagramma delle attività 
		\item Apportare modifiche all'attività selezionata tramite il suo bubble flowchart; %% menu sul lato? non credo
		\item Confermare le modifiche apportate all'attività. 
	\end{itemize} 
	
	
	\subsubsection{Test TV25} 
	L'utente vuole verificare le funzionalità del diagramma delle classi eliminando un'attività. 
	All'utente è richiesto di: 
	\begin{itemize} 
		\item Aprire un progetto precedentemente creato che contenga un'attività o crearne uno nuovo e aggiungere un'attività;
		\item Selezionare un package al diagramma dei package; 
		\item Selezionare una classe e visualizzare il diagramma delle attività di uno dei suoi metodi; 
		\item Selezionare un'attività dal diagramma delle attività; 
		\item Cancellare l'attività premendo il bottone apposito o con il tasto Canc; 
		\item Confermare la cancellazione dell'attività; 
	\end{itemize} 
	
	
	\subsubsection{Test TV26} 
	L'utente vuole verificare le funzionalità del diagramma delle attività aggiungendo un nuovo evento temporale. 
	All'utente è richiesto di: 
	\begin{itemize} 
		\item Aprire un progetto precedentemente creato che contenga un'attività o crearne uno nuovo e aggiungere un'attività;
		\item Selezionare un package e visualizzare il relativo diagramma delle classi; 
		\item Selezionare una classe e visualizzare il diagramma delle attività relativo a un suo metodo; %%%CULO metodo, funzione o cosa? 
		\item Aggiungere un nuovo evento temporale al diagramma delle attività; 
		\item Inserire il nome e la durata per l'evento temporale appena creata; %%%serve il nome o qualche altro parametro 
		\item Confermare la creazione della classe. 
	\end{itemize} 
	
	\subsubsection{Test TV27} 
	L'utente vuole verificare le funzionalità del diagramma delle attività apportando una modifica ad un evento temporale. 
	All'utente è richiesto di: 
	\begin{itemize} 
		\item Aprire un progetto precedentemente creato che contenga un'evento temporale o crearne uno nuovo e aggiungere un'evento temporale;
		\item Selezionare un package e visualizzare il relativo diagramma delle classi; 
		\item Selezionare una classe e visualizzare il diagramma delle attività relativo a un suo metodo; 
		\item Selezionare un'evento temporale del diagramma delle attività 
		\item Modificare il nome o la durata dell'evento temporale; 
		\item Confermare le modifiche apportate all'evento temporale. 
	\end{itemize} 
	
	
	\subsubsection{Test TV28} 
	L'utente vuole verificare le funzionalità del diagramma delle classi eliminando un'evento temporale. 
	All'utente è richiesto di: 
	\begin{itemize} 
		\item Aprire un progetto precedentemente creato che contenga un'evento temporale o crearne uno nuovo e aggiungere un'evento temporale;
		\item Selezionare un package al diagramma dei package; 
		\item Selezionare una classe e visualizzare il diagramma delle attività di uno dei suoi metodi; 
		\item Selezionare un'evento temporale dal diagramma delle attività; 
		\item Cancellare l'evento temporale premendo il bottone apposito o il pulsante Canc; 
		\item Confermare la cancellazione dell'evento temporale; 
	\end{itemize} 
	
	
	
	\subsubsection{Test TV29} 
	L'utente vuole verificare le funzionalità del diagramma delle attività aggiungendo una nuova trasformazione tra pin . 
	All'utente è richiesto di: 
	\begin{itemize} 
		\item Aprire un progetto precedentemente creato che contenga una classe con un metodo o crearne uno nuovo e aggiungere una classe che abbia un metodo;
		\item Selezionare un package e visualizzare il relativo diagramma delle classi; 
		\item Selezionare una classe e visualizzare il diagramma delle attività relativo a un suo metodo; %%%CULO metodo, funzione o cosa? 
		\item Aggiungere una nuova trasformazione tra pin al diagramma delle attività; 
		\item Confermare la creazione della trasformazione tra pin.%%%%Cosa devo inserire ?? culo culo 
	\end{itemize} 
	
	\subsubsection{Test TV30} 
	L'utente vuole verificare le funzionalità del diagramma delle attività apportando una modifica ad una trasformazione tra pin. 
	All'utente è richiesto di: 
	\begin{itemize} 
		\item Aprire un progetto precedentemente creato che contenga un trasformazione di pin o crearne uno nuovo e aggiungere un trasformazione di pin;
		\item Selezionare un package e visualizzare il relativo diagramma delle classi; 
		\item Selezionare una classe e visualizzare il diagramma delle attività relativo a un suo metodo; 
		\item Selezionare una trasformazione tra pin del diagramma delle attività 
		\item Modificare la trasformazione tra pin;%%%%CULO CULO non ne ho idea 
		\item Confermare le modifiche apportate alla trasformazione tra pin. 
	\end{itemize} 
	
	
	\subsubsection{Test TV31} 
	L'utente vuole verificare le funzionalità del diagramma delle classi eliminando una trasformazione tra pin . 
	All'utente è richiesto di: 
	\begin{itemize} 
		\item Aprire un progetto precedentemente creato che contenga un trasformazione di pin o crearne uno nuovo e aggiungere un trasformazione di pin;
		\item Selezionare un package al diagramma dei package; 
		\item Selezionare una classe e visualizzare il diagramma delle attività di uno dei suoi metodi; 
		\item Selezionare una trasformazione tra pin dal diagramma delle attività; 
		\item Cancellare la trasformazione tra pin premendo il bottone apposito o il pulsante Canc; 
		\item Confermare la cancellazione della trasformazione tra pin. 
	\end{itemize} 
	
	
	
	\subsubsection{Test TV32} 
	L'utente vuole verificare le funzionalità del diagramma delle attività aggiungendo una nuova regione d'espansione. 
	All'utente è richiesto di: 
	\begin{itemize} 
		\item Aprire un progetto precedentemente creato che contenga una classe con un metodo o crearne uno nuovo e aggiungere una classe che abbia un metodo;
		\item Selezionare un package e visualizzare il relativo diagramma delle classi; 
		\item Selezionare una classe e visualizzare il diagramma delle attività relativo a un suo metodo; %%%CULO metodo, funzione o cosa? 
		\item Aggiungere una nuova  regione d'espansione al diagramma delle attività; 
		\item Confermare la creazione della regione d'espansione.%%%%Cosa devo inserire ?? culo culo 
	\end{itemize} 
	
	\subsubsection{Test TV33} 
	L'utente vuole verificare le funzionalità del diagramma delle attività apportando una modifica ad una regione d'espansione. 
	All'utente è richiesto di: 
	\begin{itemize} 
		\item Aprire un progetto precedentemente creato che contenga un una regione d'espansione o crearne uno nuovo e aggiungere una regione d'espansione;
		\item Selezionare un package e visualizzare il relativo diagramma delle classi; 
		\item Selezionare una classe e visualizzare il diagramma delle attività relativo a un suo metodo; 
		\item Selezionare una regione d'espansione del diagramma delle attività 
		\item Modificare la regione d'espansione;%%%%CULO CULO non ne ho idea 
		\item Confermare le modifiche apportate alla regione d'espansione. 
	\end{itemize} 
	
	
	\subsubsection{Test TV34} 
	L'utente vuole verificare le funzionalità del diagramma delle classi eliminando una regione d'espansione. 
	All'utente è richiesto di: 
	\begin{itemize} 
		\item Aprire un progetto precedentemente creato che contenga un una regione d'espansione o crearne uno nuovo e aggiungere una regione d'espansione;
		\item Selezionare un package al diagramma dei package; 
		\item Selezionare una classe e visualizzare il diagramma delle attività di uno dei suoi metodi; 
		\item Selezionare una regione d'espansione dal diagramma delle attività; 
		\item Cancellare la regione d'espansione premendo il bottone apposito o il pulsante Canc; 
		\item Confermare la cancellazione della regione d'espansione. 
	\end{itemize} 
	
	
	
	%%%%%%%%%%%%%%%%%%%%%%%bubble Flow chart 
	
	
	
	\subsubsection{Test TV35} 
	L'utente vuole verificare le funzionalità del bubble flowchart aggiungendo una nuova bubble. 
	All'utente è richiesto di: 
	\begin{itemize} 
		\item Aprire un progetto precedentemente creato che contenga un'attività o crearne uno nuovo e aggiungere un'attività;
		\item Selezionare un package e visualizzare il relativo diagramma delle classi; 
		\item Selezionare una classe e visualizzare il diagramma delle attività relativo a un suo metodo; %%%CULO metodo, funzione o cosa? 
		\item Selezionare un'attività e visualizzare il suo bubble flowchart; 
		\item Aggiungere una nuova bubble al diagramma delle attività; 
		\item Inserire eventuali parametri per la bubble; 
		\item Confermare la creazione della bubble.%%%%Cosa devo inserire ?? culo culo 
	\end{itemize} 
	
	\subsubsection{Test TV36} 
	L'utente vuole verificare le funzionalità del bubble flowchart apportando una modifica ad una bubble. 
	All'utente è richiesto di: 
	\begin{itemize} 
		\item Aprire un progetto precedentemente creato che contenga una bubble o crearne uno nuovo e aggiungere una bubble;
		\item Selezionare un package e visualizzare il relativo diagramma delle classi; 
		\item Selezionare una classe e visualizzare il diagramma delle attività relativo a un suo metodo; 
		\item Selezionare un'attività e visualizzare il suo bubble flowchart; 
		\item Selezionare una bubble del bubble flowchart; 
		\item Modificare i parametri della bubble tramite il menù sul lato;%%%%CULO CULO non ne ho idea 
		\item Confermare le modifiche apportate alla bubble. 
	\end{itemize} 
	
	
	\subsubsection{Test TV37} 
	L'utente vuole verificare le funzionalità del bubble flowchart eliminando una bubble. 
	All'utente è richiesto di: 
	\begin{itemize} 
		\item Aprire un progetto precedentemente creato che contenga una bubble o crearne uno nuovo e aggiungere una bubble;
		\item Selezionare un package al diagramma dei package; 
		\item Selezionare una classe e visualizzare il diagramma delle attività di uno dei suoi metodi; 
		\item Selezionare un'attività e visualizzare il suo bubble flowchart; 
		\item Selezionare una bubble dal bubble flowchart; 
		\item Cancellare la bubble premendo il bottone apposito o il tasto Canc; 
		\item Confermare la cancellazione della bubble. 
	\end{itemize} 
	
	
	\subsubsection{Test TV38} 
	L'utente vuole verificare le funzionalità del bubble flowchart aggiungendo un nuovo elemento di decisione. 
	All'utente è richiesto di: 
	\begin{itemize} 
		\item Aprire un progetto precedentemente creato che contenga un'attività o crearne uno nuovo e aggiungere un'attività;
		\item Selezionare un package e visualizzare il relativo diagramma delle classi; 
		\item Selezionare una classe e visualizzare il diagramma delle attività relativo a un suo metodo; %%%CULO metodo, funzione o cosa? 
		\item Selezionare un'attività e visualizzare il suo bubble flowchart; 
		\item Aggiungere un nuovo evento temporale al diagramma delle attività; 
		\item Inserire il nome e la durata per l'evento temporale appena creata; %%%serve il nome o qualche altro parametro 
		\item Confermare la creazione dell'elemento di decisione. 
	\end{itemize} 
	
	\subsubsection{Test TV39} 
	L'utente vuole verificare le funzionalità del bubble flowchart apportando una modifica ad un elemento di decisione;
	All'utente è richiesto di: 
	\begin{itemize} 
		\item Aprire un progetto precedentemente creato che contenga un'elemento di decisione o crearne uno nuovo e aggiungere un'elemento di decisione;
		\item Selezionare un package e visualizzare il relativo diagramma delle classi; 
		\item Selezionare una classe e visualizzare il diagramma delle attività relativo a un suo metodo; 
		\item Selezionare un'attività e visualizzare il suo bubble flowchart; 
		\item Selezionare un'evento temporale del diagramma delle attività; 
		\item Modificare almeno un parametro dell'elemento di decisione dal menù al lato; 
		\item Confermare le modifiche apportate all'elemento di decisione. 
	\end{itemize} 
	
	
	\subsubsection{Test TV40} 
	L'utente vuole verificare le funzionalità del bubble flowchart eliminando un elemento di decisione. 
	All'utente è richiesto di: 
	\begin{itemize} 
		\item Aprire un progetto precedentemente creato che contenga un'elemento di decisione o crearne uno nuovo e aggiungere un'elemento di decisione;
		\item Selezionare un package al diagramma dei package; 
		\item Selezionare una classe e visualizzare il diagramma delle attività di uno dei suoi metodi; 
		\item Selezionare un'attività e visualizzare il suo bubble flowchart; 
		\item Selezionare un'elemento di decisione dal diagramma delle attività; 
		\item Cancellare l'elemento di decisione premendo il bottone apposito o il pulsante Canc; 
		\item Confermare la cancellazione dell'elemento di decisione.
	\end{itemize} 
	
	\subsection{Test di sistema} 
	I test di sistema vengono eseguiti sul prodotto finito, cioè quando tutte le sue componenti 
	sono integrate.  
	\subsection{Test di integrazione} 
	%    \subsection{Test di unità}  DA INSERIRE IN PROGETTAZIONE DI DETTAGLIO 
\end{document} 