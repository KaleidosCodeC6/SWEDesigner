\documentclass[../PianoDiQualifica.tex]{subfiles}
\begin{document}
	\section{Introduzione}
		\subsection{Scopo del documento} 
			Questo documento definisce gli obiettivi e le metodologie che ogni membro
			del gruppo \kaleidoscode\ adotterà per garantire un determinato livello di
			qualità del prodotto.\\
			A tal proposito ogni membro del gruppo è tenuto a leggere, perseguire e
			raggiungere gli obiettivi definiti in esso.
		\subsection{Scopo del prodotto}
			Lo scopo del progetto è la realizzazione di un software di
			costruzione di diagrammi \gl{UML} con la relativa generazione
			di codice \gl{Java} e \gl{Javascript} utilizzando tecnologie
			web. Il prodotto deve essere conforme ai vincoli qualitativi
			richiesti dal committente.
		\subsection{Glossario}
			Al fine di evitare ogni ambiguità di linguaggio e massimizzare la
			comprensione dei documenti i termini tecnici, di dominio, gli
			acronimi e le parole che necessitano di essere chiarite sono
			riportate nel documento \glossariov.\\
			La prima occorrenza di ciascuno di questi vocaboli è
			marcata da una ``G'' maiuscola in pedice.
		\subsection{Riferimenti utili}
			\subsubsection{Riferimenti normativi}
    			\begin{itemize}
    				\item \textbf{\gl{Capitolato} d'appalto}:\\
    				\url{http://www.math.unipd.it/~tullio/IS-1/2016/Progetto/C6.pdf} (09/03/2017);
    				\item \textbf{Norme di progetto}: \normediprogettov.
				\end{itemize}
			\subsubsection{Riferimenti informativi}	
				\begin{itemize}
					\item \textbf{Slide dell'insegnamento di Ingegneria del Software
					1\ap{o} semestre}:
						\begin{itemize}
							\item Qualità del software:\\
							\url{http://www.math.unipd.it/~tullio/IS-1/2016/Dispense/L10.pdf} (02/04/2017);
							\item Qualità di Processo:\\
							\url{http://www.math.unipd.it/~tullio/IS-1/2016/Dispense/L11.pdf} (02/04/2017).
						\end{itemize}
					\item \textbf{Slide dell'insegnamento di Ingegneria del Software
					2\ap{o} semestre}:
						\begin{itemize}
							\item Metodi e obiettivi di quantificazione:\\
							\url{http://www.math.unipd.it/~tullio/IS-1/2016/Dispense/L15.pdf} (02/04/2017).
						\end{itemize}
					\item \textbf{\gl{ISO} 9001}:\\
					\url{https://it.wikipedia.org/wiki/Norme_della_serie_ISO_9000#ISO_9001} (02/04/2017);
					\item \textbf{ISO 9126}: \url{https://it.wikipedia.org/wiki/ISO/IEC_9126} (02/04/2017);
					\item \textbf{ISO 12207}: \url{https://en.wikipedia.org/wiki/ISO/IEC_12207} (02/04/2017);
					\item \textbf{Indice \gl{Gulpease}}:\\
					\url{https://it.wikipedia.org/wiki/Indice_Gulpease} (02/04/2017);
					\item \textbf{Complessità ciclomatica}:\\
					\url{https://en.wikipedia.org/wiki/Cyclomatic_complexity} (02/04/2017);
					\item \textbf{Capability Maturity Model (\gl{CMM})}:\\
					\url{https://en.wikipedia.org/wiki/Capability_Maturity_Model} (02/04/2017);
					\item \textbf{Analisi dei requisiti}: \analisideirequisitiv;
					\item \textbf{Piano di progetto}: \pianodiprogettov;
					\item \textbf{Glossario}: \glossariov.
				\end{itemize}
\end{document}
